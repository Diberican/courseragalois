%% -*- coding:utf-8 -*-
\chapter{Solvability by radicals, Abel's theorem. A few words on
  relation to representations and topology}

We finally arrive to the source of Galois theory, the question which
motivated Galois himself: which equation are solvable by radicals and
which are not? We explain Galois' result: an equation is solvable by
radicals if and only if its Galois group is solvable in the sense of
group theory. In particular we see that the "general" equation of
degree at least 5 is not solvable by radicals. We briefly discuss the
relations to representation theory and to topological coverings.

\section{Extensions solvable by radicals. Solvable groups. Example} 

\subsection{Extensions solvable by radicals}

Let $K$ is a field of characteristic 0: $char K = 0$. It is embedded
into its \nameref{def:algebraicclosure}.

\begin{definition}[Extension solvable by radicals]
  A finite extension $E$ of $K$ is solvable by radicals if
  $\exists \alpha_1, \dots, \alpha_r$ generating $E$ such that
  $\alpha_i^{n_i} \in K\left(\alpha_1, \dots, \alpha_{i-1}\right)$ for
  some $n_i \in \mathbb{N}$.
  \label{def:solvableextension}
\end{definition}

\begin{example}
  Let $K = \mathbb{Q}$, $E = \mathbb{Q}\left(\sqrt[3]{2 + 3
    \sqrt{7}}, \sqrt[5]{4 + 5 \sqrt{11}}\right)$. We have
    $\alpha_1 = \sqrt{7}, \alpha_2 = \sqrt{11},
    \alpha_3 = \sqrt[3]{2 + 3\sqrt{7}}, \alpha_4 = \sqrt[5]{4 + 5
      \sqrt{11}}$. 
\end{example}

\begin{definition}[Polynomial solvable by radicals]
  $P \in K\left[X\right]$ is called solvable by radicals if exists a
  $E$ - \nameref{def:solvableextension} and containing all roots of
  $P$. 
  \label{def:solvablepolynomial}
\end{definition}
So more precisely, it would say that the equation, $P = 0$ is solvable
by radicals.

\begin{property}
  \begin{enumerate}
  \item \nameref{def:compositeextension} of solvable by radicals is itself
    solvable by radicals
  \item If $L$ extension of $K$ is solvable by radicals (by definition
    $L$ should be finite extension of $K$) then exists a finite
    \nameref{def:galoisextension} $E$ containing $L$ and solvable by
    radicals.    
  \end{enumerate}
  \begin{proof}
    For the first property (the proof is missing in the lectures): let
    $L = L_1 L_2$ where $L_1$ and $L_2$ are solvable i.e.
    $L_1 = K\left(\alpha_1, \dots, \alpha_n\right),
    L_2 = K\left(\beta_1, \dots, \beta_m\right)$ with $\alpha_i,
    \beta_j$ which satisfies properties from definition
    \ref{def:solvableextension}. In the case we can assume
    $L = L_1 L_2 = K\left(L_1 \cup L_2\right) = 
    K\left(\alpha_1, \dots, \alpha_n, \beta_1, \dots, \beta_m\right)$
    and all properties from definition
    \ref{def:solvableextension} will also be satisfied. Therefore
    the composite extension $L = L_1 L_2$ is also solvable.
    
    For the second property: Indeed take a composite of all images of $L$ in
    $\bar{K}$ or, in other words, images of $L$ by
    $Gal\left(\bar{K}/K\right)$.
    \footnote{
      ??? First claim. If $g \in Gal\left(\bar{K}/K\right)$ and $L$ is
      solvable then $L_g = g(L)$ i.e. image of $L$ is also
      solvable. Lets prove it by induction if $L =
      K\left(\alpha\right)$ then $\exists n_1 \in \mathbb{N}$ such that
      $\alpha^{n_1} = a \in K$ therefore $\alpha$ is a root of the
      following polynomial $X^{n_1} - a$. We have, as soon as $g$ permutes the
      roots of the polynomial, $g\left(K\left(\alpha\right)\right) =
      K\left(\beta\right)$ where $\beta = g(\alpha)$ and $\beta$ is a
      root of $X^{n_1} -a$ i.e. $\beta^{n_1} \in K$ or, in other words,
      $g\left(K\left(\alpha\right)\right)$ is also solvable by
      radicals. Using induction hypothesis we have that if
      $K\left(\alpha_1, \dots, \alpha_i\right)$ is solvable then
      $g\left(K\left(\alpha_1, \dots, \alpha_i\right)\right)$ is also
      solvable ???
    }
  \end{proof}
  \label{property:solvable}
\end{property}

\subsection{Solvable groups}
This shall be a brief reminder since this is not a course on group
theory, you are supposed to know some group theory already. So I
somehow I presume that you are familiar with this definition but I
will recall the definition of basic properties.

\begin{definition}[Solvable group]
  $G$ is called solvable if it has a filtration 
  i. e. $G = G_0 \supset G_1 \supset \dots \supset G_{r-1} \supset G_r
  = \left\{e\right\}$, such that $G_i$ is a
  \nameref{def:normalsubgroup} of 
  $G_{i-1}$ and the \nameref{def:quotientgroup} $G_{i-1}/G_i$ is
  an \nameref{def:abeliangroup}.
  \label{def:solvablegroup}
\end{definition}

\begin{example}[Group of permutations $S_3$]
  Consider $S_3$ - the group of permutations (see also example
  \ref{ex:s3group}). It's solvable because 
  $S_3 \supset A_3 \supset \left\{e\right\}$.

  We have $\left|S_3/A_3\right| = 2$ (see example
  \ref{ex:s3a3quotientgroup}) i.e. $S_3/A_3$ is cyclic of order 
  2. $\left|A_3\right| = 3$ i.e. $A_3$ - cyclic of order 3.
  \footnote{
    As it was mentioned at \cite{wiki:finitegroup} there is only one
    group of order 3. We also have (with theorem
    \ref{thm:cyclic_group_is_abelian}) that $A_3$ is
    \nameref{def:abeliangroup} 
  }
  \label{ex:lec8_s3}
\end{example}

\begin{example}[Group of permutations $S_4$]
  Consider $S_4$ - the group of permutations (see also example
  \ref{ex:s3group}). It's solvable because 
  $S_4 \supset A_4 \supset K \supset \left\{e\right\}$, where $K$ -
  is a subgroup
  \footnote{
    There is a well known Klein four group $V_4$ \cite{wiki:klein4group} -
    the only non-cyclic group of order 4. It also denoted as $K_4$. We
    have already seen it at example \ref{ex:lec7_cyclotomic8}.
  }
  of double transpositions
  (see example
  \ref{ex:permutation} for permutation cycles notation):
  \[
  K = \left\{
  e, (12)(34), (13)(24), (14)(23)
  \right\}.
  \]
  A double transposition is a product of two transpositions with
  distinct support, right, which permute the distinct elements.
  $A_4 \triangleleft S_4$, $\left|S_4/A_4\right| = 2$, i.e.
  $S_4/A_4$ is cyclic of order 2.
  \footnote{
    i.e. theorem
    \ref{thm:cyclic_group_is_abelian} gives us that $S_4/A_4$ is
    \nameref{def:abeliangroup}
  }

  $K \triangleleft A_4$, $\left|A_4/K\right| = 3$, i.e.
  $A_4/K$ is cyclic of order 3.
  \footnote{
    i.e. theorem
    \ref{thm:cyclic_group_is_abelian} gives us that $A_4/K$ is
    \nameref{def:abeliangroup}
  }
  
  $K$ is \nameref{def:abeliangroup} and
  $K \cong \mathbb{Z}/2\mathbb{Z} \times \mathbb{Z}/2 \mathbb{Z}$.

  So this shows that $S_4$ is solvable.
  \label{ex:lec8_s4}
\end{example}

\section{Properties of solvable groups. Symmetric group}

\begin{property}
  If $G$ is solvable and $H \subset G$ is a subgroup of $G$ then $H$
  is solvable.
  \begin{proof}
    Indeed $G_i \cap H$ gives a filtration with required property.
    \footnote{
      We have $H_i = G_i \cap H$ and as soon as
      $G_i \triangleleft G_{i-1}$ we also get
      $H_i \triangleleft H_{i-1}$. Really $\forall h_i \in H_i$ we
      have $h_i \in G_i \cap H \subset G_i$. Thus, using normality
      $G_i \triangleleft G_{i-1}$,
      \[
      h_i G_{i-1} = G_{i-1} h_i.
      \]
      The last equation also holds for any subset of $G_{i-1}$ and
      especially for $H_i = G_i \cap H$. As result we can conclude
      $H_i \triangleleft H_{i-1}$.

      We have that $G_{i-1}/G_i$ is an
      \nameref{def:abeliangroup} and now we have to prove that the
      \nameref{def:quotientgroup} 
      $H_{i-1}/H_i$ is abelian.
    }
  \end{proof}
  \label{property:lec8_solvable1}
\end{property}

\begin{property}
  If $G$ is solvable and $H \triangleleft G$ is a normal subgroup of
  $G$ then $G/H$ is solvable.
  \begin{proof}
    Indeed consider a projection map
    \begin{equation}
      \pi: G \to G/H
      \label{eq:lec8_solvable_pi}
    \end{equation}
    then $\pi\left(G_i\right)$ gives a filtration $\left(G/H\right)_i$
    on $G/H$ with required properties.
    \footnote{
      ???
    }
  \end{proof}
  \label{property:lec8_solvable2}
\end{property}

\begin{property}
  If $H \triangleleft G$, $H$ and $G/H$ are solvable then $G$ is
  solvable. 
  \begin{proof}
    Put togeter the filtration $H_i$ and
    $\pi^{-1}\left(\left(G/H\right)_j\right)$ (see
    (\ref{eq:lec8_solvable_pi}) for $\pi$ definition).
    \footnote{
      ???
    }
  \end{proof}
  \label{property:lec8_solvable3}
\end{property}

\begin{property}
  If $G$ is finite than $G$ is solvable (i.e. has a finite filtration with
  Abelian quotients) if and only if there exists a
  finite filtration with cyclic quotients.
  \begin{proof}
    This is just because a finite \nameref{def:abeliangroup} is just a
    product of cyclic groups.
    \footnote{
      ???
    }
  \end{proof}
  \label{property:lec8_solvable4}
\end{property}

Lets also look at another definition of solvable group
\begin{definition}[Solvable group]
  $G$ is called solvable if the following sequence is finite:
  \[
  G
  \supseteq \left[G, G\right] = G^{(1)}
  \supseteq \left[G^{(1)}, G^{(1)}\right] = G^{(2)}
  \supseteq \dots \supseteq
  \left[G^{(n-1)}, G^{(n-1)}\right] = G^{(n)} = \left\{e\right\}
  \]
  where $G^{(i)} = \left[G^{(i-1)}, G^{(i-1)}\right]$ is the
  \nameref{def:commutatorsubgroup}.
  \label{def:solvablegroupadd}
\end{definition}

\begin{remark}
  Definitions of solvable group \ref{def:solvablegroupadd} and
  \ref{def:solvablegroup} are equivalent.
  \begin{proof}
    Our filtration with \nameref{def:commutatorsubgroup}s
    $G \supseteq G^{(1)} \supseteq \dots \supseteq G^{(n)} =
    \left\{e\right\}$ is a filtration with abelian quotient because
    $G^{(i)}/\left[G^{(i)}, G^{(i)}\right] = G^{(i)}/G^{(i+1)}$ is an
    \nameref{def:abeliangroup}.

    From the other hand if $G/H$ is an \nameref{def:abeliangroup} then
    $H \supset \left[G, G\right]$. So if a finite filtration with
    abelian quotient exists then the filtration given by $G^{(i)}$ is
    also finite. It must terminate after a finite steps. So, this
    proves the equivalence.  
  \end{proof}
  \label{rem:lec8_solvable}
\end{remark}

\begin{theorem}[$S_n$ solvability]
  $S_n$ - the permutation of $n$ elements (see example
  \ref{ex:sngroup}) is not solvable for $n \ge 5$. 
  \begin{proof}
    It's easy to use definition \ref{def:solvablegroupadd}.
    Main steps are the following
    \begin{enumerate}
      \item we know that $\left[S_n, S_n\right] = A_n$ - subgroup of
        even permutations (see definition
        \ref{def:alternatinggroup}). It can be see from the fact that
        any 3-cycle is a \nameref{def:commutatorsubgroup} and 3-cycles
        generate $A_n$
        \footnote{
          3-cycle are even permutations and result of
          their compositions is also even
        }
      \item If $n \ge 5$ then $\left[A_n, A_n\right] = A_n$ thus the
        filtration generated by commutators will never terminate
        i.e. will never reach the unity ($\left\{e\right\}$) and will
        stabilize on $A_n$. How we can see it? We can remember that
        $\left[A_4, A_4\right] = K$ (see example \ref{ex:lec8_s4}) -
        the subgroup of double transpositions. $A_4 \hookrightarrow
        A_n$ in many different ways. Because you can pick any 4
        elements, our $n$ elements and just consider the permutations of
        those 4 elements as a subgroup of permutations of $n$
        elements and then taking the commutators of those $A_4$, we
        see that all double transpositions are in the $\left[A_n,
          A_n\right]$ (\nameref{def:commutatorsubgroup} of $A_n$). But
        if $n \ge 5$, they generate $A_n$.          
    \end{enumerate}
  \end{proof}
  \label{thm:lec8_sn_solvability}
\end{theorem}

\section{Galois theorem on solvability by radicals}

\begin{theorem}
  Let $P \in K\left[X\right]$. $P$ is a
  \nameref{def:solvablepolynomial} if and only if
  $Gal\left(P\right)$ is solvable. There
  $Gal\left(P\right)$ is (by definition) $Gal\left(F/K\right)$ where
  $F$ is a \nameref{def:splittingfield} of $P$ over $K$.
  \begin{proof}
    First of all lets proof that if $Gal\left(P\right)$ is solvable
    then $P$ is solvable. Let $n - \left[F:K\right]$ and consider
    $L = K\left(\zeta_n\right)$ where $\zeta_n$ - $n$-th root of 1.
    Let $M = FL$ - a \nameref{def:compositeextension}. So this is the
    splitting and field of $P$ of which we have adjoined all the
    $n$-th roots of unity.  Then $M$ is a
    \nameref{def:galoisextension} and
    $Gal\left(M/K\right) \hookrightarrow Gal\left(F/K\right)$.
    $\forall g \in Gal\left(M/K\right)$ leaves $F$ invariant.
    If $g\mid_F = id$ then $g = id$. Then the image in fact of this
    map is of the Galois group of $F$ over the intersection of $F$ and
    $L$.  So $G = Gal\left(M/K\right)$ is solvable i.e.
    \[
    G = G_0 \supset G_1 \supset \dots \supset G_r = \left\{e\right\}
    \]
    and $G_i/G_{i+1}$ - cyclic of order $n_i \mid n$. And as soon as
    $n_i \mid n$ (very important), all $n$-th roots of 1 are in $M$
    (this is why we adjoin the $L$).

    Let $M_i = M^{G_i}$. We know $M_i \hookrightarrow M_{i+1}$ is a
    cyclic Galois extension of order $n_i \mid n$ and roots of 1 are
    in it therefore there is Kummer extension (see section
    \ref{sec:kummerextension}). So
    $M_{i+1} = M_i\left(\sqrt[n_i]{a_i}\right)$ (see proposition
    \ref{prop:lec7_2}).
    So $M = K\left(\zeta_n, \alpha_1, \dots, \alpha_r\right)$ where
    $\alpha_i = \sqrt[n_i]{a_i}$. Therefore $M$ is solvable by
    radicals.

    For another direction: if $P$ is solvable then $G$ is
    solvable. Let $E$ is solvable extension containing $F$. We may
    suppose that this is Galois. Then write
    $E = K\left(\alpha_1, \dots, \alpha_r\right)$ where
    $\alpha_i^{n_i} \in K\left(\alpha_1, \dots,
    \alpha_{i-1}\right)$. Then let $L = K\left(\zeta_n\right)$
    where $n = LCM\left(\left\{n_i\right\}\right)$ so
    $\forall n_i: n_i \mid n$. And
    take $M = LE$. We have
    $K\left(\alpha_1, \dots, \alpha_{i-1}\right) \hookrightarrow
    K\left(\alpha_1, \dots, \alpha_i\right)$ - cyclic extension of
    order $n_i$. We have $Gal\left(M/L\right)$ is solvable by this
    cyclic subgroups. $Gal\left(M/K\right)$ is also solvable since
    $Gal\left(M/L\right)$ subgroup and the quotient
    $\cong Gal\left(L/K\right)$ which is abelian.
    $Gal\left(F/K\right)$ is a quotient of $Gal\left(M/K\right)$ thus
    is solvable too.    
  \end{proof}
  \label{thm:lec8_1}
\end{theorem}

\section{Examples of equations not solvable by radicals."General
  equation"}
As we can see there exist equations which are not solvable in
radicals.

\begin{example}[Not solvable polynomial of degree 5]
  Let $P \in \mathbb{Q}\left[X\right]$ is an irreducible polynomial
  with rational 
  coefficients of degree 5. It has 3 real roots (and 2 complex
  conjugate roots) as it shown on the picture.
  
  \begin{tikzpicture}
    \draw[->] (-1.5,0) -- (3,0) node[right] {$x$};
    \draw[->] (0,-1.5) -- (0,1.5) node[above] {$y$};
    \draw[scale=0.5,domain=-2.5:4.2,smooth,variable=\x,blue] plot
    ({\x},{0.01*(\x-1)*(\x-1)*(\x-1)*(\x-1)*(\x-1) - 0.5*(\x-1) + 0.01});
  \end{tikzpicture}

  We claim that $Gal\left(P\right) = S_5$. This is because
  \begin{enumerate}
    \item $Gal\left(P\right)$ contains the complex conjugation (we
      have 2 complex conjugated roots but \nameref{def:galoisgroup} is
      the group of automorphisms which exchange roots and the complex
      conjugation will exchange the 2 complex roots).
      The complex conjugation is the transposition of roots
    \item As soon as $P$ is irreducible then $Gal\left(P\right)$
      should act transitively (see definition \ref{def:transitive}) on
      roots. We have an irreducible polynomial. We can always send
      one of its roots to another of its roots. We have this
      isomorphism of stem fields which extends to an automorphism of
      the splitting field. But, what is the subgroup of $S_5$, which adds
      transitively?

      $Gal\left(P\right) \subset S_5$ acts transitively. This means
      that $5 \mid \left|Gal\left(P\right)\right|$. That is because
      (see \nameref{thm:orbitstabilizertheorem})
      $\left|G\right| = \left|G\left(x\right)\right| \left|G_x\right|$
      ($G\left(x\right)$ - is the \nameref{def:orbit},
      $G_x$ - \nameref{def:stabilizersubgroup})
      but the orbit has 5 elements and therefore 5 divides
      the cardinality of $G$. This means, by \nameref{cor:sylow} 
      theorems, that our group contains something of order 5. But only
      5-cycle has order 5. But a 5-cycle and transposition generate
      $S_5$. So $Gal\left(P\right) = S_5$.      
  \end{enumerate}
  In fact, the same argument is valid for $S_p$ with every $p$ - prime. 
  I.e. applies to an arbitrary prime number $p$ instead of 5.

  So $Gal\left(P\right) = S_5$ - not solvable and therefore $P$ is not
  solvable by radicals.
  \label{ex:lec8_notsolvable1}
\end{example}

\begin{example}[General equation of degree $n$]
  What's the general equation. It is the following
  \[
  X^n - T_1 X^{n-1} + T_2 X^{n-2} + \dots + \left(-1\right)^n T_n,
  \]
  where $T_i$ is a variable. Where does it come from? Let
  $X_1, \dots, X_n$ are roots of a polynomial of degree $n$ when the
  polynomial itself is
  \begin{eqnarray}
  \left(X - X_1\right) \cdot \dots \cdot \left(X - X_n\right) =
  X^n - \left(\sum_i X_i \right) X^{n-1} +
  \nonumber \\
  +
  \left(\sum_{i,j} X_i X_j \right) X^{n-2} + \dots +
  \left(-1\right)^n \prod_i X_i,
  \nonumber
  \end{eqnarray}
  i.e.
  $T_1 = \sum_i X_i, T_2 = \sum_{i,j} X_i X_j, \dots, T_n = \prod_i
  X_i$.

  One has $K\left[T_1, \dots, T_n\right] \subset K\left[X_1, \dots,
    X_n\right]$ (multi-variable polynomial rings). We have the same
  also for field extensions:
  $K\left(T_1, \dots, T_n\right) \subset K\left(X_1, \dots,
  X_n\right)$. The $K\left(X_1, \dots, X_n\right)$ is algebraic and a
  splitting field for our general polynomial. So it has degree at most
  $n!$. On the other hand
  $K\left(T_1, \dots, T_n\right) \subset K\left(X_1, \dots,
  X_n\right)^{S_n}$ so degree of the extension is $n!$ and
  \[
  K\left(T_1, \dots, T_n\right) = K\left(X_1, \dots,
  X_n\right)^{S_n}.
  \]
  In particular the Galois group is $S_n$ and our
  general polynomial is not solvable by radicals if $n \ge 5$.
  This is known as Abel theorem
  \label{ex:lec8_generalequation}
\end{example}

\section{Galois action as a representation. Normal base theorem}

Connection with group representations.
\begin{definition}[Group representation]
  Let $G$ is a finite group. $V$ is a \nameref{def:vectorspace} over
  $K$. Representation of $G$ is a \nameref{def:homomorphism}
  $\rho: G \to GL\left(V\right)$
  (where $GL\left(V\right)$ is the \nameref{def:glv} i.e. the group of
  \nameref{def:automorphism}s of the vector space $V$).
  \label{def:grouprepresentation}
\end{definition}

If $L$ is a finite extension of $K$ we can talk about it as about
$K$-vector space. So we have a representation of $G$ as
\nameref{def:galoisgroup} $Gal\left(L/K\right)$:
$\rho: G \to GL_K\left(L\right)$ - this is something that we have as
the definition because we define the Galois group as the group of
automorphisms of $L$ over $K$.

We can ask the question: what's kind of representation is the
$\rho$. We claim that $\rho$ is something that is called as
\nameref{def:regularrepresentation}.
\begin{definition}[Regular representation]
  Let a vector space $V$ has a basis indexed by elements of group $G$:
  $e_g$ where $g \in G$. $\rho_{reg}\left(h\right)$ acts by
  permutations:
  \[
  \rho_{reg}\left(h\right) e_g = e_{hg}.
  \]
  \label{def:regularrepresentation}
\end{definition}

We claim that the representation of Galois group is the regular
representation. We have seen that (see proof of the theorem
\ref{thm:primitiveelement}) 
\[
L \otimes_K \bar{K} \cong \bar{K}^n.
\]
The sum $\bar{K}^n$ of $n$ ($n = \left|G = Gal\left(L/K\right)\right|$)
copies of $\bar{K}$ is indexed by the embeddings of $L$ into
$\bar{K}$. Pick one $j: L \hookrightarrow \bar{K}$ and all others can
be obtained by group \nameref{def:action} $j \circ g, g \in G$. So
$\bar{K}^n$ has a basis indexed by $G$ and the \nameref{def:action} of
$G$ permutes the basis vectors. So $L \otimes_K \bar{K} \cong
\bar{K}^n \cong$ \nameref{def:regularrepresentation} of $G$ over
$\bar{K}$. In particular $\exists x$ such that $gx \mid_{g \in G}$
form a basis of $L \otimes_K \bar{K}$ over $\bar{K}$.

Elements of $G$ are linearly independent in in the space of
\nameref{def:endomorphism}s
$End_{\bar{K}}\left(L \otimes_K \bar{K}\right)$

\begin{theorem}[Normal base]
  $\exists x \in L$ such that
  $\left\{gx \mid g \in G\right\}$ is a $K$ basis of $L$.
  \begin{proof}
    First of all consider a case when $K$ is infinite. Let pick some
    basis $e_1, \dots, e_n$ - $K$-basis in $L$. $g_1, \dots, g_n \in
    G$. Let $x \in L$ then $g_1\left(x\right), \dots,
    g_n\left(x\right)$ is a basis if and only if matrix formed by
    $g_i\left(x\right)$ in the basis $e_j$ has non zero
    determinant. But this determinant is a polynomial in the
    coefficient, which is not identically zero. Well, why? Because if
    it was identically zero, it would remain identically zero also
    after the base changed to $\bar{K}$. Since it has a $\bar{K}$
    point where it does not vanish.

    There are many $x \in L \otimes_K \bar{K}$ such that
    $g_i\left(x\right)$ form a basis. And over an infinite field, a
    polynomial which is not identically zero cannot vanish
    identically. And over an infinite field, only a polynomial which
    is identically 0 can vanish at every point.

    Let me to clarify the point: $P \in K\left[X\right]$ has at most
    $\deg P$ roots. So if $K$ infinite and $P$ has every element of
    $K$ as a root then $P = 0$ ($P$ is zero as an element of
    $K\left[X\right]$).

    By induction we can get the same statement for a polynomial in
    several variables. so, our polynomial which is the determinant of the matrix, 
    is non zero, as a polynomial of several variable because it 
    has non rules over algebraic closure. 
    And so, it also has to have rules, well non rules over $K$. So, there
    exists a point $x \in L$ (not anymore in $L \otimes_K \bar{K}$)
    such that $det(\dots) \ne 0$ at $x$ so $g_i\left(x\right)$ form a
    basis.

    If $K$ is finite then the argument with roots of a polynomial does
    not apply any more. But in the case \nameref{def:galoisgroup}s are
    cyclic i.e. $G = \left<\sigma\right>$. We have $id, \sigma, \dots,
    \sigma^{n-1}$ are linearly independent since this is the case over
    $\bar{K}$. Then the minimal polynomial of $\sigma$ as an
    \nameref{def:endomorphism} of $L$ over $K$ is $X^n - 1$. Thus
    \[
    L \cong K\left[X\right]/\left(X^n - 1\right)
    \]
    as a $K$-module with $X$ acting by $\sigma$. This is a cyclic
    module and any generator $x$ shall do i.e. $x, \sigma x, \dots,
    \sigma^{n-1} x$ form a basis.
  \end{proof}
  \label{thm:normalbase}
\end{theorem}

\section{Relation with coverings}

\begin{remark}
  If $L$ is a finite \nameref{def:galoisextension} of $K$ then
  $ L \otimes_K L$ is a direct sum of fields
  \footnote{
    see also definition \ref{def:directsummodules}
  }
  which are
  isomorphic to $L$. Sums are permuted by $G = Gal\left(L/k\right)$.
  \begin{proof}
  So if $L = K\left(\alpha\right)$ is a splitting field of the
  polynomial $P = \left(X - \alpha_1\right) \cdot \dots \cdot \left(X
  - \alpha_n\right)$ (where $\alpha \in \{\alpha_1, \dots,
  \alpha_n\}$) that is isomorphic to $K\left[X\right]/(P)$. If 
  we tensor it to $L$ we will get
  \[
  L\left[X\right]/\left(X - \alpha_1\right) \cdot \dots \cdot \left(X
  - \alpha_n\right) \cong
  L\left[X\right]/\left(X - \alpha_1\right) \times
  \dots \times
  L\left[X\right]/\left(X - \alpha_n\right)
  \]
  that is a product of copies of $L$ permuted by Galois action
  \end{proof}
  \label{rem:lec8_1}
\end{remark}

In topology one has Galois covering $Y \to X$. $G$ acts on $Y$, $X$
quotient. The covering is characterized by the property that $Y
\times_X Y = \sqcup_{g \in G} Y_g$, $Y_g = \{\left(y, gy\right)\}$.
\footnote{
  ??? add an explanation
}
