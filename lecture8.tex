%% -*- coding:utf-8 -*-
\chapter{Solvability by radicals, Abel's theorem. A few words on
  relation to representations and topology}

We finally arrive to the source of Galois theory, the question which
motivated Galois himself: which equation are solvable by radicals and
which are not? We explain Galois' result: an equation is solvable by
radicals if and only if its Galois group is solvable in the sense of
group theory. In particular we see that the "general" equation of
degree at least 5 is not solvable by radicals. We briefly discuss the
relations to representation theory and to topological coverings.

\section{Extensions solvable by radicals. Solvable groups. Example} 

\subsection{Extensions solvable by radicals}

Let $K$ is a field of characteristic 0: $char K = 0$. It is embedded
into its \nameref{def:algebraicclosure}.

\begin{definition}[Extension solvable by radicals]
  A finite extension $E$ of $K$ is solvable by radicals if
  $\exists \alpha_1, \dots, \alpha_r$ generating $E$ such that
  $\alpha_i^{n_i} \in K\left(\alpha_1, \dots, \alpha_{i-1}\right)$ for
  some $n_i \in \mathbb{N}$.
  \label{def:solvableextension}
\end{definition}

\begin{example}
  Let $K = \mathbb{Q}$, $E = \mathbb{Q}\left(\sqrt[3]{2 + 3
    \sqrt{7}}, \sqrt[5]{4 + 5 \sqrt{11}}\right)$. We have
    $\alpha_1 = \sqrt{7}, \alpha_2 = \sqrt{11},
    \alpha_3 = \sqrt[3]{2 + 3\sqrt{7}}, \alpha_4 = \sqrt[5]{4 + 5
      \sqrt{11}}$. 
\end{example}

\begin{definition}[Polynomial solvable by radicals]
  $P \in K\left[X\right]$ is called solvable by radicals if exists a
  $E$ - \nameref{def:solvableextension} and containing all roots of
  $P$. 
  \label{def:solvablepolynomial}
\end{definition}
So more precisely, it would say that the equation, $P = 0$ is solvable
by radicals.

\begin{property}
  \begin{enumerate}
  \item \nameref{def:compositeextension} of solvable by radicals is itself
    solvable by radicals
  \item If $L$ extension of $K$ is solvable by radicals (by definition
    $L$ should be finite extension of $K$) then exists a finite
    \nameref{def:galoisextension} $E$ containing $L$ and solvable by
    radicals.    
  \end{enumerate}
  \begin{proof}
    For the first property: ???
    
    For the second property: Indeed take a composite of all images of $L$ in
    $\bar{K}$. Or those are the same as images of $L$ by
    $Gal\left(\bar{K}/K\right)$ 
  \end{proof}
  \label{property:solvable}
\end{property}

\subsection{Solvable groups}
This shall be a brief reminder since this is not a course on group
theory, you are supposed to know some group theory already. So I
somehow I presume that you are familiar with this definition but I
will recall the definition of basic properties.

\begin{definition}[Solvable group]
  $G$ is called solvable if it has a filtration 
  i. e. $G = G_0 \supset G_1 \supset \dots \supset G_{r-1} \supset G_r
  = \left\{e\right\}$, such that $G_i$ is a normal subgroup of
  $G_{i-1}$ and the \nameref{def:quotientgroup} $G_{i-1}/G_i$ is abelian.
  \label{def:solvablegroup}
\end{definition}

\begin{example}[Group of permutations $S_3$]
  Consider $S_3$ - the group of permutations (see also example
  \ref{ex:s3group}). It's solvable because 
  $S_3 \supset A_3 \supset \left\{e\right\}$.

  We have $\left|S_3/A_3\right| = 2$ (see example
  \ref{ex:s3a3quotientgroup}) i.e. $S_3/A_3$ is cyclic of order 
  2. $\left|A_3\right|$ i.e. $A_3$ - cyclic of order 3.
  \label{ex:lec8_s3}
\end{example}

\begin{example}[Group of permutations $S_4$]
  Consider $S_4$ - the group of permutations (see also example
  \ref{ex:s3group}). It's solvable because 
  $S_4 \supset A_4 \supset K \supset \left\{e\right\}$, where $K$ -
  is a subgroup of double transpositions (see example
  \ref{ex:permutation} for permutation cycles notation):
  \[
  K = \left\{
  e, (12)(34), (13)(24), (14)(23)
  \right\}.
  \]
  A double transposition is a product of two transpositions with
  distinct support, right, which permute the distinct elements.

  $A_4 \triangleleft S_4$, $\left|S_4/A_4\right| = 2$, i.e.
  $S_4/A_4$ is cyclic of order 2.

  $K \triangleleft A_4$, $\left|A_4/K\right| = 3$, i.e.
  $A_4/K$ is cyclic of order 3.

  $K$ is \nameref{def:abeliangroup} and
  $K \cong \mathbb{Z}/2\mathbb{Z} \times \mathbb{Z}/2 \mathbb{Z}$.

  So this shows that $S_4$ is solvable.
  \label{ex:lec8_s4}
\end{example}

\section{Properties of solvable groups. Symmetric group}

\begin{property}
  If $G$ is solvable and $H \subset G$ is a subgroup of $G$ then $H$
  is solvable.
  \begin{proof}
    Indeed $G_i \cap H$ gives a filtration with required property.
  \end{proof}
  \label{property:lec8_solvable1}
\end{property}

\begin{property}
  If $G$ is solvable and $H \triangleleft G$ is a normal subgroup of
  $G$ then $G/H$ is solvable.
  \begin{proof}
    Indeed consider a projection map
    \begin{equation}
      \pi: G \to G/H
      \label{eq:lec8_solvable_pi}
    \end{equation}
    then $\pi\left(G_i\right)$ gives a filtration $\left(G/H\right)_i$
    on $G/H$ with required properties.
  \end{proof}
  \label{property:lec8_solvable2}
\end{property}

\begin{property}
  If $H \triangleleft G$, $H$ and $G/H$ are solvable then $G$ is
  solvable. 
  \begin{proof}
    Put togeter the filtration $H_i$ and
    $\pi^{-1}\left(\left(G/H\right)_j\right)$ (see
    (\ref{eq:lec8_solvable_pi}) for $\pi$ definition).
  \end{proof}
  \label{property:lec8_solvable3}
\end{property}

\begin{property}
  If $G$ is finite than $G$ is solvable (i.e. has a finite filtration with
  Abelian quotients) if and only if there exists a
  finite filtration with cyclic quotients.
  \begin{proof}
    This is just because a finite \nameref{def:abeliangroup} is just a
    product of cyclic groups.
  \end{proof}
  \label{property:lec8_solvable4}
\end{property}

Lets also look at another definition of solvable group
\begin{definition}[Solvable group]
  $G$ is called solvable if the following sequence is finite:
  \[
  G
  \supseteq \left[G, G\right] = G^{(1)}
  \supseteq \left[G^{(1)}, G^{(1)}\right] = G^{(2)}
  \supseteq \dots \supseteq
  \left[G^{(n-1)}, G^{(n-1)}\right] = G^{(n)} = \left\{e\right\}
  \]
  where $G^{(i)} = \left[G^{(i-1)}, G^{(i-1)}\right]$ is the
  \nameref{def:commutatorsubgroup}.
  \label{def:solvablegroupadd}
\end{definition}

\begin{remark}
  Definitions of solvable group \ref{def:solvablegroupadd} and
  \ref{def:solvablegroup} are equivalent.
  \begin{proof}
    Our filtration with \nameref{def:commutatorsubgroup}s
    $G \supseteq G^{(1)} \supseteq \dots \supseteq G^{(n)} =
    \left\{e\right\}$ is a filtration with abelian quotient because
    $G^{(i)}/\left[G^{(i)}, G^{(i)}\right] = G^{(i)}/G^{(i+1)}$ is an
    \nameref{def:abeliangroup}.

    From the other hand if $G/H$ is an \nameref{def:abeliangroup} then
    $H \supset \left[G, G\right]$. So if a finite filtration with
    abelian quotient exists then the filtration given by $G^{(i)}$ is
    also finite. It must terminate after a finite steps. So, this
    proves the equivalence.  
  \end{proof}
  \label{rem:lec8_solvable}
\end{remark}

\begin{theorem}[$S_n$ solvability]
  $S_n$ - the permutation of $n$ elements (see example
  \ref{ex:sngroup}) is not solvable for $n \ge 5$. 
  \begin{proof}
    It's easy to use definition \ref{def:solvablegroupadd}.
    Main steps are the following
    \begin{enumerate}
      \item we know that $\left[S_n, S_n\right] = A_n$ - subgroup of
        even permutations (see definition
        \ref{def:alternatinggroup}). It can be see from the fact that
        any 3-cycle is a \nameref{def:commutatorsubgroup} and 3-cycles
        generate $A_n$
        \footnote{
          3-cycle are even permutations and result of
          their compositions is also even
        }
      \item If $n \ge 5$ then $\left[A_n, A_n\right] = A_n$ thus the
        filtration generated by commutators will never terminate
        i.e. will never reach the unity ($\left\{e\right\}$) and will
        stabilize on $A_n$. How we can see it? We can remember that
        $\left[A_4, A_4\right] = K$ (see example \ref{ex:lec8_s4}) -
        the subgroup of double transpositions. $A_4 \hookrightarrow
        A_n$ in many different ways. Because you can pick any 4
        elements, our $n$ elements and just consider the permutations of
        those 4 elements as a subgroup of permutations of $n$
        elements and then taking the commutators of those $A_4$, we
        see that all double transpositions are in the $\left[A_n,
          A_n\right]$ (\nameref{def:commutatorsubgroup} of $A_n$). But
        if $n \ge 5$, they generate $A_n$.          
    \end{enumerate}
  \end{proof}
  \label{thm:lec8_sn_solvability}
\end{theorem}

\section{Galois theorem on solvability by radicals}

\begin{theorem}
  Let $P \in K\left[X\right]$. $P$ is a
  \nameref{def:solvablepolynomial} if and only if
  $Gal\left(P\right)$ is solvable. There
  $Gal\left(P\right)$ is (by definition) $Gal\left(F/K\right)$ where
  $F$ is a \nameref{def:splittingfield} of $P$ over $K$.
  \begin{proof}
    First of all lets proof that if $Gal\left(P\right)$ is solvable
    then $P$ is solvable. Let $n - \left[F:K\right]$ and consider
    $L = K\left(\zeta_n\right)$ where $\zeta_n$ - $n$-th root of 1.
    Let $M = FL$ - a \nameref{def:compositeextension}. So this is the
    splitting and field of $P$ of which we have adjoined all the
    $n$-th roots of unity.  Then $M$ is a
    \nameref{def:galoisextension} and
    $Gal\left(M/K\right) \hookrightarrow Gal\left(F/K\right)$.
    $\forall g \in Gal\left(M/K\right)$ leaves $F$ invariant.
    If $g\mid_F = id$ then $g = id$. Then the image in fact of this
    map is of the Galois group of $F$ over the intersection of $F$ and
    $L$.  So $G = Gal\left(M/K\right)$ is solvable i.e.
    \[
    G = G_0 \supset G_1 \supset \dots \supset G_r = \left\{e\right\}
    \]
    and $G_i/G_{i+1}$ - cyclic of order $n_i \mid n$. And as soon as
    $n_i \mid n$ (very important), all $n$-th roots of 1 are in $M$
    (this is why we adjoin the $L$).

    Let $M_i = M^{G_i}$. We know $M_i \hookrightarrow M_{i+1}$ is a
    cyclic Galois extension of order $n_i \mid n$ and roots of 1 are
    in it therefore there is Kummer extension (see section
    \ref{sec:kummerextension}). So
    $M_{i+1} = M_i\left(\sqrt[n_i]{a_i}\right)$ (see proposition
    \ref{prop:lec7_2}).
    So $M = K\left(\zeta_n, \alpha_1, \dots, \alpha_r\right)$ where
    $\alpha_i = \sqrt[n_i]{a_i}$. Therefore $M$ is solvable by
    radicals.

    For another direction: if $P$ is solvable then $G$ is
    solvable. Let $E$ is solvable extension containing $F$. We may
    suppose that this is Galois. Then write
    $E = K\left(\alpha_1, \dots, \alpha_r\right)$ where
    $\alpha_i^{n_i} \in K\left(\alpha_1, \dots,
    \alpha_{i-1}\right)$. Then let $L = K\left(\zeta_n\right)$
    where $n = LCM\left(\left\{n_i\right\}\right)$ so
    $\forall n_i: n_i \mid n$. And
    take $M = LE$. We have
    $K\left(\alpha_1, \dots, \alpha_{i-1}\right) \hookrightarrow
    K\left(\alpha_1, \dots, \alpha_i\right)$ - cyclic extension of
    order $n_i$. We have $Gal\left(M/L\right)$ is solvable by this
    cyclic subgroups. $Gal\left(M/K\right)$ is also solvable since
    $Gal\left(M/L\right)$ subgroup and the quotient
    $\cong Gal\left(L/K\right)$ which is abelian.
    $Gal\left(F/K\right)$ is a quotient of $Gal\left(M/K\right)$ thus
    is solvable too.    
  \end{proof}
  \label{thm:lec8_1}
\end{theorem}

\section{Examples of equations not solvable by radicals."General
  equation"}
As we can see there exist equations which are not solvable in
radicals.

\begin{example}[Not solvable polynomial of degree 5]
  Let $P \in \mathbb{Q}\left[X\right]$ is an irreducible polynomial
  with rational 
  coefficients of degree 5. It has 3 real roots (and 2 complex
  conjugate roots) as it shown on the picture.
  \begin{tikzpicture}
    \draw[->] (-1.5,0) -- (3,0) node[right] {$x$};
    \draw[->] (0,-1.5) -- (0,1.5) node[above] {$y$};
    \draw[scale=0.5,domain=-2.5:4.2,smooth,variable=\x,blue] plot
    ({\x},{0.01*(\x-1)*(\x-1)*(\x-1)*(\x-1)*(\x-1) - 0.5*(\x-1) + 0.01});
  \end{tikzpicture}

  We claim that $Gal\left(P\right) = S_5$. This is because
  \begin{enumerate}
    \item $Gal\left(P\right)$ contains the complex conjugation (we
      have 2 complex conjugated roots but \nameref{def:galoisgroup} is
      the group of automorphisms which exchange roots and the complex
      conjugation will exchange the 2 complex roots).
      The complex conjugation is the transposition of roots
    \item As soon as $P$ is irreducible then $Gal\left(P\right)$
      should act transitively (see definition \ref{def:transitive}) on
      roots. We have an irreducible polynomial. We can always send
      one of its roots to another of its roots. We have this
      isomorphism of stem fields which extends to an automorphism of
      the splitting field. But, what is the subgroup of $S_5$, which adds
      transitively?

      $Gal\left(P\right) \subset S_5$ acts transitively. This means
      that $5 \mid \left|Gal\left(P\right)\right|$. That is because
      $\left|G\right| = \left|orbit x\right| \left|stabilizer
      x\right|$ but the orbit has 5 elements and therefore 5 divides
      the cardinality of $G$. This means, by \nameref{cor:sylow} 
      theorems, that our group contains something of order 5. But only
      5-cycle has order 5. But a 5-cycle and transposition generate
      $S_5$. So $Gal\left(P\right) = S_5$.      
  \end{enumerate}
  In fact, the same argument is valid for $S_p$ with every $p$ - prime. 
  I.e. applies to an arbitrary prime number $p$ instead of 5.

  So $Gal\left(P\right) = S_5$ - not solvable and therefore $P$ is not
  solvable by radicals.
  \label{ex:lec8_notsolvable1}
\end{example}

\begin{example}[General equation of degree $n$]
  What's the general equation. It is the following
  \[
  X^n - T_1 X^{n-1} + T_2 X^{n-2} + \dots + \left(-1\right)^n T_n,
  \]
  where $T_i$ is a variable. Where does it come from? Let
  $X_1, \dots, X_n$ are roots of a polynomial of degree $n$ when the
  polynomial itself is
  \begin{eqnarray}
  \left(X - X_1\right) \cdot \dots \cdot \left(X - X_n\right) =
  X^n - \left(\sum_i X_i \right) X^{n-1} +
  \nonumber \\
  +
  \left(\sum_{i,j} X_i X_j \right) X^{n-2} + \dots +
  \left(-1\right)^n \prod_i X_i,
  \nonumber
  \end{eqnarray}
  i.e.
  $T_1 = \sum_i X_i, T_2 = \sum_{i,j} X_i X_j, \dots, T_n = \prod_i
  X_i$.

  One has $K\left[T_1, \dots, T_n\right] \subset K\left[X_1, \dots,
    X_n\right]$ (multi-variable polynomial rings). We have the same
  also for field extensions:
  $K\left(T_1, \dots, T_n\right) \subset K\left(X_1, \dots,
  X_n\right)$. The $K\left(X_1, \dots, X_n\right)$ is algebraic and a
  splitting field for our general polynomial. So it has degree at most
  $n!$. On the other hand
  $K\left(T_1, \dots, T_n\right) \subset K\left(X_1, \dots,
  X_n\right)^{S_n}$ so degree of the extension is $n!$ and
  $K\left(T_1, \dots, T_n\right) = K\left(X_1, \dots,
  X_n\right)^{S_n}$. In particular the Galois group is $S_n$ and our
  general polynomial is not solvable by radicals if $n \ge 5$.
  This is known as Abel theorem
  \label{ex:lec8_generalequation}
\end{example}

\section{Galois action as a representation. Normal base theorem}
\section{Normal base theorem (cont'd). Relation with coverings}
