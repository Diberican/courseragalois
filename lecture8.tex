%% -*- coding:utf-8 -*-
\chapter{Solvability by radicals, Abel's theorem. A few words on
  relation to representations and topology}

We finally arrive to the source of Galois theory, the question which
motivated Galois himself: which equation are solvable by radicals and
which are not? We explain Galois' result: an equation is solvable by
radicals if and only if its Galois group is solvable in the sense of
group theory. In particular we see that the "general" equation of
degree at least 5 is not solvable by radicals. We briefly discuss the
relations to representation theory and to topological coverings.

\section{Extensions solvable by radicals. Solvable groups. Example} 

Let $K$ is a field of characteristic 0: $char K = 0$. It is embedded
into its \nameref{def:algebraicclosure}.

\begin{definition}[Extension solvable by radicals]
  A finite extension $E$ of $K$ is solvable by radicals if
  $\exists \alpha_1, \dots, \alpha_r$ generating $E$ such that
  $\alpha_i^{n_i} \in K\left(\alpha_1, \dots, \alpha_{i-1}\right)$ for
  some $n_i \in \mathbb{N}$.
\end{definition}

\section{Properties of solvable groups. Symmetric group}
\section{Galois theorem on solvability by radicals}
\section{Examples of equations not solvable by radicals."General equation"}
\section{Galois action as a representation. Normal base theorem}
\section{Normal base theorem (cont'd). Relation with coverings}
