%% -*- coding:utf-8 -*-
\chapter{Ring extensions, norms and traces, reduction bp}
We build a tool for finding elements in Galois groups, learning to use
the reduction modulo $p$. For this, we have to talk a little bit about
integral ring extensions and also about norms and traces.


\section{Integral elements over a ring}

Let $P \in \mathbb{Z}\left[X\right]$. We want to know what is
$Gal\left(P\right)$. Just a reminder that $Gal\left(P\right) =
Gal\left(K/\mathbb{Q}\right)$ where $K$ is a
\nameref{def:splittingfield} of $P$. We have already done the work for
several types of polynomials: cyclotomic polynomial,
\nameref{sec:kummerextension} and so on.

Sometimes if, our polynomial is a kind of combination of then the
explicit information about the roots helps to calculate the Galois
group. For instance if we have polynomial $X^5 - 2$ we know it's
roots: $\sqrt[5]{2}, j^k \sqrt[5]{2}$, where $j= e^{\frac{2 \pi
    i}{5}}, 1 \le k \le 4$. Now we have a lot about
\nameref{def:galoisgroup}. If $K$ is the splitting field of the
polynomial then we have the following towers:

  \begin{tikzpicture}[descr/.style={fill=white,inner sep=2.5pt}]
    \matrix (m) [matrix of math nodes, row sep=3em,
      column sep=3em]
            { & \mathbb{Q}\left(j\right) & \\
              \mathbb{Q} & & \mathbb{Q}\left(\sqrt[5]{2}, j\right) = K\\
              & \mathbb{Q}\left(\sqrt[5]{2}\right) & \\ };
            \path[->,font=\scriptsize]
            (m-2-1) edge node[descr] {$ 4 $} (m-1-2)
            (m-1-2) edge node[descr] {$ 5 $} (m-2-3)
            (m-2-1) edge node[descr] {$ 5 $} (m-3-2)
            (m-3-2) edge node[descr] {$ 4 $} (m-2-3);
  \end{tikzpicture}
  
From that we know we can conclude that it follows that our Galois
group, contains a normal cyclic subgroup of a order of five
$\mathbb{Z}/5\mathbb{Z}$. And then 
the quotient is the Galois group of cyclotomic extension, so this is
$\left(\mathbb{Z}/5\mathbb{Z}\right)^\times$. So this is a group of
order 20. You can show that this 
is noncommutative, and from this exact sequence, you have some
information about it. But what will we do if we don't know the
roots. One of the tool that we will use is the reduction of modulo
prime and this will be the subject of the lecture.

\subsection{Ring extensions}
\begin{definition}[Integral element]
  Let $A$ be an \nameref{def:integraldomain}, i.e. a ring without zero
  divisors. And let $B$ is an extension of $A$. The element $\alpha
  \in B$ is called integral over $A$ if $\alpha$ is a root of a
  \nameref{def:monicpolynomial} $P \in A\left[X\right]$.

  So one can write the following relation
  \[
  \alpha^n + a_{n-1}\alpha^{n-1} + \dots + a_1 \alpha + a_0 = 0, a_i
  \in A.
  \]
  \label{def:integralelement}
\end{definition}

\begin{example}
  $\frac{1}{2}$ is not integral element over $\mathbb{Z}$ but
  $\sqrt{2}$ is an \nameref{def:integralelement} over $\mathbb{Z}$.

  This is because the polynomial in the definition
  \ref{def:integralelement} is monic i.e. the leading coefficient is
  1. 
\end{example}

\begin{lemma}
  The following conditions are equvivalent
  \begin{enumerate}
  \item $\alpha$ is integral over $A$.
  \item $A\left[\alpha\right]$ is a finitely generated $A$-module
    (see definition \ref{def:fgmodule}).
  \item $A \subset C \subset B$ where $C$ is a a finitely generated
    $A$-module (see definition \ref{def:fgmodule}). I.e. $A$ is contained in a finitely generated $A$-module.
  \end{enumerate}
  \begin{proof}
    $1 \to 2 \to 3$ is easy
    \footnote{
      ??? provide an explanation
    } and we will concentrate on $3 \to 1$.

    Let $x_1, \dots, x_r$ generate $C$ as $A$-module then we can write
    \[
    \alpha x_i = \sum \lambda_{ij}x_j,
    \]
    where $\lambda_{ij} \in A$. Consider the matrix
    $\Lambda = \{\lambda_{ij}\}$ and let $M = \alpha \cdot id -
    \Lambda$. Then
    \[
    M \cdot
    \begin{pmatrix}
      x_1 \\ \vdots \\ x_r
    \end{pmatrix} = 0.
    \]
    Thus
    \[
    \det M \cdot
    \begin{pmatrix}
      x_1 \\ \vdots \\ x_r
    \end{pmatrix} = 0.
    \]
    Therefore $det M \cdot C = 0$ but $1 \in C$ thus $\det M = 0$.
    The equation $\det M = 0$ can be considered as a polynomial with
    $\alpha$ as a root.
  \end{proof}
  \label{lem:lec9_1}
\end{lemma}

\section{Integral extensions, integral closure, ring of integers of a
  number field}

\subsection{Integral extensions and integral closure}

\begin{definition}[Integral extension]
  Let $A \subset B$. $B$ is integral over $A$ if $\forall \alpha \in
  B$, $\alpha$ is an \nameref{def:integralelement} over $A$.  
  \label{def:integralextension}
\end{definition}

\begin{proposition}
  Let $A \subset B \subset C$. $B$ integral over $A$, $C$ integral
  over $B$ then $C$ is an \nameref{def:integralextension} over $A$.
  \begin{proof}
    Proof is left as an exercise
    \footnote{
      ??? provide the proof
    }
  \end{proof}
  \label{prop:lec9_1}
\end{proposition}

\begin{proposition}
  Let $B$ is a finitely generated over $A$ as a module (see definition
  \ref{def:fgmodule}) if and only if
  $B=A\left[\alpha_1, \dots, \alpha_r\right]$ where each $\alpha_i$ is
  an \nameref{def:integralelement} over $A$.
  \begin{proof}
    Proof is left as an exercise
    \footnote{
      ??? provide the proof
    }
  \end{proof}
  \label{prop:lec9_2}
\end{proposition}

\begin{proposition}
  Let $A \subset B$. I.e. $B$ is an arbitrary extension of $A$. The
  elements of $B$ which are integral over $A$ form a subring of $B$
  (one calls it as the integral closure of $A$ in $B$).
  \begin{proof}
    Let $\alpha, \beta$ are integral over $A$ then $A\left[\alpha,
      \beta\right]$ - finitely generated $A$-module (see definition
    \ref{def:fgmodule}). This follows directly from lemma
    \ref{lem:lec9_1}. It contains $\alpha + \beta$ and $\alpha \beta$
    and by lemma \ref{lem:lec9_1} the $\alpha + \beta$ and $\alpha
    \beta$ are integral over $A$. But this is exactly we need to
    proof. 
  \end{proof}
  \label{prop:lec9_3}
\end{proposition}

\begin{definition}[Integrally closed]
  Let $A \subset B$. $A$ is integrally closed in $B$ if the integral
  closure of $A$ in $B$ equals to $A$.

  $A$ is integrally closed (without mention of any $B$) if it is
  integrally closed in \nameref{def:fractionfield} $\mathrm {Frac}(A)$.
  \label{def:integrallyclosed}
\end{definition}

\begin{example}
  $\mathbb{Z}$ is \nameref{def:integrallyclosed}. 
\end{example}

\begin{remark}
  More generally any
  \nameref{def:ufd} is \nameref{def:integrallyclosed}.
  \begin{proof}
    Let $A$ be a \nameref{def:ufd} and $x \in \mathrm {Frac}(A)$ such
    that $x \ne 0$. So
    $x = \frac{p}{q}$ such that $p,q \in A, \left(p, q\right) = 1$
    (this means no common prime divisor). If $x$ integral over $A$
    then
    \[
    \left(\frac{p}{q}\right)^n + a_{n-1}\left(\frac{p}{q}\right)^{n-1}
    + \dots + a_1 \frac{p}{q} + a_0 = 0.
    \]
    Thus
    \[
    \frac{p^n + q a_n p^{n-1} + q^2 a_{n-1} p^{n-2} + \dots + q^{n-1}
      a_1 p + q^n a_0}{q^n} = 0
    \]
    therefore $q \mid p^n$ which is in contradiction with
    $\left(p, q\right) = 1$. Unless $q$ is invertable that is $x \in A$.
    \end{proof}
\end{remark}

\subsection{Ring of integers in a number field}

\begin{definition}[Number field]
  Let $K$ is a finite extension of $\mathbb{Q}$ i.e.
  $\left[K:\mathbb{Q}\right] < \infty$. In the case $K$ is a number
  field. 
  \label{def:numberfield}
\end{definition}

Let $K$ is a \nameref{def:numberfield} and $\left[K:\mathbb{Q}\right]
= N$.
\begin{definition}[Ring of integers]
  Let $K$ is a \nameref{def:numberfield}. The ring of integers
  $O_K \subset K$ is the integral closure of $\mathbb{Z}$
  in $K$. 

  Note: We know that integral closure of $\mathbb{Z}$ in $\mathbb{Q}$
  is $\mathbb{Z}$ but now we consider the closure in $K$ but not in
  $\mathbb{Q}$. 
  \label{def:ringintegers}
\end{definition}

\begin{property}
  \begin{enumerate}
  \item $\forall \alpha \in K, \exists d \in \mathbb{Z} \setminus
    \{0\}$ such that $d\alpha \in O_K$.
  \item If $\alpha \in O_K$ then $P_{min}\left(\alpha,
    \mathbb{Q}\right) \in \mathbb{Z}\left[X\right]$.
  \end{enumerate}
  \begin{proof}
    For the first part lets
    $P_{min}\left(\alpha,
    \mathbb{Q}\right) = X^m + a_{m-1}X^{m-1} + \dots + a_1 X + a_0
    \in \mathbb{Q}\left[X\right]$.

    $\exists d \in \mathbb{Z}$ (the common denominator) such that
    $\forall i: d a_i \in \mathbb{Z}$. So
    $b_i = d^{m-i}a_i \in \mathbb{Z}$ for any $i$. Therefore
    \[
    \left(d \alpha\right)^m + b_{m-1} \left(d \alpha\right)^{m-1} +
    \dots + b_0 = 0.
    \]
    Thus $d \alpha \in O_K$.

    The second part is also easy. If we have such
    $\alpha \in O_K$, it is a
    root of some \nameref{def:monicpolynomial}
    $Q \in \mathbb{Z}\left[X\right]$.
    Then the $P_{min} \mid Q$. So $Q = P_{min} R$.
    If we pick $P_{min}$ to be monic, then by an
    argument very similar to that of the \nameref{lem:gauss} lemma,
    we conclude  that both $P_{min}, R \in \mathbb{Z}\left[X\right]$.  
  \end{proof}
  \label{property:lec9_2}
\end{property}

\section{Norm and trace}

\subsection{Finitely generated Abelian groups}
(The material was given inside the proof of theorem \ref{thm:lec9_1}
and can be considered as a recall)
The \nameref{def:fgagroup} is the same as
finitely generated $\mathbb{Z}$-module.



\subsection{Theorem about rings of integers}
\begin{theorem}
  $O_k$ is a finitely generated (see definition \ref{def:fgmodule})
  $\mathbb{Z}$-module that is a \nameref{def:freemodule} of
  rank (see definition \ref{def:rankfreemodule}) $N$, where $N=
  \left[K:\mathbb{Q}\right]$. 
  \begin{proof}
    If $e_1, \dots, e_n$ is a $\mathbb{Q}$-basis of $K$ then
    $\forall i \exists d_i \in \mathbb{Z} \setminus
    \{0\}$ such that $d_i e_i \in O_K$ (see property
    \ref{property:lec9_2}). Therefore $O_K$ contains a free
    $\mathbb{Z}$-submodule of rank $N$
    \footnote{
      This is because $d_1 e_1, \dots, d_n e_n$ are linearly
      independent and form a basis of a free $\mathbb{Z}$-module. The
      number of the cardinality of the basis $n = N$.
    }.

    What is the $\mathbb{Z}$-module this is a finitely generated
    \nameref{def:fgagroup} and we know a lot of things about such
    groups. 
  \end{proof}
  \label{thm:lec9_1}
\end{theorem}

\section{Norm and trace (cont'd). Ring of integers is a free module}
\section{Reduction modulo a prime}
\section{Reduction modulo a prime and finding elements in Galois groups}
