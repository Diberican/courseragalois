%% -*- coding:utf-8 -*-
\chapter{Ring extensions, norms and traces, reduction bp}
We build a tool for finding elements in Galois groups, learning to use
the reduction modulo $p$. For this, we have to talk a little bit about
integral ring extensions and also about norms and traces.


\section{Integral elements over a ring}

Let $P \in \mathbb{Z}\left[X\right]$. We want to know what is
$Gal\left(P\right)$. Just a reminder that $Gal\left(P\right) =
Gal\left(K/\mathbb{Q}\right)$ where $K$ is a
\nameref{def:splittingfield} of $P$. We have already done the work for
several types of polynomials: cyclotomic polynomial,
\nameref{sec:kummerextension} and so on.

Sometimes if, our polynomial is a kind of combination of then the
explicit information about the roots helps to calculate the Galois
group. For instance if we have polynomial $X^5 - 2$ we know it's
roots: $\sqrt[5]{2}, j^k \sqrt[5]{2}$, where $j= e^{\frac{2 \pi
    i}{5}}, 1 \le k \le 4$. Now we have a lot about
\nameref{def:galoisgroup}. If $K$ is the splitting field of the
polynomial then we have the following towers:

  \begin{tikzpicture}[descr/.style={fill=white,inner sep=2.5pt}]
    \matrix (m) [matrix of math nodes, row sep=3em,
      column sep=3em]
            { & \mathbb{Q}\left(j\right) & \\
              \mathbb{Q} & & \mathbb{Q}\left(\sqrt[5]{2}, j\right) = K\\
              & \mathbb{Q}\left(\sqrt[5]{2}\right) & \\ };
            \path[->,font=\scriptsize]
            (m-2-1) edge node[descr] {$ 4 $} (m-1-2)
            (m-1-2) edge node[descr] {$ 5 $} (m-2-3)
            (m-2-1) edge node[descr] {$ 5 $} (m-3-2)
            (m-3-2) edge node[descr] {$ 4 $} (m-2-3);
  \end{tikzpicture}
  
From that we know we can conclude that it follows that our Galois
group, contains a normal cyclic subgroup of a order of five
$\mathbb{Z}/5\mathbb{Z}$. And then 
the quotient is the Galois group of cyclotomic extension, so this is
$\left(\mathbb{Z}/5\mathbb{Z}\right)^\times$. So this is a group of
order 20. You can show that this 
is noncommutative, and from this exact sequence, you have some
information about it. But what will we do if we don't know the
roots. One of the tool that we will use is the reduction of modulo
prime and this will be the subject of the lecture.

\subsection{Ring extensions}
\begin{definition}[Integral element]
  Let $A$ be an \nameref{def:integraldomain}, i.e. a ring without zero
  divisors. And let $B$ is an extension of $A$. The element $\alpha
  \in B$ is called integral over $A$ if $\alpha$ is a root of a
  \nameref{def:monicpolynomial} $P \in A\left[X\right]$.

  So one can write the following relation
  \[
  \alpha^n + a_{n-1}\alpha^{n-1} + \dots + a_1 \alpha + a_0 = 0, a_i
  \in A.
  \]
  \label{def:integralelement}
\end{definition}

\begin{example}
  $\frac{1}{2}$ is not integral element over $\mathbb{Z}$ but
  $\sqrt{2}$ is an \nameref{def:integralelement} over $\mathbb{Z}$.

  This is because the polynomial in the definition
  \ref{def:integralelement} is monic i.e. the leading coefficient is
  1. 
\end{example}

\begin{lemma}
  The following conditions are equvivalent
  \begin{enumerate}
  \item $\alpha$ is integral over $A$.
  \item $A\left[\alpha\right]$ is a finitely generated $A$-module
    (see definition \ref{def:fgmodule}).
  \item $A \subset C \subset B$ where $C$ is a a finitely generated
    $A$-module (see definition \ref{def:fgmodule}). I.e. $A$ is contained in a finitely generated $A$-module.
  \end{enumerate}
  \begin{proof}
    $1 \to 2 \to 3$ is easy
    \footnote{
      ??? provide an explanation
    } and we will concentrate on $3 \to 1$.

    Let $x_1, \dots, x_r$ generate $C$ as $A$-module then we can write
    \[
    \alpha x_i = \sum \lambda_{ij}x_j,
    \]
    where $\lambda_{ij} \in A$. Consider the matrix
    $\Lambda = \{\lambda_{ij}\}$ and let $M = \alpha \cdot id -
    \Lambda$. Then
    \[
    M \cdot
    \begin{pmatrix}
      x_1 \\ \vdots \\ x_r
    \end{pmatrix} = 0.
    \]
    Thus
    \[
    \det M \cdot
    \begin{pmatrix}
      x_1 \\ \vdots \\ x_r
    \end{pmatrix} = 0.
    \]
    Therefore $det M \cdot C = 0$ but $1 \in C$ thus $\det M = 0$.
    The equation $\det M = 0$ can be considered as a polynomial with
    $\alpha$ as a root.
  \end{proof}
  \label{lem:lec9_1}
\end{lemma}

\section{Integral extensions, integral closure, ring of integers of a
  number field}

\begin{definition}[Integral extension]
  Let $A \subset B$. $B$ is integral over $A$ if $\forall \alpha \in
  B$, $\alpha$ is an \nameref{def:integralelement} over $A$.  
  \label{def:integralextension}
\end{definition}

\begin{proposition}
  Let $A \subset B \subset C$. $B$ integral over $A$, $C$ integral
  over $B$ then $C$ is an \nameref{def:integralextension} over $A$.
  \begin{proof}
    Proof is left as an exercise
    \footnote{
      ??? provide the proof
    }
  \end{proof}
  \label{prop:lec9_1}
\end{proposition}

\begin{proposition}
  Let $B$ is a finitely generated over $A$ as a module (see definition
  \ref{def:fgmodule}) if and only if
  $B=A\left[\alpha_1, \dots, \alpha_r\right]$ where each $\alpha_i$ is
  an \nameref{def:integralelement} over $A$.
  \begin{proof}
    Proof is left as an exercise
    \footnote{
      ??? provide the proof
    }
  \end{proof}
  \label{prop:lec9_2}
\end{proposition}

\begin{proposition}
  Let $A \subset B$. I.e. $B$ is an arbitrary extension of $A$. The
  elements of $B$ which are integral over $A$ form a subring of $B$
  (one calls it as the integral closure of $A$ in $B$).
  \begin{proof}
    Let $\alpha, \beta$ are integral over $A$ then $A\left[\alpha,
      \beta\right]$ - finitely generated $A$-module (see definition
    \ref{def:fgmodule}) 
  \end{proof}
  \label{prop:lec9_3}
\end{proposition}


\section{Norm and trace}
\section{Norm and trace (cont'd). Ring of integers is a free module}
\section{Reduction modulo a prime}
\section{Reduction modulo a prime and finding elements in Galois groups}
