%% -*- coding:utf-8 -*-
\chapter{Galois correspondence and first examples. Examples continued}
We continue to study the examples: cyclotomic extensions (roots of
unity), cyclic extensions (Kummer and Artin-Schreier extensions). We
introduce the notion of the composite extension and make remarks on
its Galois group (when it is Galois), in the case when the composed
extensions are in some sense independent and one or both of them is
Galois. The notion of independence is also given a precise sense
("linearly disjoint extensions").


\section{Cyclotomic extensions (cont'd). Examples over $\mathbb{Q}$}

Last time we discussed cyclotomic extensions which are splitting
fields of $\Phi_n$ (generated by n-th roots
(\nameref{def:primitiverootsofunity}) of 1). And we got a very precise
description of those extensions in the case when $\Phi_n$ was
irreducible, for instance, over $\mathbb{Q}$.

We have seen (see theorem \ref{thm:lec6_3}) that
$\mathbb{Q}\left(\zeta_n\right)$ is a \nameref{def:galoisextension} of
\nameref{def:galoisgroup} $\left(\mathbb{Z}/n\mathbb{Z}\right)^\times$
(see example \ref{ex:multiplicativegroup}) 
where $\zeta_n = e^{\frac{2 \pi i}{n}}$. So it acts as $g_a: \zeta_n \to
\zeta_n^a$ where $a \in \left(\mathbb{Z}/n\mathbb{Z}\right)^\times$
that can be considered as a number relatively prime to $n$: $\left(a,
n\right) = 1$.
\footnote{
  $\left(\mathbb{Z}/n\mathbb{Z}\right)^\times$ consists of elements
  which are invertible in $\mathbb{Z}/n\mathbb{Z}$. The numbers
  which are prime to $n$ are invertible.
}

Lets consider several examples

\begin{example}[$n=8$]
  Lets consider $n = 8$. In the case
  \[
  \left|\left(\mathbb{Z}/8\mathbb{Z}\right)^\times\right| = 4
  \]
  i.e. the group has 4 elements there are
  \[
  \left(\mathbb{Z}/8\mathbb{Z}\right)^\times =
  \left\{
  1,3,5,7
  \right\}.
  \]
  So our Galois group also has 4 elements
  \footnote{
    There is a well known Klein four group $V_4$ \cite{wiki:klein4group} -
    the only non-cyclic group of order 4 (really there are 2 groups of
    order 2: the first one is the Klein four group, the second one is
    the cyclic group of order 4). 
  }
  :
  \[
  Gal: \left\{
  id, \zeta_8 \to \zeta_8^3,
  \zeta_8 \to \zeta_8^5, \zeta_8 \to \zeta_8^7
  \right\} =
  \left\{
  id, \sigma_3, \sigma_5, \sigma_7
  \right\}.
  \]
  We can note that $\sigma_7 = \zeta_8 \to \zeta_8^7$ is something
  very simple - it is complex conjugation:
  $\sigma_7 = \zeta_8 \to \bar{\zeta}_8$. It's
  \nameref{def:fixedfield} $\mathbb{Q}\left(\zeta_8\right)^{\sigma_7}$
  is determined by the following expression
  \footnote{
    First of all by the \nameref{thm:galoiscorrespondence} theorem for
    any normal subgroup of the \nameref{def:galoisgroup} ($V_4$ in our
    case) there exists a sub-extension that is fixed by the normal
    sub-group. For our $V_4$ we have 3 normal subgroups:
    $\{id, \sigma_3\}, \{id, \sigma_5\}, \{id, \sigma_7\}$. Lets
    consider the last one i.e. lets find the extension that corresponds
    to $\{id, \sigma_7\}$. The extension has the following form (we
    are using triviality of $id$: $L^{id} = L$)
    \[
    \mathbb{Q}\left(\zeta_8\right)^{\{id, \sigma_7\}} =
    \mathbb{Q}\left(\zeta_8\right)^{\sigma_7}
    \]

    To calculate $\mathbb{Q}\left(\zeta_8\right)^{\sigma_7}$ we have
    to find a \nameref{ex:lec5_primitiveelement} $\nu$ such
    that $\nu \notin \mathbb{Q}$, $\nu \in
    \mathbb{Q}\left(\zeta_8\right)$ and $\sigma_7\left(\nu\right) =
    \nu$ (i.e. $\sigma_7$ fixes $\mathbb{Q}\left(\nu\right)$).
    Using the fact that $\bar{\zeta}_8 = \zeta_8^7$, it can
    be easy to check that
    \begin{eqnarray}
      \sigma_7\left(\zeta_8 + \bar{\zeta}_8\right) =
      \sigma_7\left(\zeta_8 + \zeta_8^7\right) =
      \nonumber \\
      =
      \zeta_8^7 + \zeta_8^{49} =
      \zeta_8^7 + \zeta_8^{6 \cdot 8 + 1} =
      \zeta_8^7 + \zeta_8 = \bar{\zeta}_8 + \zeta_8,
      \nonumber
    \end{eqnarray}
    i.e. we can take $\nu = \zeta_8 + \bar{\zeta}_8$ as an element
    that generates the required extension.
  }
  \[
  \mathbb{Q}\left(\zeta_8\right)^{\sigma_7} =
  \mathbb{Q}\left(\zeta_8\right) \cap \mathbb{R} =
  \mathbb{Q}\left(\zeta_8 + \bar{\zeta}_8\right) =
  \mathbb{Q}\left(\sqrt{2}\right)
  \]
  i.e. there is a quadratic extension.

  Our Galois group has 3 subgroups of order 2 so we have 3 quadratic
  sub-extensions. One o them we have already found
  ($\mathbb{Q}\left(\sqrt{2}\right)$) lets find 2 others.
  \footnote{
    The following equations were used
    \[
    \sigma_3\left(\zeta_8 + \zeta_8^3\right) =
    \zeta_8^3 + \zeta_8^9 = \zeta_8^3 + \zeta_8
    \]
    and
    \[
    \sigma_5\left(\zeta_8 \cdot \zeta_8^5\right) =
    \zeta_8^5 \cdot \zeta_8^{25} =
    \zeta_8^5 \cdot \zeta_8^{3 \cdot 8 + 1} =
    \zeta_8^5 \cdot \zeta_8.  
    \]
  }
  \[
  \mathbb{Q}\left(\zeta_8\right)^{\sigma_3} =
  \mathbb{Q}\left(\zeta_8 + \zeta_8^3\right) =
  \mathbb{Q}\left(i \sqrt{2}\right).
  \]
  and finally (with note $\zeta_8^5 = - \zeta_8, \zeta_8^6 = -i$)
  \footnote{
    we cannot choose $\zeta_8 + \zeta_8^5 = 0$ and have chosen
    $\zeta_8 \cdot \zeta_8^5$ instead of it.
  }
  \[
  \mathbb{Q}\left(\zeta_8\right)^{\sigma_5} =
  \mathbb{Q}\left(\zeta_8 \cdot \zeta_8^5\right) =
  \mathbb{Q}\left(\zeta_8^6\right) =
  \mathbb{Q}\left(i\right).
  \]
  \label{ex:lec7_cyclotomic8}
\end{example}

\begin{example}[$n=5$]
  $\mathbb{Q}\left(\zeta_5\right)$ where $\zeta_5 = e^{\frac{2 \pi
      i}{5}}$. The \nameref{def:galoisgroup} is the following:
  \[
  Gal \cong \left(\mathbb{Z}/5\mathbb{Z}\right)^\times
  \]
  that is a \nameref{def:cyclicgroup} of order 4.
  \footnote{
    As it was mentioned above there are only 2 finite group of order
    4. The first one $V_4$ was considered at example
    \ref{ex:lec7_cyclotomic8}. There is the second one that isomorphic
    to the cyclic group of order 4:
    \[
    \left(\mathbb{Z}/5\mathbb{Z}\right)^\times =
    \left\{ 1,2,3,4
    \right\}
    \]
  }
  It is generated by
  $\zeta_5 \to \zeta_5^2$
  \footnote{
    $Gal = \left<\zeta_5\right>$ and the group action on an element is
    just a multiplication by $\zeta_5$ i.e.
    $id \to \zeta_5, \zeta_5 \to \zeta_5^2, \zeta_5^2 \to \zeta_5^3,
    \zeta_5^3 \to \zeta_5^4, \zeta_5^4 \to id$
  }
  and it has only one
  \nameref{def:propersubgroup} $\cong \mathbb{Z}/2\mathbb{Z}$ so our
  field $\mathbb{Q}\left(\zeta_5\right)$
  has only one sub-field different from $\mathbb{Q}$ of course and
  this going to be a real part all of the complex conjugation which
  are part of Galois group. Now this is the same as the real part
  \(\mathbb{Q}\left(\zeta_5\right) \cap \mathbb{R} =
  \mathbb{Q}\left(\zeta_5 + \bar{\zeta}_5\right) =
  \mathbb{Q}\left(cos \frac{2 \pi}{5}\right)\).
\end{example}
So these were the examples of cyclotomic extensions of $\mathbb{Q}$
and of course the picture is exactly the same as long as the
cyclotomic polynomial is irreducible. If it is not reducible, which
can happen as we have seen, the Galois group becomes smaller.

\section{Kummer extensions}
\label{sec:kummerextension}

Consider a field $K$ such that the characteristics of $K$ is prime to
a certain number $n$: $\left(char(K), n\right) = 1$ and such that $X^n
- 1$ splits in $K$. So $K$ contains all roots of unity. Consider an
element $a$ of $K$ : $a \in K$ and let $\alpha = \sqrt[n]{a}$ (i.e. a
root of $X^n - a$). Take
\begin{equation}
  d = \min{\left\{ i \mid \alpha^i \in
    K\right\}}.
  \label{eq:lec7_d}
\end{equation}

\begin{proposition}
  $d \mid n$, minimal polynomial of $\alpha$ is $X^d - \alpha^d$ and
  $K\left(\alpha\right)$ is a \nameref{def:galoisextension} with
  cyclic \nameref{def:galoisgroup} of order $d$.
  \begin{proof}
    It's clear that $K\left(\alpha\right)$ is Galois because all the
    n-th roots of unity are in $K$. So $K\left(\alpha\right)$
    contains all roots of $X^n - a$.
    \footnote{
      Consider $\alpha_k = \alpha \zeta^k$ where
      $k = 0,1, \dots, n - 1$. All such $\alpha_k$ are roots of
      $X^n - a$ because
      \[
      \alpha_k^n - a = \alpha^n \left(\zeta^n\right)^k - a = a - a =0.
      \]
      We also have that $\forall k_1 \ne k_2: \alpha_{k_1} \ne
      \alpha_{k_2}$ because $\zeta^{k_1} \ne \zeta^{k_2}$. Therefore
      we have $n$ distinct roots i.e. all roots are in
      $K\left(\alpha\right)$
    }
    Therefore $K\left(\alpha\right)$
    is a splitting field of $X^n - a$ thus it's
    normal. The extension is also separable because
    $\left(char(K), n\right) = 1$.
    \footnote{
      If $P_n = X^n - a$ then $P_n' = n X^{n-1} \ne 0$ as soon as
      $\left(char(K), n\right) = 1$ and as result $(P_n, P_n') = 1$
      and $P_n$ does not have multiple roots and therefore it is
      separable. As result the extension $K\left(\alpha\right)$ is
      also separable.
    }
    Thus $K\left(\alpha\right)$ is \nameref{def:galoisextension}.

    Lets define a \nameref{def:homomorphism}
    $f: Gal\left(K\left(\alpha\right)/K\right) \xrightarrow[g \to
      \frac{g\left(\alpha\right)}{\alpha}]{} \mu_n$. This is correct
    because $g$ sends $\alpha$ to another root of $X^n - a$ thus the
    quotient $\frac{g\left(\alpha\right)}{\alpha}$ is a root of unity:
    \footnote{
      $g^n\left(\alpha\right) - \alpha^n =
      g^n\left(\alpha\right) - a = 0$
    }
    \[
    \left(\frac{g\left(\alpha\right)}{\alpha}\right)^n = 1.
    \]
    The homomorphism $f$ is \nameref{def:injection} because
    $g\left(\alpha\right)$ determines $g$.
    \footnote{
      If we have $g_1 \ne g_2$ then
      $g_1\left(\alpha\right) \ne g_2\left(\alpha\right)$ because
      in the case $g_1\left(\alpha\right)$ and
      $g_2\left(\alpha\right)$ are 2 different roots of $X^n - a$.
      As result the homomorphism $f$ is \nameref{def:injection}.
    }
    What's the image? It should
    be a \nameref{def:subgroup} of a \nameref{def:cyclicgroup} $\mu_n$
    but the subgroup should be also cyclic. 
    \footnote{
      see \nameref{thm:fundamentaltheoremofcyclicgroup}
    }
    Let $\delta$ is the order of the image and we want to show that
    $\delta = d$. Consider
    $g\left(\alpha^\delta\right) = f\left(g\right)^\delta \cdot
    \alpha^\delta = \alpha^\delta$ because $f\left(g\right)$ is a root
    of 1 ($f\left(g\right) = \sqrt[\delta]{1}$).
    \footnote{
      Using $g\left(\alpha\right)g\left(\alpha\right) = g\left(\alpha
      \cdot \alpha \right) = g\left(\alpha^2\right)$:
      \begin{eqnarray}
        f^\delta\left(g\right) =
        \frac{g^\delta\left(\alpha\right)}{\alpha^\delta} =
        \frac{g\left(\alpha^\delta\right)}{\alpha^\delta}.
        \nonumber
      \end{eqnarray}      
    }
    Thus $\alpha^\delta
    \in K$ ( see (\ref{eq:lec5_2})). And $\alpha^i \notin K$ for $i < \delta$ since otherwise
    $\deg P_{min}\left(\alpha, K\right) = i < \delta$. But this is
    impossible because
    \[
    \left[K\left(\alpha\right):K\right] =
    \left|Gal\left(K\left(\alpha\right)/K\right)\right| = \delta.
    \]
    Thus only possible option is $d = \delta$.
    \footnote{
      $d$ was chosen accordingly (\ref{eq:lec7_d}) and therefore
      $\delta \ge d$.
    }
    Thus
    $P_{min}\left(\alpha, K\right) = X^d - \alpha^d$.
  \end{proof}
  \label{prop:lec7_1}
\end{proposition}

\begin{proposition}
  And conversely (to \ref{prop:lec7_1}) for all cyclic extension of
  degree $n$ such that $\left(char(K), n\right) = 1$ is generated by
  $\sqrt[n]{a}$ for some $a \in K$.
  \begin{proof}
    Consider $L$ is an extension of $K$.
    $Gal\left(L/K\right) = \left<\sigma\right>$ then we have
    $\sigma^n = id$. Linear algebra says that $\sigma$ is
    \nameref{def:diagonalizable_map}.
    \footnote{
      Apply theorem \ref{thm:diagonalizable_matrix} to the
      diagonalizable $\sigma^n = id$. 
      Really a map is diagonalizable if it has $n$ distinct
      eigenvalues (see theorem
      \ref{thm:diagonalizable_map_eigenvalues}). This fact will be
      proved below. 
    }
    Now, let us show that all eigenspaces have dimension 1. Indeed if
    $x,y$ are in the same \nameref{def:eigenspace} then
    $\sigma\left(\frac{x}{y}\right) = \frac{x}{y}$
    \footnote{
      $L^{\left<\sigma\right>} = K$ i.e. $L^{\sigma} = K$.
    }
    because $x$ and $y$
    are multiplied by the same number.
    Therefore $\frac{x}{y} \in
    K$. And this is exactly means that dimension of the eigenspace is
    1, $x,y$ are proportional over $K$.
    \footnote{
      I.e. if $\mathcal{L}$ is the eigenspace then we have that
      $\exists x \in \mathcal{L}$ such that $\forall y \in
      \mathcal{L}: y = k x$, where $k \in K$. This exactly means that
      $\dim \mathcal{L} = 1$
    }
    Thus all roots of 1 are eigenvalues of $\sigma$.
    \footnote{
      Let $x$ - eigenvector of $\sigma$ and $\nu$ is the
      eigenvalue. We have $\sigma\left(x\right) = \nu x$, using
      $\sigma^n = id$ and $\sigma \circ \sigma(x) = \sigma(\nu x) =
      \nu^2 x$, one can get $\sigma^n\left(x\right) = \nu^n x
      = id(x) = x$. Thus $\nu^n = 1$ i.e. $\nu$ is a root of unity.
    }
    Then take
    $\alpha$ such that $\sigma\left(\alpha\right) = \zeta \alpha$
    where $\zeta$ is a \nameref{def:primitiverootsofunity}. Then
    $\left<\sigma\right>$ - orbit of $\alpha$ has $n$ elements
    therefore $\left[K\left(\alpha\right):K\right] = n$ (see
    explanation below) and $\alpha^n
    \in K$ since $\sigma\left(\alpha^n\right) = \zeta^n \cdot \alpha^n
    = \alpha^n$. We see that $\alpha$ is a root of $X^n - a$. This is
    irreducible by degree reason.

    Maybe I should have said here, why it follows from the formula,
    $\left<\sigma\right>$ - orbit of $\alpha$ has $n$ elements
    that the degree of the extension is exactly $n$. While this is
    easy because either degree of the extension was less than $n$,
    then also, $\alpha$ would have to be fixed by some non-trivial
    subgroup of Galois group by Galois correspondence. And then its
    orbit would have less than $n$ elements.
    \footnote{
      If we have $K \subset K\left(\alpha\right) \subset L$ and
      $\left<\sigma\right>$ is a \nameref{def:galoisgroup} $L/K$ and
      $\left[K\left(\alpha\right):\right] = d < n$ then by
      \nameref{thm:galoiscorrespondence} there exists a subgroup
      $H \subset \left<\sigma\right>$
      (cyclic as soon as  $\left<\sigma\right>$) that fixes the
      $K\left(\alpha\right)$ and especially $\forall h \in H:
      h\left(\alpha\right) = \alpha$
      therefore
      $\left|\sigma^k\left(\alpha\right)\right| < n$ i.e. the orbit
      contains less than $n$ elements.

      Another explanation as follows. If
      $\left[K\left(\alpha\right):\right] = d < n$ then
      $P_{min}\left(\alpha, K\right)$ has $d$ roots and the orbit
      $Orb\left(\alpha\right)$ consists of roots of
      $P_{min}\left(\alpha, K\right)$ and the number of roots is
      $d < n$.
    }
  \end{proof}
  \label{prop:lec7_2}
\end{proposition}
  
\section{Artin-Schreier extensions}
Let $n = char K$ this is called as Artin-Schreier extensions.

\begin{definition}[Cyclic extension]
  The Galois extension is called cyclic extension if the corresponding
  \nameref{def:galoisgroup} is cyclic.
  \label{def:cyclicextension}
\end{definition}

\begin{theorem}
  Let $p = char(K)$ and let
  $P = X^p - X - a \in K\left[X\right]$. Then $P$ is irreducible or
  splits over $K$. Let $\alpha$ be a root. If $P$ is irreducible then
  $K\left(\alpha\right)$ is \nameref{def:cyclicextension} of $K$ of
  degree $p$.

  Conversely any cyclic extension of degree $p$ is like this: $L/K,
  \exists \alpha \in K$ such that $L = K\left(\alpha\right)$, $\alpha$
  - root of $X^p - X - a$ for some $a \in K$.
  \begin{proof}
    First of all notice that roots of $P$ are $\alpha + k$ where $k
    \in \mathbb{F}_p$ ($k$ is an element of prime field).
    \footnote{
      In $\mathbb{F}_p$ we have $k^p = k$ and therefore 
      \begin{eqnarray}
        \left(\alpha + k\right)^p -
        \left(\alpha + k\right) -a =
        \nonumber \\
        =\alpha^p + k^p - \alpha -k - a =
        \alpha^p + k - \alpha -k - a =
        \nonumber \\
        =
        \alpha^p - \alpha - a =0,
        \nonumber
      \end{eqnarray}
      as soon as $\alpha$ is a root of $X^p - X - a$.
    }

    If $P$ is irreducible then \nameref{def:galoisgroup} should be
    transitive on the roots (see remark \ref{rem:lec5_onnormalext})  then
    $\exists \sigma \in Gal\left(K\left(\alpha\right)/K\right)$ such that
    $\sigma\left(\alpha\right) = \alpha + 1$ (because roots of $P$ are
    $\alpha + k$). The \nameref{def:grouporder} for $\sigma$ is
    $p = \left[K\left(\alpha\right):K\right]$
    \footnote{
      \[
      \sigma^p = \sigma(\sigma(\sigma( \dots ( \sigma(\alpha) ) \dots
      ))) = \alpha + p = \alpha
      \]
      i.e. $\sigma^p = id$ and order of $\left<\sigma\right>$ is $p$.
    }
    so the $\sigma$ must
    generate the \nameref{def:galoisgroup}:
    $Gal\left(K\left(\alpha\right)/K\right) = \left<\sigma\right>$.

    We have to show that if $P$ is not irreducible then $P$ splits
    i.e. $\alpha \in K$. Leave it for an exercise
    \footnote{
      Let $\alpha$ is a root then we can get $p$ different roots as
      $\alpha + k$ where $k \in \mathbb{F}_p$. Thus we have $p$
      different roots and the polynomial of degree $p$ ($X^p - X -a$)
      splits 
    }

    Now we will prove the converse statement. Let $L$ is a
    \nameref{def:cyclicextension} of $K$ of degree $p$. We want to
    find $\alpha$ such that $\sigma\left(\alpha\right) = \alpha + 1$
    where $\sigma$ is a generator of $Gal\left(L/K\right)$ (we know
    that the Galois group is cyclic i.e. must have the following form
    $Gal\left(L/K\right) = \left<\sigma\right>$).

    Set $f = \sigma -id$, $K = \ker f$
    \footnote{
      this is because
      $\forall x \in K: \sigma\left(x\right) = x$ (see
      (\ref{eq:lec5_2})).
    }
    and the \nameref{def:rank} $rg f = p -1$.
    \footnote{
      In lectures we can hear about range (\nameref{def:image}) not a
      \nameref{def:rank} but by future content we spoke about the rank
      but not about range (image). In any way the equation $rg f = p
      -1$ requires some explanation. As soon as $f^{p-1} \ne 0$
      $\exists x \in L$ such that $f^{p-1}\left(x\right) \ne
      0$. Therefore $f^{k}\left(x\right) \ne 0$  for all $k < p - 1$
      because in other case
      \[
      f^{p-1}\left(x\right) = f\left( \dots f\left(
      f^k\left(x\right) \right) \dots \right) =
      f\left( \dots f\left( 0 \right) \dots \right) =
      0.
      \]
      Lets show that $\left(f\left(x\right), f^2\left(x\right)\dots,
      f^{p-1}\left(x\right) \right)$ is linearly independent.
      For $k \in K, x \in L$ we can get
      \[
      f\left(k \cdot x\right) =
      \sigma\left(k \cdot x\right) - k \cdot x =
      \sigma\left(k\right) \cdot \sigma\left(x\right) -
      k \cdot x = k \cdot \sigma\left(x\right) - k \cdot x =
      k \cdot f\left(x\right).
      \]
      Thus for $k_i \in K$ such that:
      \[
      \sum_{i=1}^{p-1} k_i f^i\left(x\right) = 0
      \]
      applying $f^{p-1}$ one can get that $k_1 = 0$, applying
      $f^{p-2}$ gives us $k_2 = 0$. Continue the way we get that all
      $k_i = 0$. That proves the linear independence. Therefore
      $rg\left(f\right) \ge p - 1$. The vector
      $y = \left(f^{p-1}\left(x\right), f^{p-1}\left(x\right)\dots,
      f^{p-1}\left(x\right) \right)$ is not zero but $f(y) = 0$
      therefore $y \in \ker f$. This means that $\dim{\ker f} \ge
      1$. Using \nameref{thm:ranknullity} one can conclude that only
      possible choice there is $\dim{\ker f} = 1$ and
      $rg\left(f\right) = p - 1$. 
    }
    We have 
    $\left(\sigma - id\right)^p = 0$
    \footnote{
      In $\mathbb{F}_p$ we have
      \[
      \left(\sigma - id\right)^p = \sigma^p - id = 0
      \]
      as soon as $\left<\sigma\right>$ has order $p$.      
    }
    so the \nameref{def:kernel} must
    be included into \nameref{def:image}:
    \footnote{
      As soon as $f \ne 0$ exists $x \in L$ such that
      $y = f^{p-1}\left(x\right) \ne 0$ but $f\left(y\right) = 0$
      therefore $y \in \ker f$. Using the fact that $\dim{\ker f} = 1$
      (follows from $rg(f) = p - 1$ and \nameref{thm:ranknullity}) and
      $y \in \Ima f$ ($y = f\left(f^{p-2}\left(x\right)\right)$) one
      can conclude that  the \nameref{def:kernel} must
      be included into \nameref{def:image}.
    }
    $K = \ker f \subset \Ima f$
    because otherwise $L = \ker f \oplus \Ima f$ ($L$ is a
    \nameref{def:directsum}
    \footnote{
      see also definition \ref{def:directsummodules} and example
      \ref{ex:directsummodules} 
    }
    of 
    kernel and image) and $f^k$ is never zero (but we have $f^p = 0$)
    \footnote{
      ??? By the way $L = \ker f \oplus \Ima f$ holds if $f$ is projection i.e.
      $f^2 = f$ i.e. $f^k = f \ne 0$.
    }.

    So as soon as $K$ is in the image of $f$ then $\exists \alpha \in L$
    such that $f\left(\alpha\right) = 1$
    \footnote{
      This is because $1 \in K$ but
      $K \subset \ker\left(f\right) \subset \Ima\left(f\right)$ therefore
      $1 \in Im\left(f\right)$.
    }
    but this means that
    $\sigma\left(\alpha\right) = \alpha + 1$. Now consider
    $\sigma\left(\alpha^p - \alpha\right) =
    \left(\alpha + 1\right)^p - \left(\alpha + 1\right) = \alpha^p -
    \alpha$ (because we are in the field of characteristic $p$). This
    means that $\alpha^p - \alpha \in K$ because the field is fixed by
    Galois group (see (\ref{eq:lec5_2})). So
    $\alpha^p - \alpha =  a \in K$ and $\alpha$ is a root of $X^p - X
    - a$ and this finished the proof of the theorem.
  \end{proof}
  \label{thm:lec7_1}
\end{theorem}

\section{Composite extensions. Properties}

\begin{definition}[Composite extension]
  Let $L_1$ and $L_2$ are extensions of $K$ both contained in some
  extension $L$ (for instance the \nameref{def:algebraicclosure}
  $\bar{K}$). The composite extension $L_1 L_2$ is the extension they
  generate: $L_1 L_2 = L_2 L_1 = K\left(L_1 \cup L_2\right)$. I.e. the
  composite extension is the smallest extension that contains both
  $L_1$ and $L_2$.
  \label{def:compositeextension}
\end{definition}

Another way to view this: consider the tensor product $L_1 \otimes_K
L_2$ - there is a $K$-algebra. By \nameref{def:universalproperty}
there is a map from the tensor product to $L$:
\(
j: L_1 \otimes_K L_2 \to L  
\)
such that $j\left(l_1 \otimes l_2\right) = l_1 l_2$
\footnote{
  By the \nameref{def:universalproperty} $j$ is a 
  \nameref{def:homomorphism}, but by lemma
  \ref{lem:lec1_homomorphism_is_injection} the homomorphism is
  injection.
}

  \begin{tikzpicture}[description/.style={fill=white,inner sep=2pt}]
    \matrix (m) [matrix of math nodes, row sep=3em,
      column sep=2.5em, text height=1.5ex, text depth=0.25ex]
            { L_1 \times L_2 & & L \\
              & L_1 \otimes_K L_2 & \\ };
            %\draw[double,double distance=5pt] (m-1-1) – (m-1-3);
            \path[->]
            (m-1-1) edge node[auto] {$ f: (l_1,l_2) \to l_1 l_2 $} (m-1-3)
            edge node[auto] {$ \phi $} (m-2-2)
            (m-2-2) edge node[auto] {$ \tilde{f} = j $} (m-1-3);
  \end{tikzpicture}

The 
\nameref{def:image} $\Ima j$ is a sub-algebra of $L$. If $L$ is
algebraic then any sub algebra is a sub field (see proposition
\ref{prop:lec1_algebraicsubalgebra}) and this 
is exactly the field generated by $L_1 L_2$. In general we can take
its fraction field (to obtain a field from a ring (an algebra)) but in
our case, as it was mentioned above, as soon as $L$ is algebraic then
$L_{1,2}$ are fields.

\begin{property}
  If $L_1$ is separable (pure inseparable, normal, finite of
  degree $n$) over $K$ then $L_1 L_2$ is also separable (pure
  inseparable, normal, finite of degree $ \le n$) over $L_2$
  \begin{proof}
    Let $x \in L_1$ ($L_1 L_2$ is generated by $L_1$ over $L_2$)
    \footnote{
      We have $K \subset L_2 \subset L_1 L_2$ as the case there
    }
    then it's minimal polynomial $P_{min}\left(x, L_2\right)$ is a
    divisor of $P_{min}\left(x, K\right)$ in
    $L_2\left[X\right]$ (see proposition \ref{prop:lec1_algebraic}). 
    Therefore $P_{min}\left(x, L_2\right)$ has a
    degree $\le n$ where $n$ 
    is degree of $P_{min}\left(x, K\right)$.

    So if $P_{min}\left(x, K\right)$ is separable (pure inseparable)
    then $P_{min}\left(x, L_2\right)$ is separable (pure inseparable).
    \footnote{
      From proposition \ref{prop:lec1_algebraic}) we know that
      $P_{min}\left(x, L_2\right)$ is a divisor of $P_{min}\left(x,
      K\right)$. Therefore if $P_{min}\left(x, K\right)$ does not have
      multiply roots then $P_{min}\left(x, L_2\right)$ (its divisor)
      will also have only non-multiply roots.

      The inseparability is obvious because if the
      $P_{min}\left(x,K\right)$ has only one root the same will be the
      truth for $P_{min}\left(x, L\right)$.
    }

    The same is true for splitting so the normality is preserved.
    \footnote{
      i.e. the polynomial splits in $L_2$ if it splits in $K \supset L_2$
    }

    About dimensions (``finite extension of degree'' in the property
    formulation). By the \nameref{thm:basechange}
    \footnote{
      From \nameref{thm:basechange} one can get
      \[
      \left|Hom_K\left(L_1, \bar{K}\right)\right| =
      \left|Hom_{L_2}\left(L_2 \otimes_K L_1, \bar{K}\right)\right|
      \]
      But proposition \ref{prop:lec3_2}
      says that
      \[
      \left|Hom_K\left(L_1, \bar{K}\right)\right| =
      \deg\left(P_{min}\left(x,K\right)\right) =
      \left[L_1 : K\right] = 
      \dim_K\left(L_1\right)
      \]
      and
      \[
      \left|Hom_{L_2}\left(L_2 \otimes_K L_1, \bar{K}\right)\right| =
      \left[L_2 \otimes_K L_1 : L_2\right] = 
      \dim_{L_2}\left(L_1 \otimes_K L_2 \right)
      \]
    }: 
    \[
    \dim_K L_1 = \dim_{L_2}\left(L_1 \otimes_K L_2\right)
    \]
    and as soon as $L_1 L_2$ is the $\Ima j$:
    \footnote{
      Using \nameref{thm:ranknullity} one can conclude that for
      $j: L_1 \otimes_K L_2 \to L_1L_2$,
      $\dim{\Ima j} \le \dim{L_1\otimes_K L_2}$
      (the equal sign is when $\dim{\ker{j}} = 0$).
    }
    \[
    \dim_{L_2}\left(L_1 \otimes_K L_2\right) \ge
    \dim_{L_2}\left(L_1 L_2\right)
    \]
    i.e.
    \[
    \dim_{L_2}\left(L_1 L_2\right) \le \dim_K L_1 = n.
    \]
  \end{proof}
  \label{property:lec7_1}
\end{property}

\begin{property}
  If $L_1, L_2$ are separable (pure inseparable, normal, finite of
  degree $n$ and $m$) over $K$ then $L_1 L_2$ is also separable (pure
  inseparable, normal, finite of degree $ \le n m $) over $K$
  \begin{proof}
    We have the following towers:
    \[
    K \hookrightarrow L_1 \hookrightarrow L_1 L_2
    \]
    and all properties except normality are preserved in the towers
    so
    follows from property \ref{property:lec7_1}.
    \footnote{
      From property \ref{property:lec7_1} follows that if $L_1$ is
      separable over $K$ then $L_1 L_2$ is separable over 
      $L_2$. If $L_2$ is separable over $K$ then theorem
      \ref{thm:lec3_3} about separability says the $L_1 L_2$ is
      separable over $K$ as soon as $K \subset L_2 \subset L_1 L_2$.

      About degrees: if $n = \left[L_1 : K\right]$ and
      $m = \left[L_2 : K\right]$ then from property
      \ref{property:lec7_1} follows that
      $\left[L_1 L_2 : L_2\right] \le n$. Using theorem
      \ref{thm:mulformuladegrees}
      $\left[L_1 L_2 : K\right]  = \left[L_1 L_2 : L_2\right] \left[
        L_2 : K\right]\le nm$.
    }

    
    Normality is obvious because if $L_1$ is a splitting field of the
    family polynomials $\left\{P_i\right\}_{i \in I}$
    and $L_2$ is a splitting field of the
    family polynomials $\left\{Q_j\right\}_{j \in J}$ then
    $L_1 L_2$ is a splitting field of the union of those families
    $\left\{P_i, Q_j\right\}_{i \in I, j \in J}$. So normality is
    obviously preserved.  
  \end{proof}
  \label{property:lec7_2}
\end{property}


\section{Linearly disjoint extensions. Examples}

\begin{theorem}
  The following statements are equivalent (for algebraic extensions)
  \begin{enumerate}
  \item $L_1 \otimes_K L_2$ is a field \label{thm:lec7_2_a}
  \item $j$ is \nameref{def:injection} \label{thm:lec7_2_b}
  \item if we have $x_1, x_2, \dots, x_n \in L_1$ linearly independent
    over $K$ then they are linearly independent
    over $L_2$ \label{thm:lec7_2_c}
  \item if we have two families: $x_1, x_2, \dots, x_n \in L_1$ linearly independent
    over $K$ and  $y_1, y_2, \dots, y_m \in L_2$ linearly independent
    over $K$ then $x_i y_j$ are also linearly independent over $K$
    \label{thm:lec7_2_d}
  \end{enumerate}
  When $L_1$ finite over $K$ then all the statements are equivalent to
  $\left[L_1 L_2 : L_2\right] = \left[L_1 : K\right]$ or in other
  words
  $\left[L_1 L_2 : K\right] = \left[L_1 : K\right] \left[L_2 :
    K\right]$
  \footnote{
    Using theorem \ref{thm:mulformuladegrees} we have $\left[L_1 L_2 :
      K\right]  = \left[L_1 L_2 : L_2\right] \left[L_2 : K\right]$.
  }
  
  \begin{definition}[Linearly disjoint extensions]
    In the case $L_1$ and $L_2$ are called linearly disjoint extensions
    \label{def:linearlydisjoint}
  \end{definition}
  \begin{proof}
    Equivalence \ref{thm:lec7_2_a} and \ref{thm:lec7_2_b} is clear
    because we have that $L_1 L_2 = \Ima j$.
    \footnote{
      By the \nameref{def:universalproperty} $j$ is a 
      \nameref{def:homomorphism}, but by lemma
      \ref{lem:lec1_homomorphism_is_injection} the homomorphism is
      injection if $L_1 \otimes_K L_2$ is a field.

      If $j$ is injection then from the fact $L_1 L_2 = \Ima j$ we can
      conclude that for any $x \in L_1 \otimes_K L_2$ such that $x \ne
      0$ there exists
      $y \in L_1 \otimes_K L_2$ such that $j(x) j(y) = 1$ as soon as
      $L_1 L_2 = K\left(L_1 \cup L_2\right)$ is a field
      (both $L_1$ and $L_2$ are algebraic). Therefore
      $x y = 1$ and for any non zero element of $L_1 \otimes_K L_2$ we
      can the the inverse one. This means that $L_1 \otimes_K L_2$ is
      a field.
    }

    Then \ref{thm:lec7_2_b} implies \ref{thm:lec7_2_c}: we have
    $x_1 \otimes 1, \dots, x_n \otimes 1$ are linearly independent
    over $L_2$ by base change property
    (see proposition \ref{prop:lec4_Addon}).
    If $j$ is injective then their images $x_1, \dots, x_n$
    are also linearly independent over $L_2$. This is because an
    injective map transforms a linearly independent set of vectors
    into a linearly independent set.
    \footnote{
      Let $x_1, \dots, x_n$ are not linearly independent over $L_2$
      then there exists $\alpha_1, \dots, \alpha_n \in L_2$ such that
      $\exists \alpha_k \ne 0$ but $\sum_{i=1}^n \alpha_i x_i =
      0$. From other side $x_i = j\left(x_i \otimes 1\right)$
      therefore $j\left(\sum_{i=1}^n \alpha_i x_i \otimes 1 \right) =
      0$ but $\sum_{i=1}^n \alpha_i x_i \otimes 1 \ne 0$ as soon as
      $x_1 \otimes 1, \dots, x_n \otimes 1$ are linearly independent
      over $L_2$. Therefore we just got a contradiction: $j$ cannot be
      injection. 
    }

    \ref{thm:lec7_2_c} implies \ref{thm:lec7_2_d}: if we have some
    relation $\sum_{i,j} a_{ij} x_i y_j = 0, a_{ij} \in K$ then since $x_i$
    linearly independent over $K$ one can get $\sum_{j} a_{ij} y_j = 0$
    but as soon as $y_j$ linearly independent we will get $a_{ij} =
    0$.

    Next \ref{thm:lec7_2_d} implies \ref{thm:lec7_2_b} (remember that
    \ref{thm:lec7_2_b} is injectivity of $j$). Take
    $z \in L_1 \otimes_K L_2$ such that $j\left(z\right) = 0$. We have
    $z = \sum a_{ij} x_i \otimes y_j$ and
    $j\left(z\right) = \sum a_{ij} x_i y_j = 0$ i.e. $a_{ij} = 0$ and
    therefore $z = 0$. I.e. $j$ is \nameref{def:injection}.

    The part about finite degrees follows from the 4 properties.
    \footnote{
      Let $L_1$ and $L_2$ are linearly disjoint extensions over $K$
      with finite degree: $\left[L_1:K\right] = n, \left[L_2:K\right]
      = m$. From \ref{thm:lec7_2_c} we can conclude that
      $\left[L_1 L_2:L_2\right] = n$ and using theorem
      \ref{thm:mulformuladegrees} we will obtain 
      $\left[L_1 L_2:K\right] = \left[L_1 L_2:L_2\right]
      \left[L_2 : K\right] = nm$.
    }
  \end{proof}
  \label{thm:lec7_2}
\end{theorem}

\begin{example}
  First of all, the extensions which have relatively prime degrees are
  always linearly disjoint.

  I.e. if $\left[L_1 : K\right] = n$, $\left[L_2 : K\right] = m$ and
  $\left(m, n\right) = 1$ then $L_1$ and $L_2$ are linearly
  disjoint. Indeed $m$ and $n$ must divide $\left[L_1 L_2 : K\right]
  \le m n$. With our conditions $\left[L_1 L_2 : K\right] = m n$ but
  this is one of definition of linearly disjoint extensions
  (see definition \ref{def:linearlydisjoint}).
  \footnote{
    The claim is the following: if $\left(m, n\right) = 1$ then $L_1$
    and $L_2$ are linearly disjoint. From
    property \ref{property:lec7_2} $\left[L_1 L_2 : K\right] \le m
    n$. But from theorem 
    \ref{thm:mulformuladegrees} $n \mid \left[L_1 L_2 : K\right]$, as
    soon as $K \subset L_1 \subset L_1 L_2$, and
    $m \mid \left[L_1 L_2 : K\right]$, as
    soon as $K \subset L_2 \subset L_1 L_2$ and therefore using
    $\left(m, n\right) = 1$ one can conclude that $\left[L_1 L_2 :
      K\right] = nm$ and thus $L_1$  and $L_2$ are linearly
    disjoint by last statement of theorem \ref{thm:lec7_2}. 
  }

  In particular $\mathbb{Q}\left(\sqrt[5]{2}\right)$ and
  $\mathbb{Q}\left(\sqrt[5]{1}\right)$ are linearly disjoint
  extensions because the degrees are
  $\left[\mathbb{Q}\left(\sqrt[5]{2}\right): \mathbb{Q}\right] = 5$
  and
  $\left[\mathbb{Q}\left(\sqrt[5]{1}\right): \mathbb{Q}\right] = 4$.
  \footnote{
    $P_{min}\left(\sqrt[5]{2}, \mathbb{Q}\right) = X^5 - 2$ and
    $\deg\left(P_{min}\left(\sqrt[5]{2}, \mathbb{Q}\right)\right) =
    5$.

    $P_{min}\left(\sqrt[5]{1}, \mathbb{Q}\right) = X^4 + X^3 + X^2 + X
    + 1$ and
    $\deg\left(P_{min}\left(\sqrt[5]{1}, \mathbb{Q}\right)\right) =
    4$.
  }

  From the other side with $\sqrt[5]{1} = e^{\frac{2 \pi i}{5}}$ the
  following extensions are not linearly disjoint:
  $\mathbb{Q}\left(\sqrt[5]{2}\right)$ and
  $\mathbb{Q}\left(e^{\frac{2 \pi i}{5}} \cdot
  \sqrt[5]{2}\right)$.
  Indeed in both cases $L_1 L_2$ is a splitting
  field of $X^5 - 2$ and (for the first case) $\left[L_1 L_2 :
    \mathbb{Q}\right] = 4 \cdot 5 = 20$. In the second case both
  $\left[L_{1,2}:\mathbb{Q}\right] = 5$ and $5 \cdot 5 \ne 20$.

   So, you see that the difference is rather subtle. Well, the obvious
   reason in the second case is that those extensions are generated by
   rules of the same polynomial, but, still some effort is needed to
   formalize why this is not linearly disjoint case.
   In particular we see that $L_1 \cap L_2 = \mathbb{Q}$ does not
   imply that $L_1$ and $L_2$ are linearly disjoint over
   $\mathbb{Q}$. It's exactly what's happen in the second case. 
\end{example}

\section{Linearly disjoint extensions in the Galois case}

\begin{theorem}
  Let $L_1, L_2 \subset \bar{K}$ - extensions of $K$. $L_1$ is
  \nameref{def:galoisextension} 
  over $K$. Let $K' = L_1 \cap L_2$. Then $L_1 L_2$ is Galois over
  $L_2$.
  $Gal\left(L_1 L_2/ L_2\right)$ stabilizes $L_1$. $\phi: g \to
  \left.g\right|_{L_1}$ is an injective map of
  $Gal\left(L_1 L_2/ L_2\right) \to Gal\left(L_1/ K\right)$ with
  image $Gal\left(L_1/K'\right)$ and $L_1, L_2$ are linearly disjoint
  over $K'$.
  \begin{proof}
    The proof that $L_1 L_2$ is Galois over $L_2$ is obvious.
    \footnote{
      We have $L_1$ is \nameref{def:galoisextension} and therefore
      normal and separable over $K$. By property \ref{property:lec7_1}
      this means that $L_1 L_2$ is normal and separable over $L_2$ or in
      other words $L_1 L_2$ is Galois over $L_2$.
    }

    The next statement is that $Gal\left(L_1 L_2/ L_2\right)$
    stabilizes
    $L_1$.
    \footnote{
      I.e. $\forall g \in Gal\left(L_1 L_2/ L_2\right)$  and for
      any $x \in L_1$ we have $g(x) \in L_1$ or in other words $g(L_1)
      = L_1$ (see also definition \ref{def:stabilizersubgroup}).
    }
    Let $x \in L_1$ and $g \in Gal\left(L_1 L_2/
    L_2\right)$ then 
    $g\left(x\right)$ is a root of $P_{min}\left(x, L_2\right)$.
    \footnote{
      This is because $g$ permutes roots and the image of the
      permutation is the set of roots of the following polynomial
      $P_{min}\left(x, L_2\right)$. Note that $x$ is also one of the
      roots.  
    }
    It is
    also a root of $P_{min}\left(x, K\right)$.
    \footnote{
      This is because proposition \ref{prop:lec1_algebraic} i.e. because
       $P_{min}\left(x, L_2\right)$ divides  $P_{min}\left(x,
      K\right)$ i.e. all roots of $P_{min}\left(x, L_2\right)$ are
      also roots of $P_{min}\left(x, K\right)$.
    }
    But all such roots are
    in $L_1$ because $L_1$ is a \nameref{def:galoisextension}. 

    Therefore the map $\phi$ is well
    defined. The map is injective because if we have some $\sigma$ such that
    $\left. \sigma \right|_{L_1} = \left. \sigma \right|_{L_2} = id$
    then it should be $\sigma = id$.
    This is because our extension is
    generated by $L_1$ and $L_2$, so if it happened to be in identity
    on them both, it must be an identity.
    \footnote{
      We have that $\phi(id) = id$ i.e. $\phi$ is
      \nameref{def:injection}. Because if $g_1, g_2 \ne id$ then
      the equality $\phi(g_1) = \phi(g_2)$ holds if $g_1 g_2^{-1} =
      id$ i.e. if $g_1 = g_2$ that is accordingly with the injectivity
      definition.
    }

    So now lets find the image of $\phi$. If $g\left(x\right) = x,
    \forall g \in Gal\left(L_1 L_2/ L_2\right)$ then $x \in L_2$ by
    \nameref{thm:galoiscorrespondence}. So if also $x \in L_1$ then it
    should be $x \in K' = L_1 \cap \L_2$. So if $L_1$ is finite over
    $K$ then by   \nameref{thm:galoiscorrespondence} we can conclude
    that $\Ima \phi = Gal\left(L/K'\right)$ because the
    \nameref{def:fixedfield} is $K'$.

    In general (??? not finite $L_1$ over $K$) we have to find finite
    sub-extension of $L_1$: let denote it as $L_1'$. We also have a
    finite Galois sub extension of $L_1$ that contains $L_1'$.
    You can take the union of all images of $L_1'$ by all
    automorphisms. And this will be a finite union since $L_1'$ was
    finite, so there are finitely many possible roots of minimal
    polynomials so there are not really many possibilities for the
    images of this $L_1'$. So I shall leave it as an exercise (??? add
    proof), but the
    solution is more or less what I just have told you.

    Lets denote the Galois sub-extension as $L_1''$. We have
    $\forall L_1''$ and $L_2$ are linearly disjoint over $K'$ then it follows
    that $L_1$ and $L_2$ are also linearly disjoint over $K'$
    (see theorem \ref{thm:lec7_2} point \ref{thm:lec7_2_c}).

    Let $\gamma \in Gal\left(L_1/K\right)$ then exists an element in
    $Gal\left(L_1 L_2/L_2\right)$ which is sent by $\phi$ to $\gamma$.
    We have $j: L_1 \otimes_K L_2 \cong L_1 L_2$ and we can take
    $j \cdot \left(\gamma \otimes id\right) \cdot j^{-1}$ - this will
    be the element of Galois group (??? required element in
    $Gal\left(L_1 L_2/L_2\right)$). 
  \end{proof}
  \label{thm:lec7_3}
\end{theorem}


From the theorem \ref{thm:lec7_3} follows the following proposition
\begin{proposition}
  \begin{enumerate}
  \item $L_1$ and $L_2$ are both Galois over $K$ and linearly disjoint then
    the following map $g \to \left(\left. g \right|_{L_1}, \left. g
    \right|_{L_2}\right)$ defines the isomorphism
    \[
    Gal\left(L_1 L_2/K\right) \cong Gal\left(L_1/K\right) \times Gal\left(L_2/K\right)
    \]
  \item conversely to the first part: if
    $Gal\left(L/K\right) = G_1 \times G_2$ then $L = L^{G_1} L^{G_2}$
    which are linearly disjoint over the intersection.
    \begin{proof}
      The first part Is very sure because the interjectivity of this
      map is clear: if something is trivial both on $L_1$ and $L_2$,
      then  it's trivial on the composite, so I only have to prove the
      subjectivity.  I will use the same trick as before:
      $L_1 \otimes_K L_2 \cong_j L_1 L_2$ then
      $j \cdot \left(g_1 \otimes g_2\right) \cdot j^{-1}$ goes to
      $\left(g_1, g_2\right)$.

      The second part. $L^{G_1}$ and $L^{G_2}$ are both Galois because
      $G_1$ and $G_2$ are normal in the product:
      $G_1, G_2 \triangleleft G_1 \times G_2$. What I mean is $G_1$
      embedded to the product by identifying it with $G_1 \times e$
      where $e$ is the neutral element of $G_2$.

      The intersection $L^{G_1} \cap L^{G_2}$ is fixed by $G$ so
      $L^{G_1} \cap L^{G_2} = K$. Linear disjoint follows from
      $L^{G_1} \cap L^{G_2} = K$ since we are in the Galois case
      (???). 
    \end{proof}
  \end{enumerate}
  \label{prop:lec7_3}
\end{proposition}


\section{On the Galois group of the composite}

Let me give you a small example:
\begin{example}
  We have a \nameref{def:compositeextension}
  $\mathbb{Q}\left(\zeta_n\right) \mathbb{Q}\left(\zeta_m\right) =
  \mathbb{Q}\left(\zeta_n, \zeta_m\right)$ where
  $\zeta_n = e^{\frac{2 \pi i}{n}}$.
  $\mathbb{Q}\left(\zeta_n, \zeta_m\right) =
  \mathbb{Q}\left(\zeta_{LCM\left(n,m\right)}\right)$
  \footnote{
    LCM - least common multiple. For instance multiples for 4 are
    $4,8,12, \dots$. Multiples for 6 are $6,12,18, \dots$. Thus
    $LCM\left(4,6\right) = 12$.
  } therefore if $\left(n,m\right) = 1$ then
  $\mathbb{Q}\left(\zeta_n\right)$ and
  $\mathbb{Q}\left(\zeta_m\right)$ are linearly disjoint. It can be
  seen as follows: we can apply proposition \ref{prop:lec7_3} to our
  Galois groups then
  $\mathbb{Q}\left(\zeta_n, \zeta_m\right) =
  \mathbb{Q}\left(\zeta_{nm}\right)$ but by the
  \nameref{thm:chineseremainder}
  \[
  \left(\mathbb{Z}/nm\mathbb{Z}\right)^\times \cong
  \left(\mathbb{Z}/n\mathbb{Z}\right)^\times \times
  \left(\mathbb{Z}/m\mathbb{Z}\right)^\times. 
  \]
  Thus $Gal\left(\mathbb{Q}\left(\zeta_{nm}\right)\right) =
  Gal\left(\mathbb{Q}\left(\zeta_{n}\right)\right) \times
  Gal\left(\mathbb{Q}\left(\zeta_{m}\right)\right)$. So the linear
  disjoint is just got from the proposition \ref{prop:lec7_3}.
\end{example}
