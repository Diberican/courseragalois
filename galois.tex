%% -*- coding:utf-8 -*- 
\documentclass[12pt,a4paper]{book}
\usepackage[toc,page]{appendix}
\usepackage[utf8]{inputenc}
\usepackage[english]{babel}
\usepackage{amsmath}
\usepackage{amsthm}
\usepackage{amssymb}
\usepackage{array}
\usepackage{booktabs}
\usepackage[breaklinks]{hyperref}
\usepackage{tikz}
\usetikzlibrary{matrix,arrows}
\usepackage{mathtools}

\newtheorem{theorem}{Theorem}[chapter]
\newtheorem{definition}{Definition}[chapter]
\newtheorem{example}{Example}[chapter]
\newtheorem{lemma}{Lemma}[chapter]
\newtheorem{claim}{Claim}[chapter]
\newtheorem{proposition}{Proposition}[chapter]
\newtheorem{corollary}{Corollary}[chapter]
\newtheorem{property}{Property}[chapter]
\newtheorem{remark}{Remark}[chapter]

\title{Introduction to Galois Theory}
\author{}
%\date{}
\begin{document}
\maketitle
\tableofcontents

\input ./introduction.tex
\input ./lecture1.tex
\input ./lecture2.tex
\input ./lecture3.tex
\input ./lecture4.tex
\input ./lecture5.tex
\input ./lecture6.tex
%\input ./lecture7.tex
%\input ./lecture8.tex
%\input ./lecture9.tex
\input ./requirements.tex

\bibliographystyle{gost780s}
\bibliography{galois}

\end{document}
