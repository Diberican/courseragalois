%% -*- coding:utf-8 -*-
\chapter{Stem field, splitting field, algebraic closure}
We introduce the notion of a stem field and a splitting field (of a
polynomial). Using Zorn's lemma, we construct the algebraic closure of
a field and deduce its unicity (up to an isomorphism) from the theorem
on extension of homomorphisms.

\section{Stem field. Some irreducibility criteria}

\subsection{Stem field}

\begin{definition}[Stem field]
Let $P \in K\left[X\right]$ is an irreducible
\nameref{def:monicpolynomial}. \nameref{def:fextension1} E is a stem
field of $P$ if $\exists \alpha \in E$ - the root of polynomial
$P$ and $E = K\left[\alpha\right]$.
\label{def:stemfield}
\end{definition}

Such things exist, for instance we can take
$K\left[X\right]/\left(P\right)$. It is a field because $P$ is
irreducible moreover the root of the $P$ is in the field (see example
\ref{ex:F2overP}).

We also can say that for any stem field $E$:
\[
K\left[X\right]/\left(P\right) \cong E.
\]
We can use the following \nameref{def:isomorphism}:
$f: \forall p \in K\left[X\right]/\left(P\right) \rightarrow
p(\alpha)$, there $\alpha$ is a root of polynomial $P$.
To summarize we have the following
\begin{proposition}[About stem field existence]
  The stem field exist and if we have 2 stem fields $E$ and $E'$ which
  correspond 2 roots of $P$: $E = K\left[\alpha\right]$,
  $E' = K\left[\alpha'\right]$ then $\exists! f: E \cong E'$
  (\nameref{def:isomorphism} of K-algebras) such that $f(\alpha) =
  \alpha'$. 
  \begin{proof}
    Existence: $K\left[X\right]/\left(P\right)$ can be took as the
    stem field.

    Uniquest of the \nameref{def:isomorphism} is easy because it is
    defined by it's value on argument $\alpha$:
    \begin{eqnarray}
      \phi: K\left[X\right]/\left(P\right) \cong_{x \to \alpha} E,
      \nonumber \\
      \psi: K\left[X\right]/\left(P\right) \cong_{x \to \alpha'} E',
      \nonumber
    \end{eqnarray}
    thus
    \[
    \phi^{-1} \circ \psi: E \cong_{\alpha \to \alpha'} E'.
    \]
  \end{proof}
  \label{prop:stemfield}
\end{proposition}

\begin{remark}
  \begin{enumerate}
  \item In particular: If a stem field contains 2 roots of $P$ then
    $\exists!$ \nameref{def:automorphism} taking one root into
    another.
  \item If $E$ stem field then $\left[E:K\right] = deg P$
  \item If $\left[E:K\right] = deg P$ and $E$ contains a root of $P$
    then $E$ is a stem field
  \item If $E$ is not a stem field but contains root of $P$ then
    $\left[E:K\right] > deg P$ (???)
  \end{enumerate}
\end{remark}

\subsection{Some irreducibility criteria}

\begin{corollary}
  $P \in K\left[X\right]$ is irreducible over $K$ if and only if it
  does not have a root in \nameref{def:fextension1} $L$ of $K$ of such that
  $\left[L:K\right] \le \frac{n}{2}$, where $n = deg P$.
  \label{cor:lec2_1}
  \begin{proof}
    $\Rightarrow$: If $P$ is not irreducible then it has a polynomial $Q$ that
    divides $P$ and $deg Q \le \frac{n}{2}$ ($P = RQ$ and if
    $deg Q > \frac{n}{2}$ then we can take $R$ as $Q$). The
    \nameref{def:stemfield} $L$ for $Q$ exists and it's degree is $deg Q
    \le \frac{n}{2}$. $L$ should have root of $Q$ (as soon as root of
    $P$) by definition.

    $\Leftarrow$: If $P$ has a root $\alpha$ in $L$ then $\exists
    P_{min}\left(\alpha, K\right)$ with degree
    $\le \frac{n}{2} < n$ (because $\left[L:K\right] \le \frac{n}{2}$)
    that divides $P$.
  \end{proof}
\end{corollary}

\begin{corollary}
  $P \in K\left[X\right]$ irreducible with $deg P = n$. Let $L$ be an
  extension of $K$ such that $\left[L:K\right] = m$.
  If $gcd\left(n,m\right) = 1$ then $P$ is irreducible over $L$.
  \label{cor:lec2_2}
  \begin{proof}
    If it is not a case and $\exists Q$ such that $Q \mid P$ in
    $L\left[X\right]$. Let $M$ be a \nameref{def:stemfield} of $Q$
    over $L$.

    So we have $K \subset L \subset M = L\left(\alpha\right)$. $M$ is
    a stem field that $\left[M:L\right] = deg Q = d < n$. Thus
    $\left[M:L\right] = m d$

    Lets $K\left(\alpha\right)$ is a stem field of $P$ over $K$ then
    $\left[K\left(\alpha\right):K\right] = deg P = n$.

    $K\left(\alpha\right) \subseteq M$ and therefore $n \mid md$ thus
    using $gcd(m,n)=1$ one can get that $n \mid d$ but this is
    impossible because $d < n$.
  \end{proof}
\end{corollary}


\section{Splitting field}

\section{An example. Algebraic closure}

\section{Algebraic closure (continued)}

\section{Extension of homomorphisms. Uniqueness of algebraic closure}
