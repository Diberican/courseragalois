%% -*- coding:utf-8 -*-
\chapter{Stem field, splitting field, algebraic closure}
We introduce the notion of a stem field and a splitting field (of a
polynomial). Using Zorn's lemma, we construct the algebraic closure of
a field and deduce its unicity (up to an isomorphism) from the theorem
on extension of homomorphisms.

\section{Stem field. Some irreducibility criteria}

\subsection{Stem field}

\begin{definition}[Stem field]
Let $P \in K\left[X\right]$ is an irreducible
\nameref{def:monicpolynomial}. \nameref{def:fextension1} E is a stem
field of $P$ if $\exists \alpha \in E$ - the root of polynomial
$P$ and $E = K\left[\alpha\right]$.
\label{def:stemfield}
\end{definition}

Such things exist, for instance we can take
$K\left[X\right]/\left(P\right)$. It is a field because $P$ is
irreducible moreover the root of the $P$ is in the field (see example
\ref{ex:F2overP}).

We also can say that for any stem field $E$:
\[
K\left[X\right]/\left(P\right) \cong E.
\]
We can use the following \nameref{def:isomorphism}:
$f: \forall p \in K\left[X\right]/\left(P\right) \rightarrow
p(\alpha)$, there $\alpha$ is a root of polynomial $P$.
To summarize we have the following
\begin{proposition}[About stem field existence]
  The stem field exist and if we have 2 stem fields $E$ and $E'$ which
  correspond 2 roots of $P$: $E = K\left[\alpha\right]$,
  $E' = K\left[\alpha'\right]$ then $\exists! f: E \cong E'$
  (\nameref{def:isomorphism} of K-algebras) such that $f(\alpha) =
  \alpha'$. 
  \begin{proof}
    Existence: $K\left[X\right]/\left(P\right)$ can be took as the
    stem field.

    Uniquest of the \nameref{def:isomorphism} is easy because it is
    defined by it's value on argument $\alpha$:
    \begin{eqnarray}
      \phi: K\left[X\right]/\left(P\right) \cong_{x \to \alpha} E,
      \nonumber \\
      \psi: K\left[X\right]/\left(P\right) \cong_{x \to \alpha'} E',
      \nonumber
    \end{eqnarray}
    thus
    \[
    \phi^{-1} \circ \psi: E \cong_{\alpha \to \alpha'} E'.
    \]
  \end{proof}
  \label{prop:stemfield}
\end{proposition}

\begin{remark}[About stem field]
  \begin{enumerate}
  \item In particular: If a stem field contains 2 roots of $P$ then
    $\exists!$ \nameref{def:automorphism} taking one root into
    another.
  \item If $E$ stem field then $\left[E:K\right] = deg P$
  \item If $\left[E:K\right] = deg P$ and $E$ contains a root of $P$
    then $E$ is a stem field
  \item If $E$ is not a stem field but contains root of $P$ then
    $\left[E:K\right] > deg P$ (???)
  \end{enumerate}
  \label{rem:lec2_1}
\end{remark}

\subsection{Some irreducibility criteria}

\begin{corollary}
  $P \in K\left[X\right]$ is irreducible over $K$ if and only if it
  does not have a root in \nameref{def:fextension1} $L$ of $K$ of such that
  $\left[L:K\right] \le \frac{n}{2}$, where $n = deg P$.
  \label{cor:lec2_1}
  \begin{proof}
    $\Rightarrow$: If $P$ is not irreducible then it has a polynomial $Q$ that
    divides $P$ and $deg Q \le \frac{n}{2}$ ($P = RQ$ and if
    $deg Q > \frac{n}{2}$ then we can take $R$ as $Q$). The
    \nameref{def:stemfield} $L$ for $Q$ exists and it's degree is $deg Q
    \le \frac{n}{2}$. $L$ should have root of $Q$ (as soon as root of
    $P$) by definition.

    $\Leftarrow$: If $P$ has a root $\alpha$ in $L$ then $\exists
    P_{min}\left(\alpha, K\right)$ with degree
    $\le \frac{n}{2} < n$ (because $\left[L:K\right] \le \frac{n}{2}$)
    that divides $P$ (see lemma \ref{lem:minpolynomial}) i.e. $P$
    become reducible. 
  \end{proof}
\end{corollary}

\begin{corollary}
  $P \in K\left[X\right]$ irreducible with $deg P = n$. Let $L$ be an
  extension of $K$ such that $\left[L:K\right] = m$.
  If $gcd\left(n,m\right) = 1$ then $P$ is irreducible over $L$.
  \label{cor:lec2_2}
  \begin{proof}
    If it is not a case and $\exists Q$ such that $Q \mid P$ in
    $L\left[X\right]$. Let $M$ be a \nameref{def:stemfield} of $Q$
    over $L$.

    So we have $K \subset L \subset M = L\left(\alpha\right)$. $M$ is
    a stem field that $\left[M:L\right] = deg Q = d < n$. Thus
    $\left[M:L\right] = m d$

    Lets $K\left(\alpha\right)$ is a stem field of $P$ over $K$ then
    $\left[K\left(\alpha\right):K\right] = deg P = n$.

    $K\left(\alpha\right) \subseteq M$ and therefore $n \mid md$ thus
    using $gcd(m,n)=1$ one can get that $n \mid d$ but this is
    impossible because $d < n$.
  \end{proof}
\end{corollary}


\section{Splitting field}

\begin{definition}[Splitting field]
  Let $P \in K\left[X\right]$.
  The splitting field of $P$ over $K$ is an extension $L$ where $P$ is
  split (i.e. is a product of linear factors) and roots of $P$
  generate $L$
  \label{def:splittingfield}
\end{definition}

\begin{theorem}[About splitting fields]
  \begin{enumerate}
    \item Splitting field $L$ exists and $\left[L:K\right] \le d!$,
      where $d = deg P$.
    \item If $L$ and $M$ are 2 splitting fields then
      $\exists \phi: L \cong M$ (an \nameref{def:isomorphism}). But
      the \nameref{def:isomorphism} is not necessary to be unique.
  \end{enumerate}
  \begin{proof}
    Lets prove by induction on $d$. The first case ($d = 1$) is
    trivial the $K$ itself is the splitting field. Now assume $d > 1$
    and that the theorem is valid for any polynomial of degree $< d$
    over any field $K$. Let $Q$ be any irreducible factor of $P$. We
    can create a \nameref{def:stemfield} $L_1 = K\left(\alpha\right)$
    for $Q$ that will be also a \nameref{def:stemfield} for $P$.

    Over $L_1$ we have $P = (x - \alpha) R$, where $R$ is a polynomial
    with $deg R = d - 1$. We know (see remark \ref{rem:lec2_1}) that
    there exists a \nameref{def:splittingfield} $L$ for $R$ over $L_1$
    and its degree:
    \(
    \left[L:L_1\right] \le (d-1)!
    \)
    We have $K \subset L_1 \subset L$. The $L$ will be a splitting
    field for original polynomial $P$. Its degree (by
    \nameref{thm:mulformuladegrees}) is $ \le d \cdot (d-1)! = d!$.

    Uniqueness: Let $L$ and $M$ are 2 splitting fields. Let $\beta$ is
    a root of $Q$ (irreducible factor of $P$) in $M$.
    We have 2 stem fields: $L_1 = K\left(\alpha\right)$ and
    $M_1 = K\left(\beta\right)$. Proposition \ref{prop:stemfield} says
    as that
    \[
    \exists \phi: L_1 = K\left(\alpha\right) \cong
    K\left(\beta\right) = M_1, 
    \]
    such that $\phi(\alpha) = \beta$.

    Over $M_1$ we have $P = (x - \beta) S$, where
    $S = \phi\left(R\right)$
    \footnote{
      We have $\phi: K\left(\alpha\right) \to
      K\left(\beta\right)$. The $\phi: K \to K$ because
      $K \subset K\left(\alpha\right)$ as well as
      $K \subset K\left(\beta\right)$. Therefore
      $\phi\left(P\right) = P$ because $P \in K\left[X\right]$.
      Thus
      \[
      P = (x - \beta) S = \phi\left(P\right) =
      \phi\left((x - \alpha) R\right) =
      (x - \beta) \phi\left(R\right)
      \]
      and $S = \phi\left(R\right)$.
    }
    $M$ is a splitting field for of $S$ over $K\left[\beta\right]$
    i.e. it is a $K\left[\beta\right]$-algebra but it's is also (via
    \nameref{def:isomorphism}) $K\left[\alpha\right]$-algebra and as
    result it's a splitting field for $R$ over $K\left[\alpha\right]$
    and by induction
    \footnote{
      Induction steps are the following: we have a polynomial $P$ with
      $\deg P = n$. We suppose that the isomorphism is proved for
      polynomial with degree $n-1$
    }
    we have $K\left[\alpha\right]$ isomorphism $L \cong M$ and as
    result $K$ isomorphism $L \cong M$.
    \footnote{
      Lukas Heger comment about the prove:
      We can consider 
      another roots: $\alpha_2$ for $R$ and $\beta_2$ for $S$
      and there is an isomorphism between the 2 stem fields
      also. Continue in the way we will get the 2 following chains
      \begin{eqnarray}
        K \subset L_1, \subset L_2 \subset \dots \subset L_n \subset L
        \nonumber \\
        K \subset M_1, \subset M_2 \subset \dots \subset M_n \subset M
        \nonumber
      \end{eqnarray}
      On each step we have an isomorphism between $L_i$ and $M_i$ and as
      result the isomorphism between resulting fields $L$ and $M$ (via
      $\phi$) as $L_n$ algebras and therefore as $K$ algebras. 
    }
    %% $M$ is splitting field for $S$ over
    %% $K\left(\beta\right) = M_1$. $M$ is also $L_1$-algebra (via the
    %% \nameref{def:isomorphism} $\phi$) and as such it's a splitting
    %% field for $R$ over $L_1$. As soon as
    %% $\left[L:L_1\right] = \left[M:M_1\right]$ the $M/L_1 \cong L/L_1$
    %% because the $L_1$-algebras with the same dimension are isomorphic (see lemma
    %% \ref{lem:vsisomorphism}). Therefore we have an
    %% $L_1 = K\left(\alpha\right)$ \nameref{def:isomorphism} $L \cong M$
    %% and therefore $K$ \nameref{def:isomorphism} $L \cong M$.    
  \end{proof}
  \label{thm:lec2_1}
\end{theorem}

\begin{remark}
  The \nameref{def:isomorphism} is not unique. A splitting field  can
  have many \nameref{def:automorphism} and this is in fact the subject
  of Galois theory.
\end{remark}

\section{An example. Algebraic closure}

\subsection{An example of automorphism}

\begin{example}[$x^3-2$ over $\mathbb{Q}$]
  Let we have the following polynomial $x^3-2$ over $\mathbb{Q}$. It
  has the following roots: $\sqrt[3]{2}, j\sqrt[3]{2}$ and
  $j^2\sqrt[3]{2}$, where $j = e^{\frac{2 \pi i}{3}}$. Splitting field
  is the following $L = \mathbb{Q}\left(\sqrt[3]{2}, j\right)$. Lets
  find \nameref{def:automorphism}s of the field.

  \begin{tikzpicture}[descr/.style={fill=white,inner sep=2.5pt}]
    \matrix (m) [matrix of math nodes, row sep=3em,
      column sep=3em]
            { & \mathbb{Q}\left(j\right) & \\
              \mathbb{Q} & & \mathbb{Q}\left(\sqrt[3]{2}, j\right) = L\\
              & \mathbb{Q}\left(\sqrt[3]{2}\right) & \\ };
            \path[->,font=\scriptsize]
            (m-2-1) edge node[descr] {$ 2 $} (m-1-2)
            (m-1-2) edge node[descr] {$ 3 $} (m-2-3)
            (m-2-1) edge node[descr] {$ 3 $} (m-3-2)
            (m-3-2) edge node[descr] {$ 2 $} (m-2-3);
  \end{tikzpicture}

  As soon as $L$ is a stem field for $\mathbb{Q}\left(j\right)$ 
  and for $\mathbb{Q}\left(\sqrt[3]{2}\right)$ then 2 types of
  automorphism exist:
  \begin{enumerate}
    \item $\mathbb{Q}\left(\sqrt[3]{2}\right)$
      \nameref{def:automorphism}. We have
      $x^2+x+1$ as
      $P_{min}\left(j, \mathbb{Q}\left(\sqrt[3]{2}\right)\right)$. The
      polynomial has 2 roots: $j$ and $j^2$ and there is an
      \nameref{def:automorphism} that exchanges the root. Lets call it
      $\tau$
      \item $\mathbb{Q}\left(j\right)$ \nameref{def:automorphism}. In
        this case the automorphism of exchanging $\sqrt[3]{2}$ and
        $j \sqrt[3]{2}$. \footnote{
          ??? The minimal polynomial is $x^3 - 2$ there and thus we
          have 3 roots: $\sqrt[3]{2}$, $j \sqrt[3]{2}$ and $j^2
          \sqrt[3]{2}$ 
        }. Lets call it $\sigma$
  \end{enumerate}

  The group of automorphism of $L$ $Aut\left(L/K\right)$ is embedded
  into permutation group of 3 elements $S_3$ (see example \ref{ex:sngroup}):
  \[
  Aut\left(L/K\right) \hookrightarrow S_3.
  \]
  It's embedded because the automorphism exchanges the roots of
  $x^3-2$. Moreover
  \[
  Aut\left(L/K\right) = S_3,
  \]
  because $\sigma$ and $\tau$ generates $S_3$ because
  \begin{itemize}
  \item $\sigma$: $\sqrt[3]{2} \to j \sqrt[3]{2} \to j^2 \sqrt[3]{2}
    \to \sqrt[3]{2}$. This is a circle.
  \item $\tau$ - it keeps $\sqrt[3]{2}$ and exchanges $j$ and $j^2$:
    $\sqrt[3]{2} j \leftrightarrow \sqrt[3]{2} j^2$ (???). This is a
    transposition. 
  \end{itemize}

  Lets also look at $\mathbb{Q}\left(\sqrt[3]{2}\right)$. The question
  is the following: how many \nameref{def:homomorphism}s to $L =
  \mathbb{Q}\left(\sqrt[3]{2}, j\right)$ do we have. As we know
  \[
  L = \mathbb{Q}\left(\sqrt[3]{2}, j\right) =
  \mathbb{Q}\left(\sqrt[3]{2}, j\sqrt[3]{2}, j^2\sqrt[3]{2}\right),
  \]
  i.e. $\sqrt[3]{2}$ can be switched with one of the roots:
  $\sqrt[3]{2}, j\sqrt[3]{2}, j^2\sqrt[3]{2}$ and each permutation is a
  homomorphism. To demonstrate it lets look at the following
  permutation $\sqrt[3]{2} \leftrightarrow j\sqrt[3]{2}$. We have a
  unique \nameref{def:isomorphism}
  \[
  \mathbb{Q}\left(\sqrt[3]{2}\right) \to
  \mathbb{Q}\left(j\sqrt[3]{2}\right) \subset L.
  \]
  i.e. we have a homomorphism $\mathbb{Q}\left(\sqrt[3]{2}\right) \to
  L$ associated with the following permutation:
  $\sqrt[3]{2} \leftrightarrow j\sqrt[3]{2}$
\end{example}

\subsection{Algebraic closure}

\begin{definition}[Algebraically closed field]
  $K$ is algebraically closed if any non constant polynomial $P \in
  K\left[X\right]$ has a root in $K$ or in other words if any $P \in
  K\left[X\right]$ splits
  \label{def:algebraicallyclosed}
\end{definition}

\begin{example}[$\mathbb{C}$]
  $\mathbb{C}$ is an \nameref{def:algebraicallyclosed}. This will be
  proved later.
\end{example}

\begin{definition}[Algebraic closure]
  An algebraic closure of $K$ is a field $L$ that is
  \nameref{def:algebraicallyclosed} and
  \nameref{def:algebraicextension} 
  over $K$.
  \label{def:algebraicclosure}
\end{definition}

\begin{theorem}[About Algebraic closure]
  Any field $K$ has an \nameref{def:algebraicclosure}
  \begin{proof}
    Lets discuss the strategy of the prove.
    First construct $K_1$ such that $\forall P \in K\left[X\right]$
    has a root in $K_1$. There is not a victory because $K_1$ can
    introduce new coefficients and polynomials that can be irreducible
    over $K_1$. Then construct $K_2$ such that $\forall P \in
    K_1\left[X\right]$ has a root in $K_2$ and so forth. As result we
    will have
    \[
    K \subset K_1 \subset K_2 \subset \dots \subset K_n \subset \dots
    \]
    Take $\bar{K} = \cup_i K_i$ and we claim that $\bar{K}$ is
    algebraically closed. Really
    $\forall P \in \bar{K}\left[X\right]$ $\exists j: P \in
    K_j\left[X\right]$ thus it has a root in $K_{j+1}$ and as result
    in $\bar{K}$.

    Now how can we construct $K_1$. Let $S$ be a set of all
    irreducible $P \in K\left[X\right]$. Let
    $A = K\left[\left(X_p\right)_{p \in S}\right]$ - multi-variable
    (one variable $X_p$ for each $p \in S$) polynomial ring.

    Let $I \subset A$ is an \nameref{def:ideal} generated by $P\left(X_p\right)$
    $\forall p \in S$.\footnote
    {
      $I = \sum_i \lambda_i P_i\left(X_{p_i}\right)$, where $\lambda_i \in A$
    }
    We claim that $I$ is a \nameref{def:properideal} i.e. $I \ne
    A$. If not then we can write
    \begin{equation}
      1_A = \sum_i^n \lambda_i P_i\left(X_{p_i}\right),
      \label{eq:properideal_lec2}
    \end{equation}
    where $\lambda_i \in A$ and the sum is the finite. As soon as the
    sum is finite then I can take the product of the polynomials in
    the sum: $P = \prod_i^n P_i$ and I can create a
    \nameref{def:splittingfield} $L$ for the polynomial $P$ over $K$.
    \footnote{$\alpha_i$ is a root of $P_i$}.

    $A$ is a polynomial ring and it's very easy produce a homomorphism
    between polynomial algebra and any other algebra. Therefore there
    is a homomorphism between rings $A$ and $L$ such that 
    $\phi: A \to L$ where $X_{p_i} \to \alpha_i$ if $P =P_i$ and
    $X_{p_i} \to 0$ otherwise. From (\ref{eq:properideal_lec2}) we
    have
    \[
    \phi(1_A) = \sum_i^n \lambda_i
    \phi\left(P_i\left(X_{p_i}\right)\right) =
    \sum_i^n \lambda_i P_i\left(\alpha_i\right) = 0
    \]
    that is impossible.

    Fact: Any \nameref{def:properideal} $I \subset A$ is contained in
    the \nameref{def:maxideal} $m$ and $A/m$ is a field.

    Thus I can take $K_1 = A/m$ and continue in the same way to
    construct $K_2, K_3, \dots, K_n, \dots$.
  \end{proof}
  \label{thm:lec2_2}
\end{theorem}

\subsection{Ideals in a ring}
The ring is commutative, associative with unity. Any
\nameref{def:properideal} is in a \nameref{def:maxideal}. This is a
consequence of what one calls Zorn's lemma

\begin{definition}[Chain]
  Let $\mathcal{P}$ is a partially ordered set ($\le$ is the order
  relation). $\mathcal{C} \subset \mathcal{P}$ is a chain if
  $\forall \alpha, \beta \in \mathcal{C}$ exists a relation between
  $\alpha$ and $\beta$ i.e. $\alpha \le \beta$ or $\beta \le \alpha$.
  \label{def:chain}
\end{definition}

\begin{lemma}[Zorn]
  If any non-empty \nameref{def:chain} $\mathcal{C}$ in a non-empty set
  $\mathcal{P}$ has an upper bound (that is $M \in \mathcal{P}$ such
  that $M \ge x, \forall x \in \mathcal{C}$) then $\mathcal{P}$ has a
  maximal element.
  \label{lem:zorn}
\end{lemma}

We can use \nameref{lem:zorn} lemma to prove that any proper ideal is
in a \nameref{def:maxideal}.

Let $\mathcal{P}$ is the set of proper ideals in $A$ containing
$I$. The set is not empty because it has at least one element $I$. Any
\nameref{def:chain} $\mathcal{C} = \left\{I_\alpha\right\}$
\footnote{
  The order is the following $I_\alpha \le I_\beta$ if
  $I_\alpha \subset I_\beta$
}
has an upper bound: it's $\cup_\alpha I_\alpha$ (exercise that the
union is an ideal). So $\mathcal{P}$ has a maximal element $m$ and $I
\subset m$.

If we take a \nameref{def:quotientring} by maximal ideal it's always a
field otherwise 
it will have a proper ideal: $\exists a \in A/m$ such that $(a)$ is a
proper ideal and it pre-image in $\pi: A \to A/m$ should strictly
contain $m$
\footnote{ ??? i.e. $m$ is not a maximal ideal in the case.
}.

\section{Extension of homomorphisms. Uniqueness of algebraic closure}

Some summary about just proved existence of algebraic closure. There
exists $\bar{K} = \cup_{i=1}^\infty K_i$ - algebraic closure of $K$,
where
\[
K \subset K_1 \subset K_2 \subset \dots \subset K_{i-1} \subset K_i
\subset \dots
\]
$K_i$ is a field where each polynomial $P \in K_{i-1}$ has a root. The
field $K_i$ is \nameref{def:quotientring} of huge polynomial ring
$K_{i-1}\left[X\right]$ by a suitable \nameref{def:maxideal} that is
got by means of \nameref{lem:zorn} lemma.

Another question is the closure unique? The answer is yes. We start
the proof with the following theorem 

\begin{theorem}[About extension of homomorphism]
  Let $K \subset L \subset M$ - \nameref{def:algebraicextension}.
  $K \subset \Omega$, where $\Omega$ - \nameref{def:algebraicclosure}
  of $K$.
  $\forall \phi: L \to \Omega$ extends to $\widetilde{\phi}: M \to
  \Omega$
  \begin{proof}
    Apply \nameref{lem:zorn} lemma to the following set (of pairs)
    \[
    \mathcal{E} = \left\{
    \left(N, \psi\right): L \subset N \subset M, \psi \mbox{ extends }
    \phi 
    \right\}
    \]
    $\mathcal{E}$ is non empty because $\left(L,\phi\right) \in
    \mathcal{E}$.

    The set $\mathcal{E}$ is partially ordered by the following
    relation ($\le$):
    \[
    \left(N, \psi\right) \le \left(N', \psi'\right),
    \]
    if $N \subseteq N'$ and $\psi'/N = \psi$ ($\psi'$ extends $\psi$).
    Any \nameref{def:chain} $\left(N_\alpha, \psi_\alpha\right)$
    has an upper bound $\left(N, \psi\right)$, where
    \(
    N = \cup_\alpha N_\alpha
    \) - field, sub extension of $M$. $\psi$ defined in the following
    way: for $x \in N_\alpha$ $\psi(x) = \psi_\alpha(x)$.

    Thus $\mathcal{E}$ has a maximal element that we denote by
    $\left(N_0, \psi_0\right)$.

    Lets suppose that $N_0 \ne M$, i.e.
    $N_0 \subsetneq M$. Now it's very easy to get a
    contradiction. Lets take $x \in M \setminus N_0$ and consider
    \nameref{def:minpolynomial} $P_{min}\left(x, N_0\right)$. It
    should have a root $\alpha \in \Omega$. Now we extend $N_0$ to
    $N_0\left(x\right)$ and define $\psi'$ on
    $N_0\left(x\right)$ as follows: $\forall y \in N_0: \psi'(y)
    = \psi_0(y)$ and $\psi'(x) = \alpha$. Thus we was able to find an
    element of the chain that is greater than maximal. Therefore our
    assumption about $N_0 \ne M$ was incorrect and we can conclude
    than $N_0 = M$ and therefore $\tilde{\phi} = \psi_0$. 
  \end{proof}
  \label{thm:lec2_3}
\end{theorem}

\begin{corollary}[About algebraic closure isomorphism]
  If $\Delta$ and $\Delta'$ are 2 algebraic closures of $K$ then they
  are isomorphic as K-algebras.
  \begin{proof}
    Using theorem \ref{thm:lec2_3} one can assume
    $L = K$, $M = \Delta'$ and $\Omega = \Delta$ i.e. we have
    \[
    K \subset K \subset \Delta'
    \]
    in this case homomorphism $K \to \Delta$ can be extended to
    $\Delta' \to \Delta$ i.e. there exists a homomorphism
    (i.e. \nameref{def:injection}) from $\Delta'$ to $\Delta$.

    If we assume $M = \Delta$ and $\Omega = \Delta$ then there exists
    a homomorphism (i.e. \nameref{def:injection}) from $\Delta$ to
    $\Delta'$. The \nameref{def:injection} is also
    \nameref{def:surjection} in another direction: $\Delta' \to
    \Delta$ and as result we have \nameref{def:isomorphism} $\Delta' \to
    \Delta$ 
  \end{proof}
  \label{col:algebraic_closure_isomorphism}
\end{corollary}
