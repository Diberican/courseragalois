%% -*- coding:utf-8 -*-
\chapter{Finite fields. Separability, perfect fields}
We recall the construction and basic properties of finite fields. We
prove that the multiplicative group of a finite field is cyclic, and
that the automorphism group of a finite field is cyclic generated by
the Frobenius map. We introduce the notions of separable (resp. purely
inseparable) elements, extensions, degree. We briefly discuss perfect
fields.

\section{An example (of extension)s. Finite fields}

\begin{corollary}
  \nameref{def:algebraicclosure} of $K$ is unique up to
  \nameref{def:isomorphism} of K-algebras
  \footnote{
    There is a redefinition of corollary
    \ref{col:algebraic_closure_isomorphism}.
  }
  \label{col:lec3_1}
\end{corollary}

\begin{corollary}
  Any \nameref{def:algebraicextension} of $K$ is embedded into the
  \nameref{def:algebraicclosure} \footnote{seems to be a reformulation
    of theorem \ref{thm:lec2_3}}
  \label{col:lec3_2}
\end{corollary}

\begin{example}[Of extension of homomorphism]
  Let $K = \mathbb{Q}$ and $\overline{\mathbb{Q}}$ is the
  \nameref{def:algebraicclosure} of $K$. For instance we can consider
  $\overline{\mathbb{Q}} \subset \mathbb{C}$.
  \footnote{
    Really $\overline{\mathbb{Q}} = \mathbb{A}$ - the set of all algebraic
    numbers, i.e. roots of polynomials $P \in
    \mathbb{Q}\left[X\right]$.
  }

  Let
  \[
  L = \mathbb{Q}\left(\sqrt{2}\right) =
  \mathbb{Q}\left[X\right]/\left(X^2 - 2\right),
  \]
  $\alpha$ is a class of $X$ in $L$. $L$ has 2 \nameref{def:embedding}s into
  $\overline{\mathbb{Q}}$
  \begin{enumerate}
  \item $\phi_1: \alpha \to \sqrt{2}$
  \item $\phi_2: \alpha \to -\sqrt{2}$
  \end{enumerate}

  Let
  \[
  M = \mathbb{Q}\left(\sqrt[4]{2}\right) =
  \mathbb{Q}\left[Y\right]/\left(Y^4 - 2\right),
  \]
  $\beta$ is a class of $Y$ in $M$. $M$ has 4 \nameref{def:embedding}s into
  $\overline{\mathbb{Q}}$
  \begin{enumerate}
  \item $\psi_1: \beta \to \sqrt[4]{2}$ (extends $\phi_1$)
  \item $\psi_2: \beta \to -\sqrt[4]{2}$ (extends $\phi_1$)
  \item $\psi_3: \beta \to i\sqrt[4]{2}$ (extends $\phi_2$)
  \item $\psi_4: \beta \to -i\sqrt[4]{2}$ (extends $\phi_2$)
  \end{enumerate}
  This (``extends'') is because
  \[
  M = L\left[Y\right]/\left(Y^2 - \alpha\right) 
  \]
\end{example}

\subsection{Finite fields}

\begin{definition}[Finite field]
  $K$ is a finite field if it's characteristic (see section
  \ref{sec:fieldcharacteristic}) $char K = p$, where $p$ 
  - prime number 
  \label{def:finitefield}
\end{definition}

\begin{remark}[$\mathbb{F}_{p^n}$]
  If $K$ is a finite extension of $\mathbb{F}_p$ and
  $n = \left[K:\mathbb{F}_p\right]$ then number of elements of $K$:
  $\left|K\right| = p^n$. The following notation is also used for a
  finite extension of a finite field: $\mathbb{F}_{p^n}$
  \label{rem:fpn}
\end{remark}

\begin{remark}[Frobenius homomorphism]
  If $char K = p$, then exists a \nameref{def:homomorphism}
  $F_p: K \to K$ such that $x \in K \to x^p \to K$.
  Really if we consider $\left(x+y\right)^p$ and $\left(xy\right)^p$
  then we can get $\left(x+y\right)^p = x^p + y^p$ and
  $\left(xy\right)^p = x^p y^p$. The second property is the truth in
  the all fields (of course) but the first one is the special property
  of $\mathbb{F}_p$ fields.
  \label{rem:frobeniushomomorphism}
\end{remark}

\begin{remark}
  Also $F_{p^n} : x \in K \to x^{p^n} \in K$ is also homomorphism (a
  power of \nameref{rem:frobeniushomomorphism}.
  \label{rem:frobeniuspowerhomomorphism}
\end{remark}

\section{Properties of finite fields}

\begin{theorem}
  Lets fix $\mathbb{F}_P$ and it's \nameref{def:algebraicclosure}
  $\overline{\mathbb{F}_P}$.

  The \nameref{def:splittingfield} of $x^{p^n} - x$ has $p^n$
  elements. Conversely any field of $p^n$ elements is a splitting
  field of $x^{p^n} - x$. Moreover there is an unique sub extension of
  $\overline{\mathbb{F}_P}$ with $p^n$ elements.
  \begin{proof}
    Note that $F_{p^n} : x \to x^{p^n}$ is a
    \nameref{def:homomorphism} (see remark
    \ref{rem:frobeniuspowerhomomorphism}) as result the following set
    $\{
    x \mid F_{p^n}\left(x\right) = x
    \}$ is a sub-field containing $\mathbb{F}_p$
    \footnote{
      For $x \in \mathbb{F}_p$ we have that $x^{p-1} = 1$ (the field
      with fixed number of elements) and therefore
      $\forall x \in \mathbb{F}_p: x^p = x$ i. e.
      $F_{p^n}\left(x\right) = x$.
    }
    Lets $Q_n\left(X\right) = X^{p^n} - X$ thus the considered set
    consists of the root of the polynomial $Q_n$. The polynomial has
    no multiple roots because $gcd(Q_n, Q_n') = 1$.
    \footnote{
      If $Q_n$ has a multiple root $\beta$ then it is divisible by
      $\left(X - \beta\right)^2$ and the $Q_n'$ is divisible by (at
      least) $\left(X - \beta\right)$ thus the $\left(X -
      \beta\right)$ should be a part of gcd.
    }
    This is because $Q_n' \equiv 1 \mod p$.
    \footnote{
      ??? $Q_n' = p^n X^{p^n - 1} - 1 \equiv -1 \mod p$
    }
    As soon as $Q_n$ has no multiple roots then there are $p^n$
    different roots and therefore the splitting field is the field
    with $p^n$ elements.

    Conversely lets $\left|K\right| = p^n$ and
    $\alpha \ne 0 \in K$.
    Using the fact that the multiplication group of $K$ has $p^n - 1$
    elements: $\left|K^*\right| = p^n - 1$
    \footnote{
      $K^* = K \setminus \{0\}$
    }
    as result the multiplication of all the elements should give us
    $1$: $\alpha^{p^n-1} = 1$ or $\alpha^{p^n} - \alpha = 0$.
    Therefore $\alpha$ is a root of $Q_n$. Thus the splitting field of
    $Q_n$ consists of elements of $K$.

    The uniqueness of sub-extension of
    $\mathbb{F}_p$ with $p^n$ elements is a result of uniqueness of
    the splitting field (see theorem \ref{thm:lec2_1}). 
  \end{proof}
  \label{thm:lec3_1}
\end{theorem}

\begin{theorem}
  $\mathbb{F}_{p^d} \subset \mathbb{F}_{p^n}$ if and only if $d \mid n$. 
  \begin{proof}
    Let $\mathbb{F}_{p^d} \subset \mathbb{F}_{p^n}$ in this case
    $\mathbb{F}_p \subset \mathbb{F}_{p^d} \subset \mathbb{F}_{p^n}$
    and
    \[
    \left[\mathbb{F}_{p^n}:\mathbb{F}_{p}\right] =
    \left[\mathbb{F}_{p^n}:\mathbb{F}_{p^d}\right]
    \left[\mathbb{F}_{p^d}:\mathbb{F}_{p}\right]
    \]
    or $n = x \cdot d$ i.e. $d \mid n$

    Conversely if $d \mid n$ then $n = x \cdot d$ or
    $p^n = \prod^x_{i=1} p^d$ thus if $x^{p^d} = x$ then
    \[
    x^{p^n} = x^{\prod^x_{i=1} p^d}
    \left(x^{p^d}\right)^{\prod^x_{i=2} p^d} = x^{\prod^x_{i=2} p^d} =
    \dots = x^{p^d} = x,
    \]
    i.e. $\forall \alpha \in \mathbb{F}_{p^d}$ we also have
    $\alpha \in \mathbb{F}_{p^n}$ or in other notation:
    $\mathbb{F}_{p^d} \subset \mathbb{F}_{p^n}$.
  \end{proof}
  \label{thm:lec3_1_2}
\end{theorem}

\begin{theorem}
  $\mathbb{F}_{p^n}$ is a \nameref{def:stemfield} and a
  \nameref{def:splittingfield} of any irreducible polynomial
  $P \in \mathbb{F}_p$ of degree $n$.
  \begin{proof}
    \nameref{def:stemfield} $K$ has to have degree $n$ over
    $\mathbb{F}_p$ i.e.
    $\left[K:\mathbb{F_p}\right] = n$ i.e. it should have $p^n$
    elements and therefore $K=\mathbb{F_{p^n}}$
    (see also remark \ref{rem:fpn}).

    About \nameref{def:splittingfield}. Using the just proved result
    we can say that if $\alpha$ is a root of $P$
    then $\alpha \in \mathbb{F_{p^n}}$ thus
    $Q_n\left(\alpha\right) = 0$. Therefore $P$ divides $Q_n$ 
    \footnote{as soon as any root of $P$ also a root of $Q_n$} and as
    result $P$ splits in $\mathbb{F_{p^n}}$.
  \end{proof}
  \label{thm:lec3_1_3}
\end{theorem}

\begin{corollary}
  Let $\mathcal{P}_d$ is the set of all irreducible,
  \nameref{def:monicpolynomial}s of degree $d$ such that
  $\mathcal{P}_d \subset \mathbb{F}_p\left[X\right]$ then 
  \[
  Q_n = \prod_{d \mid n} \prod_{P \in \mathcal{P}_d} P
  \]
  \begin{proof}
    As we just seen if $P \in \mathcal{P}_d$ and $d \mid n$ then
    $P \mid Q_n$.
    \footnote{
      Since stem field is $\mathbb{F}_{p^d} \subset \mathbb{F}_{p^n}$ 
    }
    Since all such polynomials are relatively prime of course
    \footnote{
      ??? As soon as $\mathbb{F}_p\left[X\right]$ is \nameref{def:ufd}
      then any polynomial can be written as a product of irreducible
      elements, uniquely up to order and units this means that each
      $P \in \mathcal{P}_d$ (where $d \mid n$) should be in the
      factorization of $Q_n$. It should be only one time because there
      is no multiply roots. 
    }
    and $Q_n$ have no multiple roots (as result no multiple factors) 
    then
    \[
    \prod_{d \mid n} \prod_{P \in \mathcal{P}_d} P \mid Q_n
    \]

    From other side let $R$ is an irreducible factor of $Q_n$.
    $\alpha$ is a root of $R$ then $Q_n\left(\alpha\right) = 0$ thus
    $\mathbb{F}_p\left(\alpha\right) \subset \mathbb{F}_{p^n}$
    therefore $\mathbb{F}_p\left(\alpha\right) = \mathbb{F}_{p^d}$
    where $d \mid n$ and as result, $deg R \mid n$. Thus the
    polynomial should be in the product
    $\prod_{d \mid n} \prod_{P \in \mathcal{P}_d} P$.
  \end{proof}
\end{corollary}

\begin{example}
  Let $p = n = 2$. The monic irreducible polynomials in $\mathbb{F}_2$
  whose degree divides $2$ are: $x$, $x+1$ and $x^2 +x + 1$.
  As you can see
  \begin{equation}
    x\left(x+1\right)\left(x^2+x+1\right) = x^4 + x = x^4 - x
    \nonumber
  \end{equation}
  because $2x = 0 \mod 2$ or $x = -x$.
\end{example}

\section{Multiplicative group and automorphism group of a finite
  field}
\begin{theorem}
  Let $K$ be a field and and $G$ be a finite \nameref{def:subgroup} of
  $K^*$ then $G$ is a \nameref{def:cyclicgroup}
  \begin{proof}
    Idea is to compare $G$ and the \nameref{def:cyclicgroup}
    $\mathbb{Z}/N\mathbb{Z}$ where $N = \left|G\right|$.

    Let
    \begin{itemize}
    \item $\psi\left(d\right)$ - is the number of elements of order $d$
      ( see also \nameref{def:grouporder}) in $G$. We need
      $\psi\left(N\right) \ne 0$
      \footnote{
        ???
      }
      and we know that
      $N = \sum \psi\left(d\right)$
    \item $\phi\left(d\right)$ - is the number of elements of order $d$
      ( see also \nameref{def:grouporder}) in $\mathbb{Z}/N\mathbb{Z}$
    \end{itemize}

    As $\mathbb{Z}/N\mathbb{Z}$ contains a single (cyclic) subgroup of
    order $d$ for $d \mid N$.
    \footnote{
      The one generated by $N/d$
    }
    $\phi\left(d\right)$ is the number of generators of
    $\mathbb{Z}/d\mathbb{Z}$ i.e. the number of elements between $1$
    and $d-1$ that are prime to $d$. We know that
    $\phi\left(N\right) \ne 0$.

    \begin{claim}
      Either $\psi\left(d\right) = 0$ or
      $\psi\left(d\right) = \phi\left(d\right)$
      \footnote{
        suffices since $\sum \psi\left(d\right) = \sum
        \phi\left(d\right) = N$
      }
      \begin{proof}
        In no element of order $d$ in $G$ then $\psi\left(d\right) = 0$
        otherwise if $x \in G$ has order $d$ then $x^d = 1$ or $x$ is a
        root of the following polynomial $x^d - 1$. The roots of the
        polynomial forms a cyclic subgroup of $G$. So $G$ as well as
        $\mathbb{Z}/N\mathbb{Z}$ has a single cyclic subgroup of order
        $d$ (which is cyclic) or no such group at all.

        If $\psi\left(d\right) \ne 0$ then exists such a subgroup and
        $\psi\left(d\right)$ is equal to the number of generators of that
        group or $\phi\left(d\right)$
        \footnote{
          ???
        }
        In particular $\psi\left(d\right) \le \phi\left(d\right)$ but
        there should be equality because the sum of both $\sum
        \psi\left(d\right) = \sum \phi\left(d\right) = N$.
      \end{proof}
    \end{claim}
    In particular $\psi\left(N\right) \ne 0$ and we proved the
    theorem.
  \end{proof}
  \label{thm:lec3_2}
\end{theorem}
\begin{corollary}
  If $K \subset \mathbb{F}_p$ and
  $\left[K:\mathbb{F}_p\right] = n$ then $\exists \alpha$ such that
  $K = \mathbb{F}_p\left(\alpha\right)$. In particular $\exists$ an
  irreducible polynomial of degree $n$ over $\mathbb{F}_p$
  \footnote{
    The theorem \ref{thm:lec3_1_3} says that the stem field for any
    polynomial of degree $n$ over $\mathbb{F}_p$ exists and there is
    over $\mathbb{F}_{p^n}$ but we had not proved yet that an
    irreducible polynomial of degree $n$ exists.
  }
  \begin{proof}
    We can take $\alpha = \mbox{ generator of } K^*$.
  \end{proof}
  \label{cor:lec3_1}
\end{corollary}

\begin{corollary}
  The group of automorphism of $\mathbb{F}_{p^n}$ over
  $\mathbb{F}_{p}$ is cyclic and generated by Frobenius map:
  $F_p: x \to x^p$
  (see remark \ref{rem:frobeniushomomorphism}) 
  \begin{proof}
    As we know from theorem \ref{thm:lec3_1}:
    $\forall x \in \mathbb{F_{p^n}}: x^{p^n} = x$ so
    $F_p^n = Id$
    \footnote{
      because $F_p^n: x \to x^{p^n} = x$.
    }.
    From other side if $m < n$ then
    $x^{p^m} -x =0$ has $p^m < p^n$ roots and cannot be identity
    \footnote{
      because operates only with $p^m$ elements i.e. not of all
      elements of $\mathbb{F_{p^n}}$.
    }
    Finally (from corollary \ref{cor:lec3_1} ) we have
    $\mathbb{F}_{p^n} = \mathbb{F}_p\left(\alpha\right)$ where
    $\alpha$ is a root of an irreducible polynomial of degree
    $n$. Thus there exists exactly $n$ automorphisms of
    $\mathbb{F}_{p^n}$.
    \footnote{
      Each automorphism converts the root $\alpha$ into another one
      of $n$ roots of the irreducible polynomial
    }
    So
    \[
    \left|
    Aut\left(
    \mathbb{F}_{p^n}/\mathbb{F}_{p}
    \right)
    \right| \le n
    \]
    and as we have $n$ of them (\nameref{def:automorphism}s) then
    \[
    \left|
    Aut\left(
    \mathbb{F}_{p^n}/\mathbb{F}_{p}
    \right)
    \right| = n
    \]      
  \end{proof}
  \label{cor:lec3_2}
\end{corollary}

\section{Separable elements}
\section{Separable degree, separable extensions}
\section{Perfect fields}
