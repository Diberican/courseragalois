%% -*- coding:utf-8 -*-
\chapter*{Requirements}

\section{Groups}

\begin{definition}[Monoid]
  The set of elements $M$ with defined binary operation $\circ$ we will call
  as a monoid if the following conditions are satisfied.
  \begin{enumerate}
  \item Closure: $\forall a, b \in M$: $a \circ b \in G$
  \item Associativity: $\forall a, b, c \in M$:
    $a \circ \left( b \circ c \right) =
    \left( a \circ b \right) \circ c$
  \item Identity element: $\exists e \in M$ such that
    $\forall a \in G$: $e \circ a = a \circ e = a$
  \end{enumerate}
  \label{def:monoid}
\end{definition}

\begin{definition}[Group]
  Let we have a set of elements $G$ with a defined binary operation
  $\circ$ that satisfied the following properties.
  \begin{enumerate}
  \item Closure: $\forall a, b \in G$: $a \circ b \in G$
  \item Associativity: $\forall a, b, c \in G$:
    $a \circ \left( b \circ c \right) =
    \left( a \circ b \right) \circ c$
  \item Identity element: $\exists e \in G$ such that
    $\forall a \in G$: $e \circ a = a \circ e = a$
  \item Inverse element: $\forall a \in G$ $\exists a^{-1} \in G$ such that
    $a \circ a^{-1} = e$
  \end{enumerate}
  In this case $\left(G, \circ\right)$ is called as group.
  \label{def:group}
\end{definition}
Therefore the group is a \nameref{def:monoid} with inverse element
property. 

\begin{example}[Group $\mathbb{Z}/2\mathbb{Z}$]
  Consider a set of 2 elements: $G = \left\{0, 1\right\}$ with the
  operation $\circ$ defined by the table \ref{tab:CayleyZ2Z}.
  \begin{table}
    \centering
    \caption{Cayley table for $\mathbb{Z}/2\mathbb{Z}$}
    \label{tab:CayleyZ2Z}
    \begin{tabular}{l|ll}
      \toprule
      $\circ$ & 0 & 1 \\
      \midrule
      0 & 0 & 1 \\
      1 & 1 & 0 \\
      \bottomrule
    \end{tabular}
  \end{table}

  The identity element is $0$ i.e. $e = 0$.
  Inverse element is the element itself
  because $\forall a \in G$: $a \circ a = 0 = e$.
  \label{ex:group}
\end{example}

\begin{definition}[Subgroup]
  Let we have a \nameref{def:group} $\left(G, \circ\right)$. The
  subset $S \subset G$ is called as subgroup if $\left(S,
  \circ\right)$ is a \nameref{def:group}.
  \label{def:subgroup}
\end{definition}

\begin{definition}[Abelian group]
  Let we have a \nameref{def:group} $\left(G, \circ\right)$.
  The group is called an Abelian or commutative if
  $\forall a, b \in G$ it holds $a \circ b = b \circ a$.
  \label{def:abeliangroup}
\end{definition}

\begin{definition}[Coset]
  If $G$ is a group, and $H$ is a subgroup of $G$, and $g$ is an
  element of $G$, then
  \[
  gH = \left\{ gh \vert h \in H\right\}
  \]
  is the left coset of $H$ in $G$ with respect to $g$, and
  \[
  Hg = \left\{ hg \vert h \in H\right\}
  \]
  is the right coset of $H$ in $G$ with respect to $g$.
  \label{def:coset}
\end{definition}
  
\subsection{Permutations}

\begin{example}[$S_n$ group]
  If we a have a permutation of $n$ elements then it's possible to do
  by means of $n!$ ways.
  
  $S_1$ permutation of 1 element consists of only one element $e$ -
  the simplest possible group

  $S_2$ permutation consists of 2 elements:
  \begin{enumerate}
  \item identity $e$:
    \[
    \begin{array}{c}
    1 \to 1 \\
    2 \to 2
    \end{array}
    \]
  \item transposition $\tau$:
    \[
    \begin{array}{c}
    1 \to 2 \\
    2 \to 1
    \end{array}
    \]    
  \end{enumerate}
  It's easy to see that the Cayley table has the form \ref{tab:CayleyS2}
    \begin{table}
    \centering
    \caption{Cayley table for $S_2$}
    \label{tab:CayleyS2}
    \begin{tabular}{l|ll}
      \toprule
      $\circ$ & $e$ & $\tau$ \\
      \midrule
      $e$ & $e$ & $\tau$ \\
      $\tau$ & $\tau$ & $e$ \\
      \bottomrule
    \end{tabular}
    \end{table}

    $S_3$ permutation consists of 6 elements: $e, \tau, \tau_1, \tau_2,
    \sigma, \sigma_1$. The most important are $e, \tau$ and $\sigma$
    and all others are represented via them.
  \begin{enumerate}
  \item identity $e$:
    \[
    \begin{array}{c}
    1 \to 1 \\
    2 \to 2 \\
    3 \to 3
    \end{array}
    \]
  \item transposition $\tau$:
    \[
    \begin{array}{c}
    1 \to 2 \\
    2 \to 1 \\
    3 \to 3
    \end{array}
    \]    
  \item circle $\sigma$:
    \[
    \begin{array}{c}
    1 \to 2 \\
    2 \to 3 \\
    3 \to 1
    \end{array}
    \]    
  \end{enumerate}  
  \label{ex:sngroup}
\end{example}

\section{Rings, Ideals and Fields}

\begin{definition}[Ring]
  Consider a set $R$ with 2 binary operations defined. The first one
  $\oplus$ (addition) and elements of $R$ forms an
  \nameref{def:abeliangroup}
  under this operation. The second one is $\odot$ (multiplication) and
  the elements of $R$ forms a \nameref{def:monoid} under 
  the operation. The two binary operations are connected each other
  via the following distributive law
  \begin{itemize}
  \item Left distributivity:
    $\forall a,b,c \in R$:
    $a \odot \left(b \oplus c\right) =
    a \odot b \oplus a \odot c$
  \item Right distributivity:
    $\forall a,b,c \in R$:
    $\left( a \oplus b \right) \odot c =
    a \odot c \oplus b \odot c$
    
  The identity element for $\left(R, \oplus\right)$ is denoted as $0$
  (additive identity).
  The identity element for $\left(R, \odot\right)$ is denoted as $1$
  (multiplicative identity).

  The inverse element to $a$ in $\left(R, \oplus\right)$ is denoted as $-a$
  \end{itemize}

  In this case $\left(R, \oplus, \odot\right)$ is called as ring.
  \label{def:ring}
\end{definition}

The \nameref{def:ring} is a generalization of integer numbers conception.
\begin{example}[Ring of integers $\mathbb{Z}$]
  The set of integer numbers $\mathbb{Z}$ forms a \nameref{def:ring}
  under $+$ and $\cdot$ operations i.e. addition $\oplus$ is
  $+$ and multiplication $\odot$ is $\cdot$. Thus for integer
  numbers we have the following \nameref{def:ring}:
  $\left(\mathbb{Z}, +, \cdot\right)$
  \label{ex:ring}
\end{example}

\begin{definition}[Ideal]
  Lets we have the \nameref{def:ring}
  $\left(R, \oplus, \odot\right)$. Subset $I \subset R$ will be an
  ideal if it satisfied the following conditions
  \begin{enumerate}
  \item $\left(I, \oplus\right)$ is \nameref{def:subgroup} of
    $\left(R, \oplus\right)$
  \item $\forall i \in I$ and $\forall r \in R$:
    $i \odot r \in I$ and $r \odot i \in I$
  \end{enumerate}
  \label{def:ideal}
\end{definition}

\begin{example}[Ideal $2 \mathbb{Z}$]
  Consider even numbers. They forms an \nameref{def:ideal} in
  $\mathbb{Z}$. Because multiplication of any even number to any
  integer is an even. The ideal's symbolic name is $2 \mathbb{Z}$.
  \label{ex:ideal}
\end{example}

\begin{example}[Ring of integers modulo $n$: $\mathbb{Z}/n\mathbb{Z}$]
  Let $n \in \mathbb{Z}$ and $n > 1$. Then $n \mathbb{Z}$ is an
  \nameref{def:ideal}.

  Two integer $a, b \in \mathbb{Z}$  are said to be congruent modulo
  $n$, written
  \[
  a \equiv b ( mod n )
  \]
  if their difference $a - b$ is an integer multiple of $n$.

  Thus we have a separation of set $\mathbb{Z}$ into subsets of
  numbers that are congruent. Each subset has the following form
  \[
  \left\{r\right\}_n = r + n \mathbb{Z} =
  \left\{r + n k \mid k \in \mathbb{Z}\right\}
  \],
  thus
  \[
  \mathbb{Z} = \left\{0\right\}_n \cup \left\{1\right\}_n
  \cup \dots \cup \left\{n-1\right\}_n.
  \]

  Very often use the following notation
  \[
  \bar{r} = \left\{r\right\}_n.
  \]

  We can define the following operations
  \begin{eqnarray}
    \bar{k} \oplus \bar{l} = \overline{k + l}
    \nonumber \\
    \bar{k} \odot \bar{l} = \overline{k \cdot l}
    \nonumber
  \end{eqnarray}
  
  The \nameref{def:ring} where the objects are defined is called as
  $\mathbb{Z}/n\mathbb{Z}$.
  \label{ex:intmodulo}
\end{example}

\begin{definition}[Principal ideal]
  The ideal that is generated by one element $a$ is called as
  principal ideal and is denoted as $\left(a\right)$ i.e.
  left principal ideal:
  $\left(a\right) = \left\{r a \mid \forall r \in R\right\}$ and
  right principal ideal:
  $\left(a\right) = \left\{a r \mid \forall r \in R\right\}$
  \label{def:prinicipalideal}
\end{definition}

\begin{definition}[Integral domain]
In mathematics, and specifically in abstract algebra, an integral
domain is a nonzero commutative \nameref{def:ring} in which the
product of any two nonzero elements is nonzero.
\label{def:integraldomain}
\end{definition}

\begin{definition}[Principal ideal domain]
  In abstract algebra, a principal ideal domain, or PID, is an
  \nameref{def:integraldomain} in which every ideal is principal, i.e., can be
  generated by a single element.
  \label{def:pid}
\end{definition}

\begin{definition}[Maximal ideal]
$I$ is a maximal ideal of a ring $R$ if there are no other ideals
contained between $I$ and $R$.
\label{def:maxideal}
\end{definition}

\begin{definition}[Proper ideal]
$I$ is a proper ideal of a ring $R$ if $I \subsetneq R$.
\label{def:properideal}
\end{definition}

\begin{definition}[Quotient ring]
  Quotient ring is a construction where one
  starts with a ring $R$ and a two-sided ideal $I$ in $R$, and constructs a
  new ring, the quotient ring $R/I$, whose elements are the
  \nameref{def:coset}s of $I$ 
  in $R$ subject to special $+$ and $\cdot$ operations.

  Given a ring $R$ and a two-sided ideal $I \subset R$, we may define
  an equivalence relation $\sim$ on $R$ as follows: 
  $a \sim b$ if and only if $a - b \in I$.
  The equivalence class of the element $a$ in $R$ is given by
  \[
  \left[a\right] = a + I := \left\{ a + r : r \in I \right\}.
  \]
  This equivalence class is also sometimes written as a mod $I$ and
  called the "residue class of a modulo I".
  \label{def:quotientring}
\end{definition}

\begin{definition}[Field]
  The ring $\left(R, \oplus, \odot\right)$ is called as a field if
  $\left(R \setminus \{0\}, \odot\right)$ is an \nameref{def:abeliangroup}.

  The inverse element to $a$ in
  $\left(R \setminus\{0\}, \odot\right)$ is denoted as $a^{-1}$
  \label{def:field}
\end{definition}

\begin{example}[Field $\mathbb{Q}$]
  Note that $\mathbb{Z}$ is not a field because not for every integer
  number an inverse exists. But if we consider a set of fractions
  $\mathbb{Q} = \left\{a/b \mid a \in \mathbb{Z}, b \in
  \mathbb{Z}\setminus\{0\}\right\}$ when it will be a field.

  The
  inverse element to $a/b$  in
  $\left(\mathbb{Q}\setminus\{0\}, \cdot\right)$  will be $b/a$.
  \label{ex:field}
\end{example}

\begin{definition}[Unique factorization domain]
   Unique factorization domain (UFD) is a commutative ring, which is
   an \nameref{def:integraldomain}, and in which every non-zero non-unit element
   can be written as a product of prime elements (or irreducible
   elements), uniquely up to order and units, analogous to the
   fundamental theorem of arithmetic for the integers. 
  \label{def:ufd}
\end{definition}

\section{Linear algebra}

\begin{definition}[Vector space]
  Let $F$ is a \nameref{def:field}. The set $V$ is called as vector
  space under $F$ if the following conditions are satisfied
  \begin{enumerate}
  \item We have a binary operation $V \times V \rightarrow V$
    (addition): $(x,y) \rightarrow x + y$ with the following
    properties:
    \begin{enumerate}
    \item $x + y = y + x$
    \item $(x + y) + z = x + ( y + z )$
    \item $\exists 0 \in V$ such that $\forall x \in V: x + 0 = x$
    \item $\forall x \in V \exists -x \in V$ such that $x + (-x) = x -
      x = 0$ 
    \end{enumerate}    
  \item We have a binary operation $F \times V \rightarrow V$ (scalar
    multiplication) with the following properties
    \begin{enumerate}
    \item $1_F \cdot x = x$
    \item $\forall a,b \in F, x \in V$: $a\cdot\left(b \cdot x\right)
      = \left(a b\right) \cdot x$.
    \item $\forall a,b \in F, x \in V$:
      $(a+b)\cdot x = a \cdot x + b \cdot x$
    \item $\forall a \in F, x, y \in V$:
      $a\cdot(x+y) = a\cdot x + a \cdot y$
    \end{enumerate}        
  \end{enumerate}
  \label{def:vectorspace}
\end{definition}

\begin{lemma}[About vector space isomorphism]
  2 vector spaces $L$ and $M$ with same dimension $dim L = dim M$ then
  there exists an \nameref{def:isomorphism} between them
  \label{lem:vsisomorphism}
\end{lemma}

\section{Functions}

\begin{definition}[Surjection]
  The function $f: X \rightarrow Y$ is surjective (or onto) if
  $\forall y \in Y$, $\exists x \in X$ such that
  $f\left(x\right) = y$.
  \label{def:surjection}
\end{definition}

\begin{definition}[Injection]
  The function $f: X \rightarrow Y$ is injective (or one-to-one function) if
  $\forall x_1, x_2 \in X$, such that $x_1 \ne x_2$ then
  $f\left(x_1\right) \ne f\left(x_2\right)$.
  \label{def:injection}
\end{definition}

\begin{definition}[Bijection]
  The function $f: X \rightarrow Y$ is bijective (or one-to-one
  correspondence) if it is an \nameref{def:injection} and a
  \nameref{def:surjection}. 
  \label{def:bijection}
\end{definition}

\begin{definition}[Homomorphism]
  The homomorphism is a function (map) between two sets that preserves
  its algebraic structure. For the case of groups
  $\left(X, \circ\right)$ and $\left(Y, \odot\right)$ the function
  $f: X \rightarrow Y$ is called homomorphism if
  $\forall x_1, x_2 \in X$ it holds
  $f\left(x_1 \circ x_2\right) = f\left(x_1 \right) \odot f\left( x_2\right)$.
  \label{def:homomorphism}
\end{definition}

\begin{definition}[Isomorphism]
  If a map is \nameref{def:bijection} as well as
  \nameref{def:homomorphism} when it is called as isomorphism.

  We use the following symbolic notation for isomorphism between $X$
  and $Y$: $X \cong Y$.
  \label{def:isomorphism}
\end{definition}

\begin{definition}[Automorphism]
  Automorphism is an isomorphism from a mathematical object to itself.
  \label{def:automorphism}
\end{definition}

\begin{definition}[Embedding]
  When some object X is said to be embedded in another object $Y$, the
  embedding is given by some injective and structure-preserving map
  $f : X \to Y$. The precise meaning of "structure-preserving" depends on
  the kind of mathematical structure of which $X$ and $Y$ are
  instances.
  
  The fact that a map $f : X \to Y$ is an embedding is often indicated
  by the use of a "hooked arrow", thus: $f:X\hookrightarrow Y$. On the
  other hand, this notation is sometimes reserved for inclusion maps.
  \label{def:embedding}
\end{definition}

\section{Polynomial ring $K\left[X\right]$}

Let we have a commutative \nameref{def:ring} $K$. Lets create a new
\nameref{def:ring} $B$ with the following infinite sets as elements:
\begin{equation}
  f = \left(f_0, f_1, \dots \right), \, f_i \in K,
  \label{eq:polynomial}
\end{equation}
such that only finite number of elements of the sets are non zero.

We can define addition and multiplication on $B$ as follows
\begin{eqnarray}
  f + g = \left(f_0 + g_0, f_1 + g_1, \dots \right),
  \nonumber \\
  f \cdot g = h = \left(h_0, h_1, \dots \right),
  \label{eq:polynomialops}
\end{eqnarray}
where
\[
h_k = \sum_{i + j =k} f_i g_j. 
\]

The sequences (\ref{eq:polynomial}) forms a \nameref{def:ring} with
the following identities:
\begin{itemize}
\item Additive identity: $\left(0, 0, \dots \right)$
\item Multiplicative identity: $\left(1, 0, \dots \right)$
\end{itemize}

The
sequences $k = \left(k, 0, \dots \right)$ added and multiplied as
elements of $K$ this allows say that such elements are elements of
original \nameref{def:ring} $K$. Thus $K$ is sub-ring of the new ring
$B$.

Let
\begin{eqnarray}
  X = \left(0, 1, 0, \dots \right),
  \nonumber \\
  X^2 = \left(0, 0, 1, \dots \right)
  \nonumber
\end{eqnarray}
thus if we have
\[
f = \left(f_0, f_1, f_2, \dots, f_n, 0, \dots \right),
\]
where $f_n$ is the last non-zero element of (\ref{eq:polynomial}),
when one can get
\[
f = f_0 + f_1 X + f_2 X^2 + \dots + f_n X^n.
\]

\begin{definition}[Polynomial ring]
  The \nameref{def:ring} of sequences (\ref{eq:polynomial}) with
  operations defined by (\ref{eq:polynomialops}) is called as
  polynomial ring $K\left[X\right]$.
  \label{def:polynomial}
\end{definition}

\begin{lemma}[Bézout's lemma]
  Let $a$ and $b$ be nonzero integers and let $d$ be their greatest common
  divisor. Then there exist integers $x$ and $y$ such that 
  \[
  a x + by = d.
  \]
  \label{lem:bezout}
\end{lemma}

\begin{definition}[Monic polynomial]
  Monic polynomial is a univariate polynomial in which the leading
  coefficient (the nonzero coefficient of highest degree) is equal to
  1. Therefore, a monic polynomial has the form
  \[
  x^n + a_{n-1}x^{n-1}+ \dots + a_1 x + a_0
  \]
  \label{def:monicpolynomial}
\end{definition}
