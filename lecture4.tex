%% -*- coding:utf-8 -*-
\chapter{Tensor product. Structure of finite K-algebras}
This is a digression on commutative algebra. We introduce and study
the notion of tensor product of modules over a ring. We prove a
structure theorem for finite algebras over a field (a version of the
well-known "Chinese remainder theorem").


\section{Definition of tensor product}

\subsection{Summary for previous lectures}
We considered finite \nameref{def:fextension1} $L$ i.e
$\left[L:K\right] < \infty$. We also saw that if
$L$ is generated by a finite number of \nameref{def:degsepelem}s 
$\alpha_1, \dots, \alpha_r$ then the number of
\nameref{def:homomorphism}s over $K$ from $L$ to $\bar{K}$ denoted by
$\left|Hom_K\left(L, \bar{K}\right)\right|$ is equal to
$\left[L:K\right]$. In general
\[
\left[L:K\right]_{sep} = 
\left|Hom_K\left(L, \bar{K}\right)\right| \le
\left[L:K\right].
\]

For $L = K\left(\alpha\right)$ it is clear because the number of
homomorphisms is equal to the number of roots of the
\nameref{def:minpolynomial} $P_{min}\left(\alpha, K\right)$. In
general one can use induction and multiplicativity of the degree
$\left[L:K\right]$ and number of homomorphisms (see theorem
\nameref{thm:lec3_3}). Thus  separable extension was exactly an
extension  which had the right number of homomorphisms into the
algebraic closure.

Our next goal is to characterize the separability in the terms of
tensor product.

\subsection{Tensor product}

\begin{definition}[Tensor product]
  Let $A$ is a ring, $N$, $M$ are $A$-\nameref{def:module}s. The
  tensor product $M \otimes_A N$
  is another $A$-\nameref{def:module} together with an
  $A$-bilinear map $\phi: M \times N \to M \otimes_A N$ which has
  ``\nameref{def:universalproperty}'' defined below
  \label{def:tensorproduct}
\end{definition}

\begin{definition}[Universal property]
  $A$-bilinear map $\phi: M \times N \to M \otimes_A N$ has
  ``universal property'' if
  $\forall P$ - $A$-\nameref{def:module} and
  for $A$-bilinear $f: M \times N \to P$ (
  i.e.
  \(
  \forall m, f_m:
  N \xrightarrow[n \to f(m,n)]{} P
  \)
  and
  \(
  \forall n, f_n:
  M \xrightarrow[m \to f(m,n)]{} P
  \)
  are \nameref{def:homomorphism}s of $A$-modules
  ), then
  $\exists! \tilde{f}$ - homomorphism of $A$-modules such that
  $f = \tilde{f} \circ \phi$
  \footnote{
    That means that we have a \nameref{def:commutativediagram} there
  }

  \begin{tikzpicture}[description/.style={fill=white,inner sep=2pt}]
    \matrix (m) [matrix of math nodes, row sep=3em,
      column sep=2.5em, text height=1.5ex, text depth=0.25ex]
            { M \times N & & P \\
              & M \otimes_A N & \\ };
            %\draw[double,double distance=5pt] (m-1-1) – (m-1-3);
            \path[->]
            (m-1-1) edge node[auto] {$ f $} (m-1-3)
            edge node[auto] {$ \phi $} (m-2-2)
            (m-2-2) edge node[auto] {$ \tilde{f} $} (m-1-3);
  \end{tikzpicture}
  \label{def:universalproperty}
\end{definition}

The property characterize the pair
$\left(\phi, M \otimes N\right)$.
Really if have another pair
$\left(\overline{\phi}, \overline{M \otimes N}\right)$
like this one then by definition we have mutually inverse
homomorphisms of $A$-modules between them

\begin{lemma}[About uniqueness of object defined by universal
    property]
  \footnote{
    It is out of the lecture video and can be considered as an
    explanation for the claim about having mutually inverse
    homomorphisms of $A$-modules
  }
  If we have two objects $\left(\phi, M \otimes N\right)$
  and $\left(\overline{\phi},\overline{M \otimes N}\right)$
  which both satisfies \nameref{def:universalproperty} than there is
  an unique \nameref{def:isomorphism} between them: 
  \[
  \left(\phi, M \otimes N\right) \cong \left(\overline{\phi},
  \overline{M \otimes N}\right)
  \]
\begin{proof}
Let $P = \overline{M \otimes N}$ and $f =
\overline{\phi}$. In 
the case we can consider the following diagram

\begin{tikzpicture}[description/.style={fill=white,inner sep=2pt}]
  \matrix (m) [matrix of math nodes, row sep=3em,
    column sep=2.5em, text height=1.5ex, text depth=0.25ex]
          { & & M \otimes_A N & \\
            M \times N & & \overline{M \otimes_A N} \\
            & & M \otimes_A N & \\ };
          %\draw[double,double distance=5pt] (m-1-1) – (m-1-3);
          \path[->]
          (m-2-1) edge node[auto] {$ \phi $} (m-1-3)
          (m-1-3) edge node[auto] {$ g = \tilde{\overline{\phi}} $} (m-2-3)
          (m-2-1) edge node[auto] {$ \overline{\phi} $} (m-2-3)
          (m-2-1) edge node[auto] {$ \phi $} (m-3-3)
          (m-2-3) edge node[auto] {$ \bar{g} = \tilde{\phi} $} (m-3-3);
\end{tikzpicture}

As soon as we fixed $\overline{ M \otimes_A N}$
we 2 unique homomorphisms (which are defined by the fixed
$\overline{ M \otimes_A N}$) - 
$g :  M \otimes_A N \to \overline{ M \otimes_A N}$
and
$\bar{g} :  \overline{M \otimes_A N} \to M \otimes_A N$.
Both $g$ and $\bar{g}$ are linear as mentioned above the pair is
unique (if we fix $g$ we will have only one $\bar{g}$ that corresponds
to $g$). The composition $g \circ \bar{g}$ maps $M \otimes_A N$ to
itself. Thus if we fix $g$ and choose $\bar{g} = g^{-1}$ we will get
$g \circ \bar{g} = id_{M \otimes_A N}$ that satisfied all
requirements. The choice is final because we don't have a possibility
to choose any other $\bar{g}$ (it should be unique).

Thus we have an
\nameref{def:isomorphism} and the isomorphism is unique as soon as the
function $g$ is unique due the \nameref{def:universalproperty}.

We just prove an isomorphism existence between
$M \otimes N$ and $\overline{M \otimes N}$ but the tensor product is
characterized not only by the module $M \otimes N$
but also a bilinear map $\phi$. Let $P = \overline{M \otimes N}$ thus
we can get that $\bar{\phi} = \tilde{\bar{\phi}} \circ \phi$ is
determined by the unique
relation $\phi \to \bar{\phi}$ as soon as
$\tilde{\bar{\phi}}$ is unique.
Analogues one can get the unique relation
$\bar{\phi} \to \phi$.
\end{proof}
\label{lem:universalpropertyuniqueness}
\end{lemma}

The uniqueness does not mean existence and we should proof that such
object exists.
\begin{lemma}[About tensor product existence]
  Tensor product defined via \nameref{def:universalproperty} exists
  \begin{proof}
    Lets consider $\mathcal{E}$ the maps (functions) from
    $M \times N$ to $A$ as sets which are $0$ almost everywhere
    (i.e. outside of a finite set). For example we can consider delta
    functions:
    \[
    \delta_{m,n} : M \times N \to A
    \]
    such that
    \begin{eqnarray}
      \delta_{m,n}(m,n) = 1,
      \nonumber \\
      \delta_{m,n}(m',n') = 0 \mbox{ if } (m,n) \ne (m',n')
      \nonumber 
    \end{eqnarray}
    Then $\mathcal{E}$ is a A-\nameref{def:freemodule} with basis
    $\delta_{m,n}$. Thus we have a map of sets $M \times N \to
    \mathcal{E}$ such that $(m,n) \to \delta_{m,n}$ which is not bilinear
    but we can make it bilinear by means of changing $\mathcal{E}$.
    
    Let $\mathcal{F} \subset \mathcal{E}$ a submodule generated by
    $\delta_{m+m',n} - \delta_{m,n} - \delta_{m',n}$,
    $\delta_{m,n+n'} - \delta_{m,n} - \delta_{m,n'}$,
    $\delta_{am,n} - a\delta_{m,n}$,
    $\delta_{m,an} - a\delta_{m,n}$.
    \footnote{
      The basis is chosen to be a bilinear $\mod \mathcal{F}$, for instance
      $\delta_{m+m',n} = \delta_{m,n} + \delta_{m',n} \mod \mathcal{F}$
    }
    
    It can be shown that $M \times N \to \mathcal{E}/\mathcal{F}$ is
    bilinear
    \footnote{
      Follows from the basis choice
    }
    and has the desired \nameref{def:universalproperty}.

    Really lets we have the following bilinear map:
    $f: M \times N \to P$. Then we can consider the following linear
    map (\nameref{def:homomorphism})
    $f': \mathcal{E} \to P$ that sends $\delta_{m,n}$ to
    $f(n,m)$. Using the fact that $f$ is bilinear we can get
    \begin{eqnarray}
      f'(\delta_{m+m',n}) = f'(\delta_{m,n}) + f'(\delta_{m',n}),
      \nonumber \\
      f'(\delta_{m,n+n'}) = f'(\delta_{m,n}) + f'(\delta_{m,n'}),
      \nonumber \\
      f'(\delta_{am,n}) =  a f'(\delta_{m,n}),
      \nonumber \\
      f'(\delta_{m,an}) = a f'(\delta_{m,n})
      \nonumber
    \end{eqnarray}
    with the $f'$ linearity we have
    \begin{eqnarray}
      f'(\delta_{m+m',n}) = f'(\delta_{m,n} + \delta_{m',n}),
      \nonumber \\
      f'(\delta_{m,n+n'}) = f'(\delta_{m,n} + \delta_{m,n'}),
      \nonumber \\
      f'(\delta_{am,n}) =  a f'(\delta_{m,n}),
      \nonumber \\
      f'(\delta_{m,an}) = a f'(\delta_{m,n})
      \nonumber
    \end{eqnarray}

    The kernel $\ker f' = \mathcal{F}$ thus if we want to have a
    homomorphism to $P$ we have to replace $\mathcal{E}$ with
    $\mathcal{E}/\mathcal{F}$ that is also denoted by
    $M \otimes_A N$. In the case we will replace $f'$ with
    $\tilde{f}\left(\delta_{m,n} \mod \mathcal{F}\right) = f(m,n)$. As
    soon as the images for the basis is fixed the mapping is unique.
  \end{proof}
  \label{lem:tensorproductexistence}
\end{lemma}

We will denote $\phi\left(m,n\right) = \delta_{m,n} \mod \mathcal{F}$ as
$m \otimes n$. I.e our tensor product can be considered as the 
$\left(\otimes, M \otimes_A N\right)$ pair.


\begin{remark}
  Wrong idea is to define $M \otimes_A N$
  as a set of $m \otimes n$. I.e.
  $M \otimes_A N \neq \{m \otimes n\}$.
\end{remark}
The $M \otimes_A N$ is generated by $m \otimes n$ i.e.
$\forall x \in M \otimes_A N$ we have $x = \sum_{i = 1}^k m_i \otimes n_i$

\section{Tensor product of modules}

\subsection{Advantages of the universal property}
Now, you can ask  why haven't I just defined the tensor product by
this construction? Why am I talking of this universal property? 
And the answer is because it is easier to prove things this way. 
So advantages of the universal property is as follows: the proofs
become easy.

\subsection{Several examples of universal property usage}

\begin{example}[Commutativity proof]
  We want to prove that
  \[
  M \otimes_A N \cong N \otimes_A M
  \]

  We have the following bilinear map:
  $M \times N \to N \otimes_A M$ for which the pair
  $(m,n)$ is mapped to $n \otimes m$. Thus from
  \nameref{def:universalproperty} we have that there is a linear map
  (homomorphism)
  $\alpha: M \otimes_A N \to N \otimes_A M$:
  
  \begin{tikzpicture}[description/.style={fill=white,inner sep=2pt}]
    \matrix (m) [matrix of math nodes, row sep=3em,
      column sep=2.5em, text height=1.5ex, text depth=0.25ex]
            { M \times N & & N \otimes_A M \\
              & M \otimes_A N & \\ };
            %\draw[double,double distance=5pt] (m-1-1) – (m-1-3);
            \path[->]
            (m-1-1) edge node[auto] {$ (m,n) \to n \otimes m $} (m-1-3)
            (m-1-1) edge node[description] {$ (m,n) \to m \otimes n $} (m-2-2)
            (m-2-2) edge node[description] {$ \alpha $} (m-1-3);
  \end{tikzpicture}

  
  With the same
  construction we can get also and the inverse map $a^{-1}$ that sends
  $N \otimes_A M$ to $M \otimes_A N$:
  
  \begin{tikzpicture}[description/.style={fill=white,inner sep=2pt}]
    \matrix (m) [matrix of math nodes, row sep=3em,
      column sep=2.5em, text height=1.5ex, text depth=0.25ex]
            { M \times N & & M \otimes_A N \\
              & N \otimes_A M & \\ };
            %\draw[double,double distance=5pt] (m-1-1) – (m-1-3);
            \path[->]
            (m-1-1) edge node[auto] {$ (m,n) \to m \otimes n $} (m-1-3)
            (m-1-1) edge node[description] {$ (m,n) \to n \otimes m $} (m-2-2)
            (m-2-2) edge node[description] {$ \alpha^{-1} $} (m-1-3);
  \end{tikzpicture}
  
\end{example}

Also
\[
A \otimes_A M \cong M
\]

If we have that $M$ is generated by $e_1, e_2, \dots$ and
$N$ is generated by $\epsilon_1, \epsilon_2, \dots$ than
$M \otimes_A N$ is generated by pairs $e_i \otimes \epsilon_j$. It's
obvious.

More complex fact is the following
\begin{proposition}
  Let $M$ and $N$ are \nameref{def:freemodule}s
  with corresponding basises $e_1, e_2, \dots, e_n$ and
  $\epsilon_1, \epsilon_2, \dots, \epsilon_m$ than
  $M \otimes_A N$ is also free module with basis $e_i \otimes
  \epsilon_j$ where $1 \le i \le n$ and
  $1 \le j \le m$.
  \begin{proof}
    Lets define
    $f_{i_0,j_0}: M \times N \to A$ as a map that sends
    $\left(\sum a_i e_i, \sum b_j \epsilon_j\right)$ to
    $a_{j_0} b_{j_0}$. It's bilinear
    \footnote{
      for example
      $\left(\sum (a_i + a_i') e_i, \sum b_j \epsilon_j\right)$ is
      sent to  $(a_{j_0} + a_{j_0}') b_{j_0}$.
    }
    so it factors through the tensor product
    $\tilde{f}_{i_0, j_0} : M \otimes_A N \to A$. The map
    $\tilde{f}_{i_0, j_0}$ sends $e_{i_0} \otimes \epsilon_{j_0}$ to 1
    and all others to 0.
    \footnote{
      Because $f = \tilde{f} \phi$ i.e.
      \begin{eqnarray}
      a_{j_0} b_{j_0} =
      f_{i_0,j_0}\left(\sum a_i e_i, \sum b_j\epsilon_j\right) =
      \nonumber \\
      =  f_{i_0,j_0}\left(
      \phi\left(\sum a_i e_i, \sum b_j\epsilon_j\right)\right) = 
      \nonumber \\
      =
      \tilde{f}_{i_0,j_0}\left(\sum a_i e_i \otimes \sum b_j
      \epsilon_j\right) =
      \sum_{i,j} a_i b_j \tilde{f}_{i_0,j_0}(e_i \otimes \epsilon_j).
      \nonumber
      \end{eqnarray}
    }
    So if
    \[
    \sum \alpha_{ij} e_i \otimes \epsilon_j = 0
    \]
    then applying $\tilde{f}_{i_0, j_0}$ for all indices one can get
    $\forall i,j: \alpha_{ij} = 0$.
    \footnote{
      because $\tilde{f}$ should be linear.
    }
  \end{proof}
  \label{prop:lec4_1}
\end{proposition}

In particular for the \nameref{def:vectorspace} the tensor product is
defined in the same way (as just proved in the proposition
\ref{prop:lec4_1}): the tensor product of 2 vector spaces with
basises $e_1, e_2, \dots, e_n$ and
$\epsilon_1, \epsilon_2, \dots, \epsilon_m$ is another vector space
with the following basis $e_i \otimes \epsilon_j$ i.e. the definition
does not take into consideration the \nameref{def:universalproperty}.

\begin{proposition}[Associative]
  \[
  \left(M_1 \otimes_A M_2\right) \otimes_A M_3
  \cong
  M_1 \otimes_A \left(M_2 \otimes_A M_3\right)
  \]
  \begin{proof}
    There is just a scratch of the proof.
    Introduce
    $M_1 \otimes_A M_2 \otimes_A M_3$
    as a universal object for 3-linear maps and show that 2 considered
    parts are isomorphic each other.
  \end{proof}
  \label{prop:lec4_Associative}
\end{proposition}

\section{Base change}

Let $A$ is a \nameref{def:ring} and $B$ is $A$-algebra. Let also $M$
is an $A$-\nameref{def:module} and $N$ is $B$-module.

I can of course make $N$ into $A$-module (just forgetting the
additional $A$-algebra structure). But we can also make $B$-module of
$M$, that is not a trivial thing, by considering $B \otimes_A M$.
We can introduce $B$-module structure on $B \otimes_A M$ by
\footnote{
  I.e. we introduced $B$-algebra operations for objects from
  $B \otimes_A M$. See also definition \ref{def:kalgebra}.
}
\[
b \cdot \left(b' \otimes m \right) = \left( b \cdot b' \right) \otimes m 
\]

\begin{example}[The complexification of a real vector space]
  We can ``make'' $\mathbb{R}^{2n}$ from $\mathbb{C}^n$ by forgetting
  the complex structure.
  \footnote{
    In the case we have ring $A = \mathbb{R}$ and $B = \mathbb{C}$ -
    $A$ algebra. $A$ - module is the following vector space
    $M = \mathbb{R}^{2n}$ and $B$ - module is $N = \mathbb{C}^n$.
  }
  The $\mathbb{C}^n$ has the following basis $e_1, \dots, e_n$.
  The $\mathbb{R}^{2n}$ has the following one
  $e_1, \dots, e_n, i e_1, \dots, i e_n$. Now we forgot about
  multiplication rules for $i = \sqrt{-1}$ and denote $i e_i$ as
  $v_i$. In the case the basis for $\mathbb{R}^{2n}$ is the following
  one: $e_1, \dots, e_n, v_1, \dots, v_n$.

  But we can also do the following constructions
  \[
  \mathbb{R}^n \rightarrow
  \mathbb{C}^n = \mathbb{C} \otimes \mathbb{R}^n \rightarrow
  \mathbb{R}^{2n}
  \]
  for the $\mathbb{C}^n$ basis we have
  $1 \otimes e_1, \dots, 1 \otimes e_n$ and for $\mathbb{R}^{2n}$ -
  $1 \otimes e_1, \dots, 1 \otimes e_n, i \otimes e_1, \dots, i
  \otimes e_n$.
\end{example}

\begin{proposition}
In general we have the following. If $M$ - free $A$ - module with
basis $e_1, \dots, e_n$ then $B \otimes_A M$ is a free $B$ module with
basis $1_B \otimes e_1, \dots, 1_B \otimes e_n$.
\begin{proof}
  The proof is the same as at proposition \ref{prop:lec4_1}. Again we
  construct certain bilinear maps and say that those factor over the
  tensor product and this implies that certain families are linearly
  independent.

  Really lets define bilinear map $f_{i_0}: B \times M \to A$ such that
  \[
  f_{i_0}\left(b, \sum_{i=1}^n m_i e_i\right) = b m_{i_0} e_{i_0}
  \]
  so there exists a linear map $\tilde{f}_{i_0}$ such that
  \[
  \tilde{f}_{i_0}\left(b \otimes \sum_{i=1}^n m_i e_i\right) =
  b \tilde{f}_{i_0}\left(1_B \otimes \sum_{i=1}^n m_i e_i \right) =
  b m_{i_0}
  \]
  i.e. it sends $1_B \otimes e_{i_0}$ to 1 and all others $1_B \otimes
  e_i$ to 0.
  Thus the following sum $\sum \alpha_i 1_B \otimes e_i$ is equal to 0
  if all $\alpha_i = 0$ i.e. $\alpha_i 1_B \otimes e_i$ forms a basis.
\end{proof}
\label{prop:lec4_Addon}
\end{proposition}

\begin{remark}
We have the following maps.
\begin{itemize}
  \label{item:lec4_maps}
\item For A - modules: $\alpha: M \to B \otimes_A M$ such that
  $m  \to 1_B \otimes_A m$
\item For B - modules:  $\mu: B \otimes_A N \to N$ such that
  $b \otimes n \to b n$.
\end{itemize}
\label{rem:lec4_maps}
\end{remark}

\begin{theorem}[Base-change]
  Let $A$ is a \nameref{def:ring} and $B$ is $A$-algebra. Let also $M$
  is an $A$-\nameref{def:module} and $N$ is $B$-module.
  \[
  Hom_A\left(M, N\right)
  \leftrightarrow
  Hom_B\left(B \otimes_A M, N\right)
  \]
  I.e. the homomorphisms are the same or in other words the
  corresponding groups of homomorphisms are isomorphic. 
  \begin{proof}
    First of all we have
    \footnote{
      One homomorphism from  $Hom_B\left(B \otimes_A M, N\right)$
    }
    \nameref{def:homomorphism} $f: B \otimes_A M \to N$. We also have
    the following map (see remark \ref{rem:lec4_maps}):
    $\alpha: M \to B \otimes_A M$. 
    Thus $f \cdot \alpha: M \to N$ i.e.
    we can set the following relation
    \[
    \hat{f} : Hom_B\left(B \otimes_A M, N\right)
    \to Hom_A\left(M, N\right)
    \]
    such that $\hat{f}\left(f\right) = f \alpha$.

    In other direction we have $g: M \to N$ thus
    $id_B \otimes g: B \otimes_A M \to B \otimes_A N$ but
    ( see remark \ref{rem:lec4_maps}) we have
    $\mu: B \otimes_A N \to N$ i.e. we have the following relation
    \[
    \hat{g} : Hom_A\left(M, N\right) \to
    Hom_B\left(B \otimes_A M, N\right)
    \]
    such that
    \[
    \hat g(g) = \mu \cdot (id_B \otimes g).
    \]   
    And we can check that those maps ($\hat{f}$ and $\hat{g}$) are
    mutually inverse. 
    \footnote{
      ???
    }
  \end{proof}
  \label{thm:basechange}
\end{theorem}

\section{Examples. Tensor product of algebras}

\begin{proposition}
  If $I \subset A$ - an \nameref{def:ideal} so my $B$ - $A$ algebra
  will be $B = A/I$ then
  \[
  A/I \otimes_A M \cong M/IM
  \]
  \begin{proof}
    We have map $\alpha: M \to B \otimes_A M = A/I \otimes_A M$
    (see remark \ref{rem:lec4_maps}) 
    which sends $m$ to $\bar{1} \otimes m$.
    \footnote{
      $\bar{1} = 1_A + I$
    }
    The map sends $IM$ to $0$
    because
    $\forall i \in I, m \in M : im \to \bar{1} \otimes im = \bar{i} \otimes m$
    because the tensor product is over $A$ and everything is $A$
    linear and as result $\bar{1} \otimes im = \bar{i} \otimes m$,
    but $\bar{i} \otimes m = \bar{0} \otimes m = 0$.
    \footnote{
      because
      $\bar{i} = 0 \mod I$
    }
    Thus $\alpha$ sends $IM$ to 0. So $\alpha$ induces
    $\bar{\alpha}: M/IM \to A/I \otimes_A M$ such that
    $\bar{\alpha}\left(\bar{m}\right) = \bar{1} \otimes m$.

    For other direction we apply \nameref{thm:basechange} theorem. The
    following map of $A$-modules
    \[
    M \to M/IM
    \]
    gives us the following map of $B$-modules
    \[
    \bar{\beta}: B \otimes_A M \to M/IM
    \]
    i.e.
    \[
    \bar{\beta}: A/I \otimes_A M \to M/IM
    \]
    that sends $\bar{a} \otimes m$ to $\bar{am}$
    Ones check again that this inverse to $\bar{\alpha}$.
    \footnote{
      For example
      \(
      \bar{\beta}\left(\bar{\alpha}\left(\bar{m}\right)\right) =
      \bar{\beta}\left(\bar{1} \otimes m\right) = \overline{1 \cdot m} = \bar{m}
      \)
    }
  \end{proof}
  \label{prop:lec4_prop2}
\end{proposition}

Several examples:
\begin{example}
  Let $\mathbb{Z}/2\mathbb{Z} \otimes_\mathbb{Z}
  \mathbb{Z}/3\mathbb{Z}$ what will we obtain?
  \[
  \mathbb{Z}/2\mathbb{Z} \otimes_\mathbb{Z} \cong
  ^{\mathbb{Z}/3\mathbb{Z}}/_{(2) \cdot \mathbb{Z}/3\mathbb{Z}}
  \]
  but 2 is invertible: $2^{-1} = -1 \mod 3$ thus
  $(2)\mathbb{Z}/3\mathbb{Z} =\mathbb{Z}/3\mathbb{Z}$ and as result
   \[
  \mathbb{Z}/2\mathbb{Z} \otimes_\mathbb{Z} \cong
  ^{\mathbb{Z}/3\mathbb{Z}}/_{ \mathbb{Z}/3\mathbb{Z}} = 0
  \]
\end{example}

\begin{example}
  \[
  B \otimes_A A\left[X\right] \cong B\left[X\right]
  \]
  and more interesting one
  \[
  B \otimes_A A\left[X\right]/(P) \cong B\left[X\right]/(P),
  \]
  there $(P)$ becomes an ideal generated by $P$ in
  $B\left[X\right]$.
\end{example}

\subsection{Tensor product of A algebras}

Let $B, C$ are $A$-algebras. The following maps form an algebra
structure on $A$:
\[
\alpha: A \to B
\]
\[
\beta: A \to C
\]
New $A$-algebra $B \otimes_A C$: is a ring with respect to the
following operation
\footnote{
  that makes it $A$-algebra (see \nameref{def:kalgebra})
}
\[
\left(b \otimes c \right) \cdot \left(b' \otimes c'\right) =
\left(b \cdot b'\right) \otimes \left(c \cdot c'\right)
\]

The tensor product has the following 
\begin{definition}[Universal property]
  Let we have the following maps
  \begin{eqnarray}
    \alpha: A \to B,
    \nonumber \\
    \beta: A \to C,
    \nonumber \\
    \phi: b \in B \to b \otimes 1 \in B \otimes_A C,
    \nonumber \\
    \psi: c \in C \to 1 \otimes c \in B \otimes_A C
    \nonumber
  \end{eqnarray}

  Then for any $A$-algebra $D$ one has
  \[
  Hom_A\left(B \otimes_A C, D\right)
  \leftrightarrow
  Hom_A\left(B, D\right) \times
  Hom_A\left(C, D\right)
  \]
  i.e. if I have some \nameref{def:homomorphism}
  $h \in Hom_A\left(B \otimes_A C, D\right)$ this is the same as
  giving 2 homomorphisms
  $f \in Hom_A\left(B, D\right)$ and
  $g \in Hom_A\left(C, D\right)$ such that all maps in the following
  diagram commute (see \nameref{def:commutativediagram}).
  
  \begin{tikzpicture}[description/.style={fill=white,inner sep=2pt}]
    \matrix (m) [matrix of math nodes, row sep=3em,
      column sep=2.5em, text height=1.5ex, text depth=0.25ex]
            { D & & B \otimes C & \\
              & B & & C & \\
              & & A & & \\
            };
            %\draw[double,double distance=5pt] (m-1-1) – (m-1-3);
            \path[->]
            (m-2-2) edge node[description] {$ \phi $} (m-1-3)
            (m-2-4) edge node[description] {$ \psi $} (m-1-3)
            (m-3-3) edge node[description] {$ \alpha $} (m-2-2)
            (m-3-3) edge node[description] {$ \beta $} (m-2-4)
            (m-1-3) edge [bend right=15] node[description]  {$ h $}
            (m-1-1);
            \path[->, dashed]
            (m-2-2) edge [bend left=15] node[description] {$ f $} (m-1-1)
            (m-2-4) edge [bend left=15] node[description] {$ g $} (m-1-1)
            ;
  \end{tikzpicture}

  Thus if we have $h$ then we can define
  $f = h \cdot \phi$ and $g = h \cdot \psi$. And conversely if I have
  $f$ and $g$ then I can define $h$ by the following rule:
  \[
  h\left(b \otimes c\right) = f(b) \cdot g(c)
  \]
\end{definition}

Let consider the following example
\[
\mathbb{C} \otimes_\mathbb{R} \mathbb{C}
\cong
\mathbb{C} \otimes_\mathbb{R}
{\mathbb{R}\left[X\right]}/{\left(x^2 + 1\right)}
\cong
^{\mathbb{C}\left[X\right]}/_{\left(x^2 + 1\right)}
\]
but by \nameref{thm:chineseremainder}
\[
^{\mathbb{C}\left[X\right]}/_{\left(x^2 + 1\right)}
\cong
^{\mathbb{C}\left[X\right]}/_{\left(x + i\right)}
\times
^{\mathbb{C}\left[X\right]}/_{\left(x - i\right)}
\cong
\mathbb{C} \times \mathbb{C}
\]
As result we have that
$\mathbb{C} \otimes_\mathbb{R} \mathbb{C}$ is not a field because it
has zero divisors.


\section{Relatively prime ideals. Chinese remainder theorem}

\begin{definition}[Relatively prime ideals]
  Let $A$ - \nameref{def:ring} and $I,J$ are \nameref{def:ideal}s.
  $I$ and $J$ are relatively prime if $I + J = A$.
  \label{def:relprimeideals}
\end{definition}

\begin{lemma}
  \begin{enumerate}
  \item If $I,J$ are relatively prime then $IJ = I \cap J$
  \item If $I_1, \dots, I_k$ relatively prime with $J$ then
    $\prod_{i =1}^k I_i = I_1 \cdot \dots \cdot I_k$ is also
    relatively prime with $J$.
  \item If $I, J$ relatively prime then $I^k$ and $J^l$ are also
    relatively prime for any $l$ and $k$.
  \end{enumerate}
  \begin{proof}
    \begin{enumerate}
    \item
      The following one  $IJ \subset I \cap J$ is clear
      \footnote {
        Assuming that $I$ and $J$ commute we have if
        $x \in IJ$ then $x \in I$ and if
        $x \in JI$ then $x \in J$ i.e. $x \in I \cap J$.
      }
      If $I$ and $J$ are relatively prime then $1_A = i + j$ for some
      $i \in I$ and $j \in J$. Thus $\forall x \in I \cap J$ we have the
      following ones:
      $x i \in IJ$ and $x j \in IJ$ and as result
      \[
      x = x i + x j \in IJ
      \]
      i.e. $I \cap J \subset IJ$.
    \item Suppose for simplicity that $k = 2$. In the case we have
      $1 = i_1 + j_1 = i_2 + j_2$ where $i_1 \in I_1, i_2 \in I_2$ and
      $j_1, j_2 \in J$.
      we also have
      \[
      1 = \left(i_1 + j_1\right) \left(i_2 + j_2\right) =
      i_1 i_2 + \left(j_1 i_2 + j_2 i_1 + j_1 j_2\right)
      \in I_1 I_2 + J
      \]
      thus $\forall x \in A$ we have
      \[
      x = 1 x =
      i_1 i_2 x + \left(j_1 i_2 + j_2 i_1 + j_1 j_2\right) x
      \in I_1 I_2 + J
      \]
    \item is obvious
      \footnote{
        It follows from the 2 because we can assume $I_i = I$ and will
        get that $\forall k, I^k$ is relatively prime with $J$. From
        other side we can assume $I_i = J$ and $J = I^k$ and conclude
        that $J^l$ is relatively prime with $I^k$. 
      }
    \end{enumerate}
  \end{proof}
  \label{lem:aboutrelprimeideals}
\end{lemma}

\begin{theorem}[Chinese remainder theorem]
  Let $I_1, \dots, I_n$ - ideals and map
  $\pi: A \to A/I_1 \times \dots \times A/I_n$ defined as follows
  \[
  \pi(a) = \left(a \mod I_1, \dots, a \mod I_n\right)
  \]
  The kernel $\ker \pi = I_1 \cap \dots \cap I_n$.

  The $\pi$ is \nameref{def:surjection} if and only if $I_1, \dots,
  I_n$ are pairwise relatively prime.

  In that case
  \[
  A/\cap I_k
  \cong
  A/\prod I_k
  \cong
  \prod \left(A/I_k\right)
  \]
  \footnote{
    see \nameref{thm:firstisomorphism}
  }
  
  \begin{proof}
    Let $\pi$ is \nameref{def:surjection}. In the case
    $\exists a_i \in A$ such that
    \[
    \pi(a_i) = \left(0, \dots, 1 (\mbox{ in }i\mbox{-th place }),
    0, \dots, 0\right)
    \]
    i.e.
    $a_i \mod I_j = 0$ or $a_i \in I_j$ for $i \ne j$. We also have
    that $1 - a_i \in I_i$.
    Thus $\forall j, \exists a_i \in I_j, a_k \in I_i$ such that
    $1 = a_i + a_k$ thus $A = I_j + I_i$ i.e. $I_i$ relatively prime
    with any $I_j$.

    Conversely if $I_i$ is relatively prime with any $I_j$ where $j
    \ne i$ then it also relatively prime with the product (see lemma
    \ref{lem:aboutrelprimeideals}) $\prod_{j \ne i} I_j$. In the case
    $\exists x_i \in I_i, y_i \in \prod_{j \ne i} I_j$ such that
    $1 = x_i + y_i$ in the case
    \[
    \pi(y_i) = \left(0, \dots, 1 (\mbox{ in }i\mbox{-th place }),
    0, \dots, 0\right)
    \]
    and $\forall b_i \in A/I_i$
    \[
    \pi \left(\sum_{i = 1}^n b_i y_i\right) =
    \left(b_1, \dots, b_n\right)
    \]
    i.e. $\pi$ is surjective.
  \end{proof}
  \label{thm:chineseremainder}
\end{theorem}

Let $K$ is a field and $A$ is a finite (finite dimensional vector
space) $K$-algebra.

\begin{proposition}
  \begin{enumerate}
    \item If $A$ is an \nameref{def:integraldomain} then $A$ is a
      field.
    \item (replacing the first one) Any
      \nameref{def:primeideal} of $A$ is \nameref{def:maxideal}
  \end{enumerate}
  \begin{proof}
     Well, I shall prove only the first part, the second part is just
     a consequence of definitions. In fact, a factor over a prime
     ideal, a quotient over a prime ideal is an integral domain, and a
     quotient over a maximal ideal is a field. If you don't know this,
     please look it up in any book.

     Lets prove the first part. \nameref{def:integraldomain} means
     that there is no zero divisors i.e. $\forall a \in A$
     \footnote{
       $a \ne 0_A$
     }
     multiplication by $a$ is \nameref{def:injection}. $A$ is finite
     dimensional \nameref{def:vectorspace} that implies that $\times a$ is an
     \nameref{def:isomorphism},
     \footnote{
       $\times a$ sends a vector space  into another vector space with
       the same dimension. But with lemma \nameref{lem:vsisomorphism} one
       can get that the spaces are isomorphic each others and as
       result the operation $\times a$ is an \nameref{def:isomorphism}.
     }
     in particular \nameref{def:surjection} i.e. $\exists b \in A$
     such that $b \times a = 1$ i.e. $a$ is invertible therefore
     $A$ is field.     
  \end{proof}
  \label{prop:lec4_ideals}
\end{proposition}

\section{Structure of finite algebras over a field. Examples}

\begin{theorem}[Structure of finite $K$-algebra]
  Let $A$ be a finite $K$-algebra i.e.
  $\dim_K A < \infty$. Then
  \begin{enumerate}
  \item There are only finitely many \nameref{def:maxideal}s
    $m_1, \dots, m_r$ in $A$
  \item Let $J = m_1 \cap \dots \cap m_r = m_1 \dots m_r$.
    \footnote{
      Since the ideals are relatively prime the intersection is the
      same as the product of the ideals
    }
    Then
    $J^n = 0$ for some $n$
  \item
    $A \cong A/{m_1^{n_1}} \times \dots \times A/{m_r^{n_r}}$ for
    some $n_1, \dots, n_r$.    
  \end{enumerate}
  \begin{proof}
    \begin{enumerate}
    \item
      Let $m_1, \dots, m_i$ be a several maximal ideals. By
      \nameref{thm:chineseremainder} we have
      \footnote{
        Maximal ideals are relatively prime because
        in a commutative ring with unity, every
        \nameref{def:maxideal} is a
        \nameref{def:primeideal} see also proposition
        \ref{prop:lec4_ideals}. 
      }
      \[
      A/m_1 \dots m_i \cong
      A/m_1 \times \dots \times A/m_i.
      \]
      We know that $A$ as well as $A/m_1 \dots m_i$ and $A/m_k$ are
      finite dimensional $K$-\nameref{def:vectorspace}. Thus we have the
      following relations
      \[
      \dim_K A \ge \dim_K  A/m_1 \dots m_i  =
      \sum_{j=1}^i dim_K A/m_j \ge i.
      \]
      Therefore if $N$ the number of maximal ideals then
      $\dim_K A \ge N$ i.e. the number of maximal ideal is limited by
      the vector space dimension.
    \item
      $J = m_1 \cap \dots \cap m_r = m_1 \dots m_r$ is finite
      dimensional vector space over $K$ as well as its powers
      $J^k$. We have the following sequence
      \footnote {
        Let $j \in J \subset A$ then $\forall y \in A: j y \in J$.
        But if $x \in JJ$ then $x = jj = j y$ there $y = i \in A$. and
        as soon as $j \in J$ then $x = jj$ is also an element of
        $J$. As result $J^2 \subseteq J$.
      }
      \[
      \dots \subseteq J^k \subseteq \dots \subseteq
      J^2 \subseteq J.
      \]
      and the sequence should stop somewhere i.e.
      $\exists n$ such that $J^n = J^{n+1}$. We claim that $J^n = 0$
      in the case. Indeed if not we have the following basis of $J^n$:
      $e_1, \dots, e_s$. And as soon as $J^n = J J^n$ we can write a
      vector $e_i \in J^n$ as a vector from $J^n$ multiplied on an
      object from $J$ i.e.
      \[
      e_i = \sum \lambda_{ij} e_j,
      \]
      there $e_j \in J^n, \lambda_{ij} \in J$. Thus if
      $M = Id - \lambda_{ij}$
      \[
      M \cdot \left(
      \begin{array}{c}
        e_1 \\
        \vdots \\
        e_s
      \end{array}
      \right) = 0.
      \]
      It's possible over ring to find a matrix $\tilde{M}$ such that
      \[
      \tilde{M} M = \det M \cdot Id,
      \]
      i.e.
      \[
      \det M \cdot \left(
      \begin{array}{c}
        e_1 \\
        \vdots \\
        e_s
      \end{array}
      \right) = 0.
      \]
      But $\det M = 1 + \lambda$ where $\lambda \in J$.
      \footnote{
        Because the $\det$ consists of the following 
        items $\prod (1-\lambda_{ii}) = 1 + (-1)^s\prod \lambda_{ii}$
        and $\prod \lambda_{ij}$. The sum of the items ($\det$)
        consists of $1$ and another sum in which all items are from
        $J$. Thus the second sum is an element of $J$ i.e.
        $\det M = 1 + \sum \prod \lambda_{ij} = 1 + \lambda$.
      }
      Since  $J = m_1 \cap \dots \cap m_r$ then
      $\forall i: \lambda \in m_i$
      so $\nexists i$ such that $1 + \lambda \in m_i$
      \footnote{
        ???
      }
      thus $1 + \lambda$ is invertable
      \footnote {
        ???
      }
      therefore $e_1 = \dots = e_s = 0$
      \footnote{
        Because $\det M = 1 + \lambda \ne 0$
      }
    \item Using part 2
      $\exists n_1, \dots, n_r$ such that
      $m_1^{n_1} \dots m_r^{n_r} = 0$ (for example we can assume
      $n_i = n$). Then by \nameref{thm:chineseremainder}
      \[
      A \cong
      A/{m_1^{n_1}} \times \dots \times A/{m_r^{n_r}}.
      \]
      We used the following facts:
      \begin{itemize}
      \item $A = A /m_1^{n_1} \dots m_r^{n_r}$
        \footnote{
          Because $A = A/\{0\}$. For example if $I = \{0\}$ and
          $x \in A$ then $\bar{x} \in A/I$ if $\bar{x} = x + I$. In
          our case $\bar{x} = x + \{0\} = x$ i.e. $\forall x \in A$ we
          have $x \in A/\{0\}$.
          (See also \nameref{def:quotientring})
        }
      \item $m_i^{n_i}$ are pairwise relatively prime
        \footnote{
          As soon as $\{m_i\}$ - \nameref{def:maxideal}s and as result
          \nameref{def:primeideal}s then with lemma
          \ref{lem:aboutrelprimeideals} one can get that
          $\forall i \ne j$ $m_i^{n_i}$ is relatively prime with
          $m_j^{n_j}$.
        }
      \end{itemize}
    \end{enumerate}
  \end{proof}
  \label{thm:structurefinitekalgebra}
\end{theorem}

\begin{remark}
  The $n_i$s are not uniquely defined. For example
  \[
  A = ^{K\left[X\right]}/_{\left(X^2 \left(X+1\right)^3\right)}.
  \]
  We have 2 ideals there:
  $m_1 = (X)$ and $m_2 = (X+1)$.
  We of course have
  \[
  A \cong A/m_1^2 \times A/m_2^3
  \]
  but also we have
  \[
  A \cong A/m_1^3 \times A/m_2^3
  \]
  as soon as $m_1^2 = m_1^3$ in $A$:
  $(X)^2 \subset (X)^3$ but also
  $(X)^3 \subset (X)^2$
  \footnote{
    ??? exercise 
  }
\end{remark}

Several examples:
\[
\mathbb{C} \otimes_\mathbb{R} \mathbb{C} =
\mathbb{C} \times \mathbb{C}.
\]

Another example
\[
\mathbb{Q}\left(\sqrt{2}\right)
\otimes
\mathbb{Q}\left(\sqrt{3}\right) =
\mathbb{Q}\left(\sqrt{2}, \sqrt{3}\right)
\]
And you see that those algebras are products of fields.
So all $n_i$'s may be taken equal to 1. In other words, we don't have
\nameref{def:nilpotent}s in our algebra. So, it is a reduced
algebras. Reduced, by definition, is without nilpotents.     
It's general phenomena because the presence of nilpotents is due to the
inseparability of extensions ome from inseparable extensions.  
