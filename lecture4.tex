%% -*- coding:utf-8 -*-
\chapter{Tensor product. Structure of finite K-algebras}
This is a digression on commutative algebra. We introduce and study
the notion of tensor product of modules over a ring. We prove a
structure theorem for finite algebras over a field (a version of the
well-known "Chinese remainder theorem").


\section{Definition of tensor product}

\subsection{Summary for previous lectures}
We considered finite \nameref{def:fextension1} $L$ i.e
$\left[L:K\right] < \infty$. We also saw that if
$L$ is generated by a finite number of \nameref{def:degsepelem}s 
$\alpha_1, \dots, \alpha_r$ then the number of
\nameref{def:homomorphism}s over $K$ from $L$ to $\bar{K}$ denoted by
$\left|Hom_K\left(L, \bar{K}\right)\right|$ is equal to
$\left[L:K\right]$. In general
\[
\left[L:K\right]_{sep} = 
\left|Hom_K\left(L, \bar{K}\right)\right| \le
\left[L:K\right].
\]

For $L = K\left(\alpha\right)$ it is clear because the number of
homomorphisms is equal to the number of roots of the
\nameref{def:minpolynomial} $P_{min}\left(\alpha, K\right)$. In
general one can use induction and multiplicativity of the degree
$\left[L:K\right]$ and number of homomorphisms (see theorem
\nameref{thm:lec3_3}). Thus  separable extension was exactly an
extension  which had the right number of homomorphisms into the
algebraic closure.

Our next goal is to characterize the separability in the terms of
tensor product.

\subsection{Tensor product}


\section{Tensor product of modules}

\section{Base change}

\section{Examples. Tensor product of algebras}

\section{Relatively prime ideals. Chinese remainder theorem}

\section{Structure of finite algebras over a field. Examples}
