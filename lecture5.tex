%% -*- coding:utf-8 -*-
\chapter{Structure of finite K-algebras continued}

We apply the discussion from the last lecture to the case of field
extensions. We show that the separable extensions remain reduced after
a base change: the inseparability is responsible for eventual
nilpotents. As our next subject, we introduce normal and Galois
extensions and prove Artin's theorem on invariants.

\section{Structure of finite K-algebras, examples (cont'd)}

Last time we have seen that a finite $K$-algebra $A$
($\left[A:K\right] < \infty$) has only finitely many maximal ideals
$m_1, \dots, m_r$ and the following equation holds (see theorem
\ref{thm:structurefinitekalgebra}):
\[
A \cong A/m_1^{k_1} \times \dots \times A/m_r^{k_r}
\]
This is a general for of \nameref{thm:chineseremainder}.

\begin{example}
  Let
  \[
  A = K\left[X\right]/\left(F\right)
  \]
  And the polynomial $F = P_1^{k_1} \dots P_r^{k_r}$ is not necessary
  reducible then by the  \nameref{thm:chineseremainder}
  \footnote{
    and theorem \ref{thm:structurefinitekalgebra}
  }
  one can get
  \[
  A \cong
  K\left[X\right]/\left(P_1 \mod F \right)^{k_1} \times \dots
  \times K\left[X\right]/\left(P_r \mod F \right)^{k_r}
  \]
\end{example}

\begin{definition}[Nilpotent element]
  Let $A$ is a \nameref{def:ring} than $x \in A$ is nilpotent if $x \ne 0$ but
  $\exists k: x^k = 0$.
  \footnote{
    Alternative definition from \cite{wiki:nilpotent}: An element,
    $x$, of a ring, $R$, is called nilpotent if there exists 
    some positive integer, $n$, such that $x^n = 0$.
  }
  \label{def:nilpotent}
\end{definition}

\begin{definition}[reduced]
  $K$-algebra $A$ is reduced if it has no \nameref{def:nilpotent}s.
  Or in other words
  \footnote{
    ??? require proof
  }
  if in the decomposition
  \[
  A \cong A/m_1^{k_1} \times \dots \times A/m_r^{k_r}
  \]
  $\forall i: k_i = 1$.
  Or 
  \footnote{
    $A/m_i$ is a field as soon as $m_i$ is a \nameref{def:maxideal}
  }
  if $A$ is a product of fields. 
  \label{def:reduced}
\end{definition}

\begin{definition}[local]
  \nameref{def:ring} $A$ is called local if it has only one
  \nameref{def:maxideal} i.e. $A \cong A/m^k$. 
  \label{def:local}
\end{definition}

If $A$ is local then all elements of $A$ are nilpotents
\footnote{
  ???
} i.e. any element of $A$ is a identity,zero or nilpotent.

Most of our last examples were examples of reduced $K$-algebras such
as
\[
\mathbb{C} \otimes_{\mathbb{R}} \mathbb{C} =
\mathbb{C} \times \mathbb{C}
\]
or
\[
\mathbb{Q}\left(\sqrt{2}\right) \otimes_{\mathbb{Q}}
\mathbb{Q}\left(i\right) =
\mathbb{Q}\left(i, sqrt{2}\right)
\]
that is a field and if we start producing similar examples then mostly
they are reduced. Well, why? Because in fact the presence of
nilpotents has to do with inseparability. The presence of nilpotents
reflects inseparability.

So let me give you one more example. Finally of tensor product of
extensions which is not reduced. Let $K$ be a field of characteristic
$p$, for instance $\mathbb{F}_p$. Consider a field of rational
functions over $K$: $K\left(X\right)$. We will consider
$K\left(X\right)$ as an extension of $K\left(X^p\right)$ (or with new
variable $Y = X^p$ - $K\left(Y\right)$. We will be interested in
\(
K\left(X\right)
\otimes_{K\left(Y\right)}
K\left(X\right)
\) where $X$ is a $p$th root of $Y$ so
\footnote{
  as soon as
  $K\left(X\right) = K\left[T\right]/\left(T^p - Y\right)$
}
\begin{eqnarray}
K\left(X\right)
\otimes_{K\left(Y\right)}
K\left(X\right)
\cong
\nonumber \\
\cong
K\left(X\right)
\otimes_{K\left(Y\right)}
K\left[T\right]/\left(T^p - Y\right)
\cong
\nonumber \\
\cong
K\left(X\right)\left[T\right]/\left(T^p - Y\right) =
\nonumber \\
=
K\left(X\right)\left[T\right]/\left(T^p - X^p\right) =
K\left(X\right)\left[T\right]/\left(T - X\right)^p
\nonumber
\end{eqnarray}
where $T$ is another variable. As result we have got a ring with
nilpotents for example $T - X$ and of course the reason is that our
extension $K\left(X\right)$ is pure inseparable extension (see
definition \ref{def:inseparabledegree}) of $K\left(Y\right)$.  



\section{Separability and base change}
\section{Separability and base change (cont'd). Primitive element theorem}
\section{Examples. Normal extensions}
\section{Galois extensions}
\section{Artin's theorem}
