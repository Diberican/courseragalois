%% -*- coding:utf-8 -*-
\chapter{Structure of finite K-algebras continued}

We apply the discussion from the last lecture to the case of field
extensions. We show that the separable extensions remain reduced after
a base change: the inseparability is responsible for eventual
nilpotents. As our next subject, we introduce normal and Galois
extensions and prove Artin's theorem on invariants.

\section{Structure of finite K-algebras, examples (cont'd)}

Last time we have seen that a finite $K$-algebra $A$
($\left[A:K\right] < \infty$) has only finitely many maximal ideals
$m_1, \dots, m_r$ and the following equation holds (see theorem
\ref{thm:structurefinitekalgebra}):
\[
A \cong A/m_1^{k_1} \times \dots \times A/m_r^{k_r}
\]
This is a general for of \nameref{thm:chineseremainder}.

\begin{example}
  Let
  \[
  A = K\left[X\right]/\left(F\right)
  \]
  And the polynomial $F = P_1^{k_1} \dots P_r^{k_r}$ is not necessary
  reducible then by the  \nameref{thm:chineseremainder}
  \footnote{
    and theorem \ref{thm:structurefinitekalgebra}
  }
  one can get
  \[
  A \cong
  K\left[X\right]/\left(P_1 \mod F \right)^{k_1} \times \dots
  \times K\left[X\right]/\left(P_r \mod F \right)^{k_r}
  \]
\end{example}

\begin{definition}[Nilpotent element]
  Let $A$ is a ring than $x \in A$ is nilpotent if $x \ne 0$ but
  $\exists k: x^k = 0$.
  \footnote{
    Alternative definition from \cite{wiki:nilpotent}: An element,
    $x$, of a ring, $R$, is called nilpotent if there exists 
    some positive integer, $n$, such that $x^n = 0$.
  }
  \label{def:nilpotent}
\end{definition}

\begin{definition}[reduced]
  \nameref{def:ring} $A$ is reduced if it has no \nameref{def:nilpotent}s.
  Or in other words
  \footnote{
    ??? require proof
  }
  if in the decomposition
  \[
  A \cong A/m_1^{k_1} \times \dots \times A/m_r^{k_r}
  \]
  $\forall i: k_i = 1$.
  Or 
  \footnote{
    $A/m_i$ is a field as soon as $m_i$ is a \nameref{def:maxideal}
  }
  if $A$ is a product of fields. 
  \label{def:reduced}
\end{definition}

\begin{definition}[local]
  \nameref{def:ring} $A$ is called local if it has only one
  \nameref{def:maxideal} i.e. $A \cong A/m^k$. 
  \label{def:local}
\end{definition}

If $A$ is local then all elements of $A$ are nilpotents
\footnote{
  ???
} i.e. any element of $A$ is a identity,zero or nilpotent.

\section{Separability and base change}
\section{Separability and base change (cont'd). Primitive element theorem}
\section{Examples. Normal extensions}
\section{Galois extensions}
\section{Artin's theorem}
