%% -*- coding:utf-8 -*-
\chapter{Galois correspondence and first examples}
We state and prove the main theorem of these lectures: the Galois
correspondence. Then we start doing examples (low degree,
discriminant, finite fields, roots of unity).

\section{Some further remarks on normal extension. Fixed field}

Some definitions from previous lecture. $L$ over $K$ is
\nameref{def:galoisextension} if and only if it is a
\nameref{def:separableextension} and \nameref{def:normalextension} or
in other words $L$ is a \nameref{def:splittingfield} of a family of
separable irreducible polynomial over $K$. We also seen (see theorem
\ref{thm:lec5_4}) that in the case of finite extension
$\left[L:K\right] < \infty$ the number of automorphisms
$\left|Aut\left(L/K\right)\right| = \left[L:K\right]$.

There are several remarks on \nameref{def:normalextension}s which show
that the extensions behave sometimes differently compare to other
types of extensions. Especially we have seen for that an extension 
$L$ over $M$ over $K$ was finite or algebraic or separable or purely
inseparable if, and only if, it was true for $L$ over $M$ and $M$ over
$K$. So, for a normal extensions, this is not the case anymore.

\begin{remark}
  Let we have a tower of extensions $K \subset L \subset M$. If $M$ is
  normal over $K$ then of course the $M$ is normal over $L$. It is
  clear because if $M$ is a splitting field of a family of polynomials
  over $K$ the one can just consider them as being polynomials over
  $L$ and say that $M$ is a splitting field of ephemeral polynomials
  over $L$.

  But $L$ does not have to be normal over $K$ (see example
  \ref{ex:lec6_1}). 
\end{remark}

\begin{example}
  Consider
  \[
  \mathbb{Q} \subset
  \mathbb{Q}\left(\sqrt[4]{2}\right)
  \subset
  \mathbb{Q}\left(\sqrt[4]{2}, i\right)
  \]

  We have $\mathbb{Q}\left(\sqrt[4]{2}, i\right)$ to be a splitting
  field for polynomial $X^4 -2$ but
  $\mathbb{Q}\left(\sqrt[4]{2}\right)$ is just a
  \nameref{def:stemfield} (not \nameref{def:splittingfield}) for this
  polynomial. And as result $\mathbb{Q}\left(\sqrt[4]{2}\right)$ is
  not a normal over $\mathbb{Q}$.  
  \label{ex:lec6_1}
\end{example}

\begin{remark}
  A quadratic extension is normal. This is by formula for roots of a
  quadratic equation.
  \footnote {
    ??? Extensions with degree 2. 
  }

  If $P$ quadratic over $K$ has 1 root in $L$ then its another root is
  also in $L$.
\end{remark}

\begin{remark}
  One often has $K \subset L \subset M$ with $L$ normal over $K$, $M$
  normal over $L$ but $M$ not normal over $K$ (see example
  \ref{ex:lec6_2}). 
\end{remark}

\begin{example}
  Consider
  \[
  \mathbb{Q} \subset
  \mathbb{Q}\left(\sqrt{2}\right)
  \subset
  \mathbb{Q}\left(\sqrt[4]{2}\right)
  \]

  We have $\mathbb{Q}\left(\sqrt{2}\right)$ normal over $\mathbb{Q}$
  as well as $\mathbb{Q}\left(\sqrt[4]{2}\right)$ normal over
  $\mathbb{Q}\left(\sqrt{2}\right)$ because they both are quadratic
  extensions. But 
  $\mathbb{Q}\left(\sqrt[4]{2}\right)$ is not normal over $\mathbb{Q}$
  (as it was mentioned in example \ref{ex:lec6_2})
  \label{ex:lec6_2}
\end{example}

We also seen at last lecture the following definition:
\begin{definition}[Fixed field]
  If $L$ is a field and $G \subset Aut\left(L\right)$ then
  \[
  L^G = \left\{
  x \in L \mid \forall g \in G: g x = x 
  \right\}
  \]  
  is a fixed field
  \label{def:fixedfield}
\end{definition}

If we have a sub-field $K \subset L$ then we can consider the
following group of automorphisms of $L$ over $K$:
$Aut\left(L/K\right)$ in the case if $L$ is normal.
Because otherwise the group will be too small to give information
about $L$. But in the normal case it makes sense to consider  
the group of automorphisms of $L$ over $K$.

We have seen (see (\ref{eq:lec5_2})) that if $L$ is separable over $K$
then 
\[
L^{Aut\left(L/K\right)} = K
\]
This is because of the group of automorphisms was permuting the roots
over the minimal polynomial of $x$ over $K$. So, if it was fix and $x$
was on it, was meaning that $x$ was the only root over its minimal
polynomial. So this was meaning that $x$ was purely inseparable over
$K$.  

We also have seen (see theorem \ref{thm:artin}) that if $G$ is finite
the $L$ is \nameref{def:galoisextension} over $L^G$ and
$\left[L:L^G\right] = \left|G\right|$.

And now we are going to summarize all these in a theorem which is in
fact the main subject of this lecture course and this theorem is
called the Galois correspondence. 

\section{The Galois correspondence}

Let $L$ over $K$ be a \nameref{def:galoisextension}. By definition the
group automorphisms $Aut\left(L/K\right)$ is called
\nameref{def:galoisgroup} and denoted a $Gal\left(L/K\right)$
\begin{theorem}[Galois correspondence]
  \begin{enumerate}
  \item If $L$ is finite over $K$ then there is a
    \nameref{def:bijection} between sub-extension $F$
    ($K \subset F \subset L$) and subgroup $H \subset
    Gal\left(L/K\right)$. The correspondence is the following
    \begin{eqnarray}
      F \rightarrow Gal\left(L/F\right)
      \nonumber \\
      L^H \leftarrow H
      \nonumber
    \end{eqnarray}
  \item The following statement are equivalent (if and only if)
    \begin{enumerate}
    \item $F$ is Galois over $K$
    \item $\forall g \in Gal\left(L/K\right) g\left(F\right) = F$
    \item $Gal\left(L/F\right)$ is a \nameref{def:normalsubgroup} in
      $Gal\left(L/K\right)$ 
    \end{enumerate}
    In this case  $g$ goes to $g$ restricted to $F$: $g \to
    \left.g\right|_F$ this is injection (???)
    $Gal\left(L/K\right) \twoheadrightarrow Gal\left(L/F\right)$
  \end{enumerate}
  \begin{proof}
    \begin{enumerate}
    \item Most work have been done before. What have we got by now?
      $L^{Gal\left(L/F\right)} = F$ (see (\ref{eq:lec5_2})). By
      the theorem definition we have $H \subset Gal\left(L/K\right)$.
      \nameref{thm:artin} theorem gives us
      $\left[L:L^H\right] = \left|H\right|$ but with theorem
      \ref{thm:lec5_4} we also have
      $\left[L:L^H\right] = \left|Gal\left(L/L^H\right)\right|$ so one
      must have $H = Gal\left(L/L^H\right)$.

      This means that the maps that we have in the theorem :
      $F \rightarrow Gal\left(L/F\right)$ and
      $L^H \leftarrow H$ are mutually inverse
      \footnote{
        ???
      }
      and if a map is invert able it is \nameref{def:bijection}.
    \item We should proof equivalence of the following statements:
      \begin{enumerate}
      \item $F$ is Galois over $K$ \label{item:galoiscorrespondence1}
      \item $\forall g \in Gal\left(L/K\right) g\left(F\right) = F$
        \label{item:galoiscorrespondence2}
      \item $Gal\left(L/F\right) \triangleleft Gal\left(L/K\right)$
        \label{item:galoiscorrespondence3}
      \end{enumerate}      
    \end{enumerate}

    Lets show that \ref{item:galoiscorrespondence1} implies
    \ref{item:galoiscorrespondence2}. Fix $x \in F$ the minimal
    polynomial $P_{min}\left(x, K\right)$ splits in $L$ but it has a
    root in $F$ thus it should have all roots in $F$ by normality
    i.e. as soon as $F$ is \nameref{def:normalextension}
    $P_{min}\left(x, K\right)$ splits in $F$. This means, of course,
    that any map from Galois group preserves $F$ since it premutes the
    roots: $\forall g \in Gal\left(L/K\right)$ $g$ permutes the roots
    of $P_{min}\left(x, K\right)$ and that is the true for any $x \in
    F$ therefore $g\left(F\right) \subset F$ since $F$ is generated
    (consists of) such roots.

    Lets show that \ref{item:galoiscorrespondence2} implies
    \ref{item:galoiscorrespondence1}. If $g\left(F\right) \subset F$
    then all roots of $P_{min}\left(x, K\right)$, $x \in F$ are in $F$
    since $g$ permutes those roots or, in other words, since Galois
    group acts transitively (??? see theorem \ref{thm:lec5_3}) on
    roots of an irreducible polynomial therefore $F$ is normal by
    definition.

    Lets show that \ref{item:galoiscorrespondence1} and
    \ref{item:galoiscorrespondence2} are equivalent to
    \ref{item:galoiscorrespondence3} i.e. let $g \in G$,
    $g\left(F\right) \subset L$ then if
    $h \in Gal\left(L/F\right)$ is such that $\left.h\right|_F = id$
    then $\left.g h g^{-1}\right|_{g\left(F\right)} = id$. This means
    that $g h g^{-1} \in Gal\left(L/g\left(F\right)\right)$ so the
    statement $g\left(F\right) = F$ is the same to say
    \[
    g Gal\left(L/F\right) g^{-1} = Gal\left(L/F\right)
    \]
    So apply this to all $g \in Gal\left(L/K\right)$ one can get that
    $Gal\left(L/F\right)$ is a \nameref{def:normalsubgroup} of
    $Gal\left(L/K\right)$

    Finally if all this statements
    (\ref{item:galoiscorrespondence1}
    $\Longleftrightarrow$
    \ref{item:galoiscorrespondence2}
    $\Longleftrightarrow$
    \ref{item:galoiscorrespondence3}
    ) are true then we can consider map (make sense by
    \ref{item:galoiscorrespondence2}) 
    \[
    \phi: Gal\left(L/K\right)
    \xrightarrow[g \to \left.g\right|_F]{}
    Gal\left(L/F\right).
    \]
    This is a \nameref{def:surjection} by theorem \ref{thm:lec2_3}
    (???) and the kernel $\ker \phi = Gal\left(L/F\right)$ by
    definition because the kernel consists of things which are identity on $F$. 
  \end{proof}
  \label{thm:galoiscorrespondence}
\end{theorem}

\begin{remark}
  If $L$ over $K$ is not finite then Galois correspondence is not
  \nameref{def:bijection} i.e. the maps which are in the theorem still
  make sense, but they will not be mutually inverse bijections and we
  shall see an example (see section \ref{sec:lec6_finitefield}).
  \label{rem:lec6_gcnotbijection}
\end{remark}

\section{First examples (polynomials of degree 2 and 3)}

\begin{example}[Degree 2]
  Let $\left[L:K\right] = 2$ and $char K \ne 2$. The extension $L$ is
  generated by a root of 
  quadratic polynomial i.e. $x \in L \setminus K$ then
  $P_{min}\left(x, K\right)$ is quadratic and if we look at the
  formula for the root of the equation we will see that the extension
  is generated by a root of discriminant $\Delta$:
  $L = K\left(\sqrt{\Delta}\right)$.

  What can we say about the $Gal\left(L/K\right)$. It consists of 2
  elements and there is only one cyclic group of 2 elements:
  $\mathbb{Z}/2\mathbb{Z}$. Therefore
  \[
  Gal\left(L/K\right) \cong \mathbb{Z}/2\mathbb{Z}
  \].
  The elements of the group is identity and an element that exchanges
  the 2 roots i.e. permutes $\sqrt{\Delta}$ and $-\sqrt{\Delta}$.
\end{example}

\begin{example}[Degree 3]
    Let $\left[L:K\right] = 3$ and $char K \ne 3$ (separable
    extensions) then $L$ is generated by $x$ - root of 
    degree 3 polynomial $P$ and there are 2 cases
    \begin{enumerate}
    \item $P$ splits in $L$ therefore $L$ is a Galois group but the
      Galois group of 3 elements must be cyclic i.e.
      $Gal\left(L/K\right) \cong \mathbb{Z}/3\mathbb{Z}$ - cyclic
      group of order 3.
    \item $P$ does not split in $L$ then there exists
      $M = K\left(x_1, x_2, x_3\right)$ - splitting field where
      $x_{1,2,3}$ are roots of $P$ and $L = K\left(x_1\right)$. $M$ is
      Galois and the Galois group is embeded into a group of
      permutation of 3 elements (because Galois group permutes the roots):
      $Gal\left(M/K\right) \hookrightarrow S_3$.

      As soon $L \subsetneq M$ then $\left[M:K\right] > 3$ so
      $Gal\left(M/K\right) = S_3$. In particular
      $\left[M:K\right] = 6$
    \end{enumerate}

    If you see a cubic polynomial how will you decide is its Galois
    group is cyclic or $S_3$? This is determined by a
    \nameref{def:discriminant} of polynomial which is a subject of
    next section (see example \ref{ex:lec6_discriminant3degree}).

    \label{ex:lec6_degree3}
\end{example}

\section{Discriminant. Degree 3 (cont'd). Finite fields}

\subsection{Discriminant}

\begin{definition}[discriminant]
  Let $P \in K\left[X\right]$. The polynomial has the following roots
  in $\bar{K}$: $x_1, x_2, \dots, x_n$. The following product is
  called discriminant:
  \[
  \Delta = \prod_{i < j} \left(x_i - x_j\right)^2
  \]
  \label{def:discriminant}
\end{definition}

If we take group $G = Gal\left(P\right)$ then
$G \subset S_n$ and any permutation preserves $\Delta$ and as result
we have $\Delta \in K$ (see (\ref{eq:lec5_2})).

Lets take a root of discriminant (we have to choose some roots order
for this operation) then
\[
\sqrt{\Delta} = \prod_{i < j} \left(x_i - x_j\right)
\]
this quantity is preserved by even (and not by odd) permutations.

\begin{proposition}
  Let $G = Gal\left(P\right)$ - \nameref{def:galoisgroup} then $G
  \subset A_n$
  \footnote {
    $A_n$ is a group of even permutations
  }
  if and only if $\sqrt{\Delta} \in K$.
  \begin{proof}
    Since if the Galois group is even then, this will be preserved by
    an element of Galois group and so will be in $K$ and conversely, if
    it is an element of $K$, then it must be preserved by the Galois
    group, but we know it is preserved only by even permutations. 
  \end{proof}
  \label{prop:lec6_1}
\end{proposition}

If we return to our example \ref{ex:lec6_degree3} we can get the
following one
\begin{example}[Discriminant of polynomial degree 3]
  Lets compute the discriminant for the following polynomial:
  $X^3 + p X + q$.
  \footnote {
    $X^2$ element can be always hidden via a variable change. Thus the
    polynomial can be considered as a common case for cubic polynomials.
  }

  The discriminant easy to compute:
  \footnote{
    ??? compute it
  }
  \(
  \Delta = -4 p^3 - 27 q^2
  \).
  So if $\Delta$ is a square in $K$ then $Gal\left(P\right) \cong A_3$
  (cyclic of order 3). If not then $Gal\left(P\right) \cong S_3$ (non
  commutative group of 6 elements).

  What can we say about sub-extensions for the two cases? If $M$ is a
  splitting field of $P$ over $K$ then for the first case there is no
  any sub-extension. For the second case there are several
  sub-extensions (they are determined by sub-groups of the Galois
  group: $S_3$ for our case). Especially we have 3 sub-extension of
  degree 3: $K\left(x_1\right)$, $K\left(x_2\right)$ and
  $K\left(x_3\right)$ (fixed by non-normal sub-groups of order 2 -
  because $M$ is degree 2 over $K\left(x_{1,2,3}\right)$ -
  transpositions roots ???). And we have one quadratic sub-extension
  (of degree 2) fixed by $A_3 \subset S_3$ this is
  $K\left(\sqrt{\Delta}\right)$.

  \nameref{thm:galoiscorrespondence} says us that there are no other
  sub-extensions. Because those sub extensions correspond objectively
  to subgroups of the \nameref{def:galoisgroup}. And in this case, it
  does not have so many subgroups. These are just three subgroups of order 2
  generated by transpositions, and one subgroup of order 3 generated
  by a three cycle.  
  \label{ex:lec6_discriminant3degree}
\end{example}

\subsection{Finite fields}
\label{sec:lec6_finitefield}
We have seen that theory of finite fields is easy. Especially all
\nameref{def:galoisgroup}s are cyclic (see corollary
\ref{cor:lec3_2}). I.e. we have the field $\mathbb{F}_{q^n}$ over
$\mathbb{F}_{q}$. The Galois group is cyclic and generated by
Frobenius map (see remark \ref{rem:frobeniushomomorphism}) which is
$F_p: x \to x^q$.

More interesting are infinite extensions of a finite field, for
instance the \nameref{def:algebraicclosure}.
Thus consider $\bar{\mathbb{F}}_p$ as an extension of
$\mathbb{F}_p$. If we take an invariant generated by Frobenius $F_p$
\footnote {
  The group generated by one element $F_p$ is denoted as $\left<F_p\right>$.
}
then
\[
\bar{\mathbb{F}}_p^{\left<F_p\right>} = \mathbb{F}_p
\]
but
\[
Gal\left(\bar{\mathbb{F}}_p/\mathbb{F}_p\right) \ne
\left<F_p\right>
\]
therefore there is no bijective correspondence between sub-fields and
sub-groups. In particular the \nameref{thm:galoiscorrespondence} is
not \nameref{def:bijection} (as it was mentioned at remark
\ref{rem:lec6_gcnotbijection})

So how to see that the Galois group is not cyclic:
$Gal\left(\bar{\mathbb{F}}_p/\mathbb{F}_p\right) \ne
\left<F_p\right>$?

Really a smaller group is not cyclic. Lets look at the following:
\[
\mathbb{F}_p \subset \mathbb{F}_{p^2} \subset \dots
\mathbb{F}_{p^{2^n}} \subset \dots
\]
and let
\[
L = \cup \mathbb{F}_{p^{2^n}}
\]
We claim that $Gal\left(L/\mathbb{F}_p\right)$ is not cyclic. Consider
the following number $a_n = 1+ 2 + 4 + \dots + 2^n$ then $\forall x
\in \mathbb{F}_{p^{2^n}}$
$F_p^{a_n}\left(x\right) = F_p^{a_m}\left(x\right)$ for any $m >
n$. This is because the Frobenius map $F_p$ sends $x$ to $x^p$ is an
identity on $\mathbb{F}_p$ therefore $F_p^{2^{n+l}}$ is identity on
$\mathbb{F}_{p^{2^n}}$ $\forall l \ge 0$. This implies that
there exists an automorphism $\phi: L \to L$ such that $\forall n
\ge 0$
\[
\left.\phi\right|_{\mathbb{F}_{p^{2^n}}} = F^{a_n}
\]
but $\forall k \in \mathbb{Z}$ $F_p^k \ne \phi$. One can look at
$\phi$ as $\phi = F_p^{1+2+4+ \dots + 2^n + \dots}$ but this is, of
course, very informal. The rigorous conclusions we can draw from this
is that our Galois group is not a cyclical group generated by the
Frobenius map i.e.
$Gal\left(\bar{\mathbb{F}}_p/\mathbb{F}_p\right) \ne
\left<F_p\right>$. And also, that we don't have a bijective Galois
correspondents like we have for finite field extensions. 

\section{An infinite degree example. Roots of unity: cyclotomic polynomials}
\section{Irreducibility of cyclotomic polynomial.The Galois group}
