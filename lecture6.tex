%% -*- coding:utf-8 -*-
\chapter{Galois correspondence and first examples}
We state and prove the main theorem of these lectures: the Galois
correspondence. Then we start doing examples (low degree,
discriminant, finite fields, roots of unity).

\section{Some further remarks on normal extension. Fixed field}

Some definitions from previous lecture. $L$ over $K$ is
\nameref{def:galoisextension} if and only if it is a
\nameref{def:separableextension} and \nameref{def:normalextension} or
in other words $L$ is a \nameref{def:splittingfield} of a family of
separable irreducible polynomial over $K$. We also seen (see theorem
\ref{thm:lec5_4}) that in the case of finite extension
$\left[L:K\right] < \infty$ the number of automorphisms
$\left|Aut\left(L/K\right)\right| = \left[L:K\right]$.

There are several remarks on \nameref{def:normalextension}s which show
that the extensions behave sometimes differently compare to other
types of extensions. Especially we have seen for that an extension 
$L$ over $M$ over $K$ was finite or algebraic or separable or purely
inseparable if, and only if, it was true for $L$ over $M$ and $M$ over
$K$. So, for a normal extensions, this is not the case anymore.

\begin{remark}
  Let we have a tower of extensions $K \subset L \subset M$. If $M$ is
  normal over $K$ then of course the $M$ is normal over $L$. It is
  clear because if $M$ is a splitting field of a family of polynomials
  over $K$ the one can just consider them as being polynomials over
  $L$ and say that $M$ is a splitting field of ephemeral polynomials
  over $L$.

  But $L$ does not have to be normal over $K$ (see example
  \ref{ex:lec6_1}). 
\end{remark}

\begin{example}
  Consider
  \[
  \mathbb{Q} \subset
  \mathbb{Q}\left(\sqrt[4]{2}\right)
  \subset
  \mathbb{Q}\left(\sqrt[4]{2}, i\right)
  \]

  We have $\mathbb{Q}\left(\sqrt[4]{2}, i\right)$ to be a splitting
  field for polynomial $X^4 -2$ but
  $\mathbb{Q}\left(\sqrt[4]{2}\right)$ is just a
  \nameref{def:stemfield} (not \nameref{def:splittingfield}) for this
  polynomial. And as result $\mathbb{Q}\left(\sqrt[4]{2}\right)$ is
  not a normal over $\mathbb{Q}$.  
  \label{ex:lec6_1}
\end{example}

\begin{remark}
  A quadratic extension is normal. This is by formula for roots of a
  quadratic equation.
  \footnote {
    ??? Extensions with degree 2. 
  }

  If $P$ quadratic over $K$ has 1 root in $L$ then its another root is
  also in $L$.
\end{remark}

\begin{remark}
  One often has $K \subset L \subset M$ with $L$ normal over $K$, $M$
  normal over $L$ but $M$ not normal over $K$ (see example
  \ref{ex:lec6_2}). 
\end{remark}

\begin{example}
  Consider
  \[
  \mathbb{Q} \subset
  \mathbb{Q}\left(\sqrt{2}\right)
  \subset
  \mathbb{Q}\left(\sqrt[4]{2}\right)
  \]

  We have $\mathbb{Q}\left(\sqrt{2}\right)$ normal over $\mathbb{Q}$
  as well as $\mathbb{Q}\left(\sqrt[4]{2}\right)$ normal over
  $\mathbb{Q}\left(\sqrt{2}\right)$ because they both are quadratic
  extensions. But 
  $\mathbb{Q}\left(\sqrt[4]{2}\right)$ is not normal over $\mathbb{Q}$
  (as it was mentioned in example \ref{ex:lec6_2})
  \label{ex:lec6_2}
\end{example}

We also seen at last lecture the following definition:
\begin{definition}[Fixed field]
  If $L$ is a field and $G \subset Aut\left(L\right)$ then
  \[
  L^G = \left\{
  x \in L \mid \forall g \in G: g x = x 
  \right\}
  \]  
  is a fixed field
  \label{def:fixedfield}
\end{definition}

If we have a sub-field $K \subset L$ then we can consider the
following group of automorphisms of $L$ over $K$:
$Aut\left(L/K\right)$ in the case if $L$ is normal.
Because otherwise the group will be too small to give information
about $L$. But in the normal case it makes sense to consider  
the group of automorphisms of $L$ over $K$.

We have seen (see (\ref{eq:lec5_2})) that if $L$ is separable over $K$
then 
\[
L^{Aut\left(L/K\right)} = K
\]
This is because of the group of automorphisms was permuting the roots
over the minimal polynomial of $x$ over $K$. So, if it was fix and $x$
was on it, was meaning that $x$ was the only root over its minimal
polynomial. So this was meaning that $x$ was purely inseparable over
$K$.  

We also have seen (see theorem \ref{thm:artin}) that if $G$ is finite
the $L$ is \nameref{def:galoisextension} over $L^G$ and
$\left[L:L^G\right] = \left|G\right|$.

And now we are going to summarize all these in a theorem which is in
fact the main subject of this lecture course and this theorem is
called the Galois correspondence. 

\section{The Galois correspondence}

Let $L$ over $K$ be a \nameref{def:galoisextension}. By definition the
group automorphisms $Aut\left(L/K\right)$ is called
\nameref{def:galoisgroup} and denoted a $Gal\left(L/K\right)$
\begin{theorem}[Galois correspondence]
  \begin{enumerate}
  \item If $L$ is finite over $K$ then there is a
    \nameref{def:bijection} between sub-extension $F$
    ($K \subset F \subset L$) and subgroup $H \subset
    Gal\left(L/K\right)$. The correspondence is the following
    \begin{eqnarray}
      F \rightarrow Gal\left(L/F\right)
      \nonumber \\
      L^H \leftarrow H
      \nonumber
    \end{eqnarray}
  \item The following statement are equivalent (if and only if)
    \begin{enumerate}
    \item $F$ is Galois over $K$
    \item $\forall g \in Gal\left(L/K\right) g\left(F\right) = F$
    \item $Gal\left(L/F\right)$ is a \nameref{def:normalsubgroup} in
      $Gal\left(L/K\right)$ 
    \end{enumerate}
    In this case  $g$ goes to $g$ restricted to $F$: $g \to
    \left.g\right|_F$ this is injection (???)
    $Gal\left(L/K\right) \twoheadrightarrow Gal\left(L/F\right)$
  \end{enumerate}
  \begin{proof}
    
  \end{proof}
  \label{thm:galoiscorrespondence}
\end{theorem}

\section{Galois correspondence (cont'd). First examples (polynomials
  of degree 2 and 3)} 
\section{Discriminant. Degree 3 (cont'd). Finite fields}
\section{An infinite degree example. Roots of unity: cyclotomic polynomials}
\section{Irreducibility of cyclotomic polynomial.The Galois group}
